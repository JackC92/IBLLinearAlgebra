%%
%% All packages and macros needed for the problemsets
%%

\usepackage{amsmath}

\usepackage{lipsum}
%\usepackage{showframe}
%\usepackage{layout}


\usepackage[charter,cal=cmcal]{mathdesign} %different font
%\usepackage{avant}

\usepackage{microtype}
\usepackage{mathtools}
\usepackage{etoolbox}
%\usepackage{amsfonts}
%\usepackage{amssymb}
\usepackage{graphicx}
\graphicspath{{images/}}
\usepackage[inline]{enumitem}
\usepackage{xparse}
\usepackage{ifthen}
\usepackage{caption}
\usepackage{subcaption}
\usepackage{color}
\usepackage{tikz}
	\usetikzlibrary{fit}
	\usetikzlibrary{fadings}
	\usetikzlibrary{calc}
	\tikzset{>=latex}
	\usetikzlibrary{cd}
	\usetikzlibrary{spy}
\usepackage{fancyhdr}
\usepackage{calc}
\usepackage{wrapfig}
\usepackage{marginnote}
\usepackage{mparhack}
\usepackage{marginfix}
\usepackage{indextools}
\usepackage[open=false]{bookmark}  % render the pdf TOC in the proper order

%\usepackage[
%  linktocpage=false,      % no page numbers are clickable
%  colorlinks=false,       % no color
%  breaklinks=true,        % break URLs
%  bookmarks,              % creates bookmarks in pdf
%  hyperfootnotes=true,    % clickable footnotes
%  pdfborder={0 0 0},      % for removing borders around links
%  bookmarksnumbered=true, % If Acrobat bookmarks are requested, include section numbers.
%  bookmarksopen=false,    % If Acrobat bookmarks are requested, show them with all the subtrees expanded.
%  %hidelinks=true,
%  %linkcolor=blue,
%  %citecolor=blue,
%  %urlcolor=blue,
%  pdfpagemode={UseOutlines}, % show pdf bookmarks (indices) on startup; does not function all the time
%  pdftitle={...}, % title
%  pdfauthor={...}, % author
%  pdfkeywords={...}, % subject of the document
%  pdfsubject={...}, % list of keywords
%  pdfmenubar=true,        % make PDF viewer’s menu bar visible
%  pdfpagelabels,
%]{hyperref}
%\usepackage[hidelinks,]{hyperref}
\usepackage{fnpct} % fancy footnote spacing
\usepackage{bm}
\usepackage{systeme}
\usepackage{datatool}% http://ctan.org/pkg/datatool for sorted lists
\usepackage{xspace}



\usepackage{pgfplots}
\pgfplotsset{compat=newest}
	\usepgfplotslibrary{fillbetween}
%%%
% Useful Linear Algebra macros
%%%
\newcommand{\declarecommand}[1]{\providecommand{#1}{}\renewcommand{#1}}
\declarecommand{\R}{\mathbb{R}}  % we don't care if it's already defined.  We really want *this* command!
\declarecommand{\Z}{\mathbb{Z}}
\declarecommand{\Q}{\mathbb{Q}}
\declarecommand{\N}{\mathbb{N}}
\declarecommand{\C}{\mathbb{C}}
\declarecommand{\d}{\mathrm{d}}
\declarecommand{\dd}{\mathbbm{d}} % exterior derivative
\DeclareMathOperator{\Span}{span}
\DeclareMathOperator{\Img}{img}
\DeclareMathOperator{\Id}{id}
\DeclareMathOperator{\Ident}{\Id}
\DeclareMathOperator{\Vol}{Vol}
\DeclareMathOperator{\VolChange}{Vol\hspace{1.5pt}Change}
\DeclareMathOperator{\Range}{range}
\DeclareMathOperator{\Rref}{rref}
\DeclareMathOperator{\Rank}{rank}
\DeclareMathOperator{\Comp}{\Vcomp}
\DeclareMathOperator{\Vcomp}{v\hspace{1pt}comp}
\DeclareMathOperator{\Null}{null}
\DeclareMathOperator{\Nullity}{nullity}
\DeclareMathOperator{\Char}{char}
\DeclareMathOperator{\Proj}{proj}
\DeclareMathOperator{\Flux}{Flux}
\DeclareMathOperator{\Circ}{Circ}
\DeclareMathOperator{\chr}{char}
\DeclareMathOperator{\Dim}{dim}
\DeclareMathOperator{\Perp}{perp}
\DeclareMathOperator{\Ker}{kernel}
\DeclareMathOperator{\Row}{row}
\DeclareMathOperator{\Col}{col}
\DeclareMathOperator{\Rep}{Rep}
\newcommand{\BasisChange}[2]{[#2\!\leftarrow\!#1]}
\newcommand{\proj}{\Proj}
\newcommand{\rref}{\Rref}
\newcommand{\xhat}{{\vec e_1}}
\newcommand{\yhat}{{\vec e_2}}
\newcommand{\zhat}{{\vec e_3}}
\newcommand{\sbasis}[1]{\vec { e}_{#1}}
\newcommand{\mat}[1]{\begin{bmatrix*}[r]#1\end{bmatrix*}}
\newcommand{\matc}[1]{\begin{bmatrix}#1\end{bmatrix}}
\newcommand{\formarg}[2]{\big(#1;\, #2\big)}
\DeclarePairedDelimiter\abs{\lvert}{\rvert}
\DeclarePairedDelimiter\Abs{\lvert}{\rvert}
\DeclarePairedDelimiter\norm{\lVert}{\rVert}
\newcommand{\Norm}[1]{\norm{#1}}
% just to make sure it exists
\providecommand\given{}
% can be useful to refer to this outside \Set
\newcommand\SetSymbol[1][]{%
	\nonscript\::%
	\allowbreak
	\nonscript\:
	\mathopen{}}
\DeclarePairedDelimiterX\Set[1]\{\}{%
	\renewcommand\given{\SetSymbol[\delimsize]}
	#1
}

\newcommand{\scaledgrid}[1]{%
	\begin{tikzpicture}[scale=#1]
		\draw[thin, white!20!black, dotted] (-4.1,-4.1) grid (4.1,4.1);
		\draw[ <->] (-4.3,0) -- (4.3,0);
		\draw[ <->] (0,-4.3) -- (0,4.3);
	\end{tikzpicture}
}
\newcommand{\scaledshortgrid}[1]{%
	\begin{tikzpicture}[scale=#1]
		\draw[thin, white!20!black, dotted] (-4.1,-2.1) grid (4.1,2.1);
		\draw[ <->] (-4.3,0) -- (4.3,0);
		\draw[ <->] (0,-2.3) -- (0,2.3);
	\end{tikzpicture}
}
\newcommand{\singlegrid}{\scaledgrid{1}}
\newcommand{\doublegrid}{\mbox{\scaledgrid{.9}\scaledgrid{.9}}\par}
\newcommand{\triplegrid}{\mbox{\scaledgrid{.6}\scaledgrid{.6}\scaledgrid{.6}}\par}

% labels for source attributions
\NewDocumentCommand{\beezer}{o}{%
	\IfNoValueTF{#1}{%
		{\color{blue}\sffamily{B}}%
	}{%
		{\color{blue}\sffamily{B}}%  XXX Todo, make this href to the appropriate problem number
	}\xspace%
}
\NewDocumentCommand{\hefferon}{o}{%
	\IfNoValueTF{#1}{%
		{\color{blue}\sffamily{H}}%
	}{%
		{\color{blue}\sffamily{H}}%  XXX Todo, make this href to the appropriate problem number
	}\xspace%
}



% Dummy, voidable environments
\DeclareDocumentEnvironment{bookonly}{o}{}{}
\DeclareDocumentEnvironment{displayonly}{o}{}{}
