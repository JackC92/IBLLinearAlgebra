\begin{exercises}
	\begin{problist}
		\prob Find the complete solution for the following systems.
		\begin{enumerate}
			\item $\systeme[xyzw]{4x+6y+3z-10w=6,5x+2y+z-7w=2,-6x+2y+z+4w=2}$
			\item $\systeme[xyzw]{2x+2y+z=-1,y-4z+2w=3,x-y-3z-4w=5}$
			\item $\systeme{x+y-2z=-5,-4x+y+5z=3}$
			\item $\systeme{3x-2y=-4,x+y+3z=3,-4x+y-3z=1}$
			\item $\systeme{x-y+2z=-1,2x+y+4z=1,3x-4y+3z=-2}$
			\item $\systeme{2x+z=8,x+y+z=4,x+3y+2z=4,3x+2y+4z=9}$
		\end{enumerate}
		\begin{solution}
			\begin{enumerate}
				\item 
				Let $X$ be the corresponding augmented matrix
				\[
					X=
					\begin{bmatrix}[cccc|c]
						4 & 6 & 3 & -10 & 6\\
						5 & 2 & 1 & -7 & 2\\
						-6 & 2 & 1 & 4 & 2
					\end{bmatrix}.
				\]
				
				By row reduction,
				\[
					\Rref(X)=
					\begin{bmatrix}[cccc|c]
						1 & 0 & 0 & -1 & 0\\
						0 & 1 & 1/2 & -1 & 1\\
						0 & 0 & 0 & 0 & 0
					\end{bmatrix}.
				\]
				
				The third and fourth column of $\Rref(X)$ are free variable columns, so we introduce the arbitrary equations $z=t$ and $w=s$, and solve the system in terms of $t$ and $s$:
				\[
					\systeme[xyzw]{x-w=0,y+1/2z-w=1,z=t,w=s}.
				\]
				
				Written in vector form, the complete solution is
				\[
					\mat{x\\y\\z\\w} = \matc{s\\1-1/2t+s\\t\\s} = t\mat{0\\-1/2\\1\\0}+s\mat{1\\1\\0\\1}+\mat{0\\1\\0\\0}.
				\]
				\item 
				Let $X$ be the corresponding augmented matrix
				\[
					X=
					\begin{bmatrix}[cccc|c]
						2 & 2 & 1 & 0 & -1\\
						0 & 1 & -4 & 2 & 3\\
						1 & -1 & -3 & -4 & 5
					\end{bmatrix}.
				\]
				
				By row reduction,
				\[
					\Rref(X)=
					\begin{bmatrix}[cccc|c]
						1 & 0 & 0 & -2 & 1\\
						0 & 1 & 0 & 2 & -1\\
						0 & 0 & 1 & 0 & -1
					\end{bmatrix}.
				\]
				
				The fourth column of $\Rref(X)$ is a free variable column, so we introduce the arbitrary equation $w=t$, and solve the system in terms of $t$:
				\[
					\systeme[xyzw]{x-2w=1,y+2w=-1,z=-1,w=t}.
				\]
				
				Written in vector form, the complete solution is
				\[
				\mat{x\\y\\z\\w} = \matc{1+2t\\-1-2t\\-1\\t} = t\mat{2\\-2\\0\\1}+\mat{1\\-1\\-1\\0}.
				\]
				\item 
				Let $X$ be the corresponding augmented matrix
				\[
					X=
					\begin{bmatrix}[ccc|c]
						1 & 1 & -2 & -5\\
						-4 & 1 & 5 & 3
					\end{bmatrix}.
				\]
				
				By row reduction,
				\[
					\Rref(X)=
					\begin{bmatrix}[ccc|c]
						1 & 0 & -7/5 & -8/5\\
						0 & 1 & -3/5 & -17/5
					\end{bmatrix}.
				\]
				
				The third column of $\Rref(X)$ is a free variable column, so we introduce the arbitrary equation $z=t$, and solve the system in terms of $t$:
				\[
					\systeme{x-7/5z=-8/5,y-3/5z=-17/5,z=t}.
				\]
				
				Written in vector form, the complete solution is
				\[
					\mat{x\\y\\z} = \matc{-8/5+7/5t\\-17/5+3/5t\\t} = t\mat{7/5\\3/5\\1}+\mat{-8/5\\-17/5\\0}.
				\]
				\item 
				Let $X$ be the corresponding augmented matrix
				\[
					X=
					\begin{bmatrix}[ccc|c]
						3 & -2 & 0 & -4\\
						1 & 1 & 3 & 3\\
						-4 & 1 & -3 & 1
					\end{bmatrix}.
				\]
				
				By row reduction,
				\[
					\Rref(X)=
					\begin{bmatrix}[ccc|c]
						1 & 0 & 6/5 & 2/5\\
						0 & 1 & 9/5 & 13/5\\
						0 & 0 & 0 & 0
					\end{bmatrix}.
				\]
				
				The third column of $\Rref(X)$ is a free variable column, so we introduce the arbitrary equation $z=t$, and solve the system in terms of $t$:
				\[
					\systeme{x+6/5z=2/5,y+9/5z=13/5,z=t}.
				\]
				
				Written in vector form, the complete solution is
				\[
					\mat{x\\y\\z} = \matc{2/5-6/5t\\13/5-9/5t\\t} = t\mat{-6/5\\-9/5\\1}+\mat{2/5\\13/5\\0}.
				\]
				\item 
				Let $X$ be the corresponding augmented matrix
				\[
					X=
					\begin{bmatrix}[ccc|c]
						1 & -1 & 2 & -1\\
						2 & 1 & 4 & 1\\
						3 & -4 & 3 & -2
					\end{bmatrix}.
				\]
				
				By row reduction,
				\[
					\Rref(X)=
					\begin{bmatrix}[ccc|c]
						1 & 0 & 0 & 4/3\\
						0 & 1 & 0 & 1\\
						0 & 0 & 1 & -2/3
					\end{bmatrix}.
				\]
				
				Written in vector form, the complete solution is
				\[
					\mat{x\\y\\z} = \mat{4/3\\1\\-2/3}.
				\]
				\item 
				Let $X$ be the corresponding augmented matrix
				\[
					X=
					\begin{bmatrix}[ccc|c]
						2 & 0 & 1 & 8\\
						1 & 1 & 1 & 4\\
						1 & 3 & 2 & 4\\
						3 & 2 & 4 & 9
					\end{bmatrix}.
				\]
				
				By row reduction,
				\[
					\Rref(X)=
					\begin{bmatrix}[ccc|c]
						1 & 0 & 0 & 5\\
						0 & 1 & 0 & 1\\
						0 & 0 & 1 & -2\\
						0 & 0 & 0 & 0
					\end{bmatrix}.
				\]
				
				Written in vector form, the complete solution is
				\[
				\mat{x\\y\\z} = \mat{5\\1\\-2}.
				\]
			\end{enumerate}
		\end{solution}
		\prob 
		\begin{enumerate}
			\item Let $\vec v_1=\mat{-4\\0\\1\\7}$, $\vec v_2=\mat{-2\\8\\0\\3}$ and $\vec v_3=\mat{-2\\7\\3\\4}$. Set up and solve a system of linear equations whose solution will determine if the vectors $\vec v_1$, $\vec v_2$ and $\vec v_3$ are linearly independent.
			\item Let $\ell_1$ and $\ell_2$ be described in vector form by
			\[
			\overbrace{\vec x=t\mat{1\\3}+\mat{1\\1}}^{\displaystyle \ell_1}
			\quad
			\overbrace{\vec x=t\mat{2\\1}+\mat{3\\4}}^{\displaystyle \ell_2}.
			\]
			Set up and solve a system of linear equations whose solution will determine if the lines $\ell_1$ and $\ell_2$ intersect.
			\item Let $\vec v_1=\mat{1\\2\\3}$, $\vec v_2=\mat{-2\\1\\0}$ and $\vec v_3=\mat{2\\7\\1}$. Set up and solve a system of linear equations whose solution will determine if the vectors $\vec v_1$, $\vec v_2$ and $\vec v_3$ span $\R^3$.
		\end{enumerate}
		\begin{solution}
			\begin{enumerate}
				\item The vectors $\vec v_1$, $\vec v_2$ and $\vec v_3$ are linearly independent when there does not exist a non-trivial solution to
				\[
					x\vec v_1+y\vec v_2+z\vec v_3=\vec 0.
				\]
				
				The vector equation is equivalent to the system of linear equations
				\[
					\systeme{-4x-2y-2z=0,8y+7z=0,x+3z=0,7x+3y+4z=0}.
				\]
				
				The solution to this system is
				\[
					\mat{x\\y\\z}=\mat{0\\0\\0}.
				\]
				
				Since the only solution is the trivial solution, the vectors $\vec v_1$, $\vec v_2$ and $\vec v_3$ are linearly independent.
				\item The lines $\ell_1$ and $\ell_2$ intersect when the components of $\vec x$ are equal. Equating the components of $\vec x$ gives the equations
				\[
					\systeme{t+1=2s+3,3t+1=s+4}.
				\]
				
				The solution to this system is
				\[
					\mat{t\\s}=\mat{4/5\\-3/5}.
				\]
				
				Since $\vec x=\mat{9/5\\17/5}$ when $t=4/5$, the intersection of $\ell_1$ and $\ell_2$ is the point $\mat{9/5\\17/5}$.
				\item 
				Since $\vec v_1, \vec v_2, \vec v_3\in\R^3$, they span $\R^3$ if they are linearly independent. We need to determine if there is a non-trivial solution to
				\[
					x\vec v_1+y\vec v_2+z\vec v_3=\vec 0.
				\]
				
				This is equivalent to the system of linear equations
				\[
					\systeme{x-2y+2z=0,2x+y+7z=0,3x+z=0}.
				\]
				
				The solution to this system is
				\[
				\mat{x\\y\\z}=\mat{0\\0\\0}.
				\]
				
				Since the only solution is the trivial solution, the vectors $\vec v_1$, $\vec v_2$ and $\vec v_3$ are linearly independent.
			\end{enumerate}
		\end{solution}
		\prob Presented below are other students' work for the previous problem. Evaluate whether their reasoning correctly answer the question, and point out the flaws in their reasoning, if any.
		\begin{enumerate}
			\item 
			The following answer is for part (a):
			
			The vectors $\vec v_1$, $\vec v_2$ and $\vec v_3$ are linearly independent when there does not exist a non-trivial solution to
			\[
				x\vec v_1+y\vec v_2+z\vec v_3=\vec 0.
			\]
			\item 
			The following answer is for part (b):
			
			The lines $\ell_1$ and $\ell_2$ intersect when the components of $\vec x$ are equal. Equating the components of $\vec x$ gives the equations
			\[
				\systeme{t+1=2t+3,3t+1=t+4}.
			\]
			
			This system is equivalent to
			\[
				\systeme{t=-2,2t=3}.
			\]
			
			Since this system is inconsistent, there does not exist a solution to this system, so the lines $\ell_1$ and $\ell_2$ do not intersect.
			\item 
			The following answer is for part (b):
			
			The lines $\ell_1$ and $\ell_2$ intersect when the components of $\vec x$ are equal. Equating the components of $\vec x$ gives the equations
			\[
				\systeme{t+1=2s+3,3t+1=s+4}.
			\]
			
			This system is equivalent to
			\[
				\systeme[ts]{t-2s=2,3t-s=3}.
			\]
			
			The solution to this system is
			\[
				\mat{t\\s}=\mat{4/5\\-3/5},
			\]
			so the lines $\ell_1$ and $\ell_2$ intersect at $\mat{4/5\\3/5}$.
		\end{enumerate}
		\begin{solution}
			\begin{enumerate}
				\item 
				\item The reasoning is incorrect because the student failed to set the parameter variables to different letters when equating the components of $\vec x$.
				
				A valid system of linear equations is
				\[
					\systeme{t+1=2s+3,3t+1=s+4}.
				\]
				\item The reasoning is incorrect because the solution $\mat{4/5\\-3/5}$ is the value of $t$ and $s$ at the intersection. To find the intersection of $\ell_1$ and $\ell_2$, the value $t=4/5$ or $s=-3/5$ needs to be plugged into the vector form of $\ell_1$ or $\ell_2$.
			\end{enumerate}
		\end{solution}
	\end{problist}
\end{exercises}
