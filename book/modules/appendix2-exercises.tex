\begin{exercises}
	\begin{problist}
		\prob Find the complete solution to the following systems.
		\begin{enumerate}
			\item $\systeme[xyzw]{4x+6y+3z-10w=6,5x+2y+z-7w=2,-6x+2y+z+4w=2}$
			\item $\systeme[xyzw]{2x+2y+z=-1,y-4z+2w=3,x-y-3z-4w=5}$
			\item $\systeme{x+y-2z=-5,-4x+y+5z=3}$
			\item $\systeme{3x-2y=-4,x+y+3z=3,-4x+y-3z=1}$
			\item $\systeme{x-y+2z=-1,2x+y+4z=1,3x-4y+3z=-2}$
			\item $\systeme{2x+z=8,x+y+z=4,x+3y+2z=4,3x+2y+4z=9}$
		\end{enumerate}
		\begin{solution}
			\begin{enumerate}
				\item 
				Let
				\[
					X=
					\begin{bmatrix}[cccc|c]
						4 & 6 & 3 & -10 & 6\\
						5 & 2 & 1 & -7 & 2\\
						-6 & 2 & 1 & 4 & 2
					\end{bmatrix}
				\]
				be the augmented matrix corresponding to the system.
				
				By row reduction,
				\[
					\Rref(X)=
					\begin{bmatrix}[cccc|c]
						1 & 0 & 0 & -1 & 0\\
						0 & 1 & 1/2 & -1 & 1\\
						0 & 0 & 0 & 0 & 0
					\end{bmatrix}.
				\]
				
				The third and fourth column of $\Rref(X)$ are free variable columns,
				so we introduce the arbitrary equations $z=t$ and $w=s$, and
				solve the system in terms of $t$ and $s$:
				\[
					\systeme[xyzw]{x-w=0,y+1/2z-w=1,z=t,w=s}.
				\]
				
				Written in vector form, the complete solution is
				\[
					\mat{x\\y\\z\\w} = \matc{s\\1-1/2t+s\\t\\s}
					=t\mat{0\\-1/2\\1\\0}+s\mat{1\\1\\0\\1}+\mat{0\\1\\0\\0}.
				\]
				\item 
				Let
				\[
					X=
					\begin{bmatrix}[cccc|c]
						2 & 2 & 1 & 0 & -1\\
						0 & 1 & -4 & 2 & 3\\
						1 & -1 & -3 & -4 & 5
					\end{bmatrix}
				\]
				be the augmented matrix corresponding to the system.
				
				By row reduction,
				\[
					\Rref(X)=
					\begin{bmatrix}[cccc|c]
						1 & 0 & 0 & -2 & 1\\
						0 & 1 & 0 & 2 & -1\\
						0 & 0 & 1 & 0 & -1
					\end{bmatrix}.
				\]
				
				The fourth column of $\Rref(X)$ is a free variable column,
				so we introduce the arbitrary equation $w=t$, and solve the system
				in terms of $t$:
				\[
					\systeme[xyzw]{x-2w=1,y+2w=-1,z=-1,w=t}.
				\]
				
				Written in vector form, the complete solution is
				\[
				\mat{x\\y\\z\\w} = \matc{1+2t\\-1-2t\\-1\\t} = t\mat{2\\-2\\0\\1}+\mat{1\\-1\\-1\\0}.
				\]
				\item 
				Let
				\[
					X=
					\begin{bmatrix}[ccc|c]
						1 & 1 & -2 & -5\\
						-4 & 1 & 5 & 3
					\end{bmatrix}
				\]
				be the augmented matrix corresponding to the system.
				
				By row reduction,
				\[
					\Rref(X)=
					\begin{bmatrix}[ccc|c]
						1 & 0 & -7/5 & -8/5\\
						0 & 1 & -3/5 & -17/5
					\end{bmatrix}.
				\]
				
				The third column of $\Rref(X)$ is a free variable column, so
				we introduce the arbitrary equation $z=t$, and solve the system
				in terms of $t$:
				\[
					\systeme{x-7/5z=-8/5,y-3/5z=-17/5,z=t}.
				\]
				
				Written in vector form, the complete solution is
				\[
					\mat{x\\y\\z} = \matc{-8/5+7/5t\\-17/5+3/5t\\t} = t\mat{7/5\\3/5\\1}+\mat{-8/5\\-17/5\\0}.
				\]
				\item 
				Let
				\[
					X=
					\begin{bmatrix}[ccc|c]
						3 & -2 & 0 & -4\\
						1 & 1 & 3 & 3\\
						-4 & 1 & -3 & 1
					\end{bmatrix}
				\]
				be the augmented matrix corresponding to the system.
				
				By row reduction,
				\[
					\Rref(X)=
					\begin{bmatrix}[ccc|c]
						1 & 0 & 6/5 & 2/5\\
						0 & 1 & 9/5 & 13/5\\
						0 & 0 & 0 & 0
					\end{bmatrix}.
				\]
				
				The third column of $\Rref(X)$ is a free variable column, so
				we introduce the arbitrary equation $z=t$, and solve the system
				in terms of $t$:
				\[
					\systeme{x+6/5z=2/5,y+9/5z=13/5,z=t}.
				\]
				
				Written in vector form, the complete solution is
				\[
					\mat{x\\y\\z} = \matc{2/5-6/5t\\13/5-9/5t\\t} = t\mat{-6/5\\-9/5\\1}+\mat{2/5\\13/5\\0}.
				\]
				\item 
				Let
				\[
					X=
					\begin{bmatrix}[ccc|c]
						1 & -1 & 2 & -1\\
						2 & 1 & 4 & 1\\
						3 & -4 & 3 & -2
					\end{bmatrix}
				\]
				be the augmented matrix corresponding to the system.
				
				By row reduction,
				\[
					\Rref(X)=
					\begin{bmatrix}[ccc|c]
						1 & 0 & 0 & 4/3\\
						0 & 1 & 0 & 1\\
						0 & 0 & 1 & -2/3
					\end{bmatrix}.
				\]
				
				Written in vector form, the complete solution is
				\[
					\mat{x\\y\\z} = \mat{4/3\\1\\-2/3}.
				\]
				\item 
				Let
				\[
					X=
					\begin{bmatrix}[ccc|c]
						2 & 0 & 1 & 8\\
						1 & 1 & 1 & 4\\
						1 & 3 & 2 & 4\\
						3 & 2 & 4 & 9
					\end{bmatrix}
				\]
				be the augmented matrix corresponding to the system.
				
				By row reduction,
				\[
					\Rref(X)=
					\begin{bmatrix}[ccc|c]
						1 & 0 & 0 & 5\\
						0 & 1 & 0 & 1\\
						0 & 0 & 1 & -2\\
						0 & 0 & 0 & 0
					\end{bmatrix}.
				\]
				
				Written in vector form, the complete solution is
				\[
					\mat{x\\y\\z} = \mat{5\\1\\-2}.
				\]
			\end{enumerate}
		\end{solution}
		\prob 
		\begin{enumerate}
			\item Let $\vec v_{1}=\mat{1\\1\\-2\\4}$,
			$\vec v_{2}=\mat{1\\4\\0\\2}$ and
			$\vec v_{3}=\mat{-2\\-2\\4\\-8}$.
			
			Set up and solve a system of linear equations whose solution
			will determine if the vectors $\vec v_{1}$, $\vec v_{2}$ and $\vec
			v_{3}$ are linearly independent.
			
			\item Let $\vec v_{1}=\mat{1\\2\\3}$, $\vec v_{2}=\mat{-2\\1\\0}$
			and $\vec v_{3}=\mat{2\\7\\1}$.
			
			Set up and solve a system of linear equations whose solution
			will determine if the vectors $\vec v_{1}$, $\vec v_{2}$ and $\vec
			v_{3}$ span $\R^{3}$.
			\item Let $\ell_1$ and $\ell_2$ be described in vector form by
			\[
				\overbrace{\vec x=t\mat{1\\3}+\mat{1\\1}}^{\displaystyle \ell_1}
				\quad
				\overbrace{\vec x=t\mat{2\\1}+\mat{3\\4}}^{\displaystyle \ell_2}.
			\]
			Set up and solve a system of linear equations whose solution will
			determine if the lines $\ell_{1}$ and $\ell_{2}$ intersect.
			\item Let $\mathcal{P}_{1}$, $\mathcal{P}_{2}$ and $\mathcal{P}_{3}$
			be described in vector form by
			\[
				\begin{aligned}
					&\mathcal{P}_1: \vec x=t\mat{1\\-1\\0}+s\mat{-1\\-1\\2},\\
					&\mathcal{P}_2: \vec x=t\mat{1\\-1\\1}+s\mat{-1\\3\\-2}+\mat{0\\1\\-1},\\
					&\mathcal{P}_3: \vec x=t\mat{1\\0\\1}+s\mat{0\\-3\\1}+\mat{-1\\4\\-3}.
				\end{aligned}
			\]
			Set up and solve a system of linear equations whose solution
			will determine if the planes $\mathcal{P}_{1}$ and
			$\mathcal{P}_{2}$ intersect.
		\end{enumerate}
		\begin{solution}
			\begin{enumerate}
				\item The vectors $\vec v_{1}$, $\vec v_{2}$ and $\vec v_{3}$
				are linearly independent if there does not exist a non-trivial
				solution to
				\[
					x\vec v_1+y\vec v_2+z\vec v_3=\vec 0.
				\]
				
				The vector equation is equivalent to the system of linear equations
				\[
					\systeme{x+y-2z=0,x+4y-2z=0,-2x+4z=0,4x+2y-8z=0}.
				\]
				
				The complete solution to this system is
				\[
					\mat{x\\y\\z}=t\mat{2\\0\\1}.
				\]
				
				Since $(x, y, z)=(2, 0, 1)$ is a non-trivial solution to this
				system, the vectors $\vec v_{1}$, $\vec v_{2}$ and $\vec v_{3}$
				are linearly dependent.
				
				\item Since $\vec v_{1}, \vec v_{2}, \vec v_{3}\in\R^{3}$, they
				span $\R^{3}$ if they are linearly independent. We need to determine
				if there exists a non-trivial solution to
				\[
					x\vec v_1+y\vec v_2+z\vec v_3=\vec 0.
				\]
				
				This is equivalent to the system of linear equations
				\[
					\systeme{x-2y+2z=0,2x+y+7z=0,3x+z=0}.
				\]
				
				The only solution to this system is
				\[
					\mat{x\\y\\z}=\mat{0\\0\\0}.
				\]
				
				Since the only solution is the trivial solution, the vectors
				$\vec v_{1}$, $\vec v_{2}$ and $\vec v_{3}$ are linearly
				independent.
				
				\item The lines $\ell_{1}$ and $\ell_{2}$ intersect when the components
				of $\vec x$ are equal. We first set the parameter of
				$\ell_{2}$ to $s$. Then, equating the components of $\vec x$
				gives the system of linear equations
				\[
					\systeme[ts]{t-2s=2,3t-s=3}.
				\]
				
				The solution to this system is
				\[
					\mat{t\\s}=\mat{4/5\\-3/5}.
				\]
				
				Since $\vec x=\mat{9/5\\17/5}$ when $t=4/5$, the intersection
				of $\ell_{1}$ and $\ell_{2}$ is the point $\mat{9/5\\17/5}$.
				
				\item The planes $\mathcal{P}_{1}$ and $\mathcal{P}_{2}$ intersect
				when the components of $\vec x$ are equal. We first set the
				parameter variables of $\mathcal{P}_{2}$ to $q$ and $r$. Then,
				equating the components of $\vec x$ gives the system of
				linear equations
				\[
					\systeme[tsqr]{t-s-q+r=0,-t-s+q-3r=1,2s-q+2r=-1}.
				\]
				
				The complete solution to this system is
				\[
					\mat{t\\s\\q\\r}=u\mat{-2\\-1\\0\\1}+\mat{-1/2\\-1/2\\0\\0}.
				\]
				
				Substitute $q=0$ and $r=u$ into the vector form of
				$\mathcal{P}_{2}$, the intersection of $\mathcal{P}_{1}$ and
				$\mathcal{P}_{2}$ is
				\[
					\mat{x\\y\\z}=u\mat{-1\\3\\-2}+\mat{0\\1\\-1}.
				\]
			\end{enumerate}
		\end{solution}
		\prob Presented below are other students' arguments for the previous
		problem. Evaluate whether their reasoning is correct, and point out the
		flaws in their reasoning, if any.
		\begin{enumerate}
			\item The following argument is for part (a):
			
			Consider the vector equation
			\[
				x\vec v_1+y\vec v_2+z\vec v_3=\vec 0
			\]
			where $x, y, z\in\R$.
			
			Since $(x, y, z)=(0, 0, 0)$ is a solution to the equation, the
			vectors $\vec v_{1}$, $\vec v_{2}$ and $\vec v_{3}$ are linearly
			independent.
			\item The following argument is for part (a):
			
			Consider the vector equation
			\[
				x\vec v_1+y\vec v_2+z\vec v_3=\vec 0
			\]
			where $x, y, z\in\R$.
			
			Notice that $(x, y, z)=(-2, 0, -1)$ is a solution to the equation.
			Since $y=0$ in this solution, it is a trivial solution and therefore
			the vectors $\vec v_{1}$, $\vec v_{2}$ and $\vec v_{3}$ are
			linearly independent.
			
			\item The following argument is for part (c):
			
			The lines $\ell_{1}$ and $\ell_{2}$ intersect when the components
			of $\vec x$ are equal. Equating the components of $\vec x$ gives
			the equations
			\[
				\systeme{t+1=2t+3,3t+1=t+4}.
			\]
			
			This system is equivalent to
			\[
				\systeme{t=-2,2t=3}.
			\]
			
			Since this system is inconsistent, there does not exist a solution
			to this system, so the lines $\ell_{1}$ and $\ell_{2}$ do not
			intersect.
			
			\item The following argument is for part (c):
			
			The lines $\ell_{1}$ and $\ell_{2}$ intersect when the components
			of $\vec x$ are equal. Equating the components of $\vec x$ gives
			the equations
			\[
				\systeme{t+1=2s+3,3t+1=s+4}.
			\]
			
			This system is equivalent to
			\[
				\systeme[ts]{t-2s=2,3t-s=3}.
			\]
			
			The solution to this system is
			\[
				\mat{t\\s}=\mat{4/5\\-3/5},
			\]
			so the lines $\ell_1$ and $\ell_2$ intersect at $\mat{4/5\\3/5}$.
			\item The following argument is for part (d):
			
			Notice that
			\[
				\vec x=\mat{1/2\\-1/2\\0}=1/2\mat{1\\-1\\0}+0\mat{-1\\-1\\2}
			\]
			and
			\[
				\vec x=\mat{1/2\\-1/2\\0}=0\mat{1\\-1\\1}-1/2\mat{-1\\3\\-2}+\mat{0\\1\\-1}.
			\]
			
			Then, $\vec x=\mat{1/2\\-1/2\\0}$ is a point on $\mathcal{P}_{1}$
			and $\mathcal{P}_{2}$, so the planes $\mathcal{P}_{1}$ and
			$\mathcal{P}_{2}$ intersect.
			
			\item The following argument is for part (d):
			
			Notice that $\vec x=\mat{1\\0\\0}$ is a point on $\mathcal{P}_{2}$,
			but it is not a point on $\mathcal{P}_{3}$. Then, we conclude
			that the planes $\mathcal{P}_{2}$ and $\mathcal{P}_{3}$ do not
			intersect.
		\end{enumerate}
		\begin{solution}
			\begin{enumerate}
				\item The reasoning is incorrect. The solution
				$(x, y, z)=(0, 0, 0)$ is the trivial solution to the vector
				equation, and it is always a solution to the homogeneous equation
				\[
					\alpha_1\vec v_1+\cdots+\alpha_k\vec v_k=\vec 0
				\]
				where $\vec v_{1}, \dots, \vec v_{k}$ are arbitrary vectors in
				$\R^{n}$.
				
				To determine if a set of vectors is linearly independent, we
				need to find out whether the trivial solution is the \emph{only}
				solution to the vector equation. That is, there does not exist
				any non-trivial solution to the vector equation.
				
				\item The reasoning is incorrect. The trivial solution is the
				solution where \emph{all} the variables equal to zero, so
				the solution $(x, y, z)=(-2, 0, -1)$ is not the trivial
				solution.
				
				\item The reasoning is incorrect. When equating the components
				of $\vec x$, the parameter variables needs to be set to different
				letters.
				
				A valid system of linear equations is
				\[
					\systeme[ts]{t-2s=2,3t-s=3}.
				\]
				
				Here we have set the parameter $t$ in the vector form of
				$\ell_{2}$ to $s$.
				
				\item The reasoning is incorrect. The solution $\mat{4/5\\-3/5}$
				to the system of linear equations gives the value of $t$ and
				$s$ at the intersection. To find the intersection of $\ell_{1}$
				and $\ell_{2}$, the value $t=4/5$ or $s=-3/5$ needs to be
				plugged into the vector form of $\ell_{1}$ or $\ell_{2}$.
				
				\item The reasoning is correct. Since
				$\vec x=\mat{1/2\\-1/2\\0}$ is a point on both
				$\mathcal{P}_{1}$ and $\mathcal{P}_{2}$, it is in the intersection
				of $\mathcal{P}_{1}$ and $\mathcal{P}_{2}$, so the planes
				$\mathcal{P}_{1}$ and $\mathcal{P}_{2}$ intersect. However,
				we can not determine if the intersection is a line or a
				plane based on only one point, so we need to set up and solve
				an appropriate system of linear equations.
				
				\item The reasoning is incorrect. Since the intersection of two planes can be a line,
				finding one point that is on $\mathcal{P}_{2}$ but not on $\mathcal{P}_{3}$ is
				not sufficient to conclude that the planes do not intersection. There may exist some
				points that are on $\mathcal{P}_{2}$ but not on $\mathcal{P}_{3}$ and some points
				that are on $\mathcal{P}_{3}$ but not on $\mathcal{P}_{2}$.
			\end{enumerate}
		\end{solution}
	\end{problist}
\end{exercises}
