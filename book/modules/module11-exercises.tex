
\begin{exercises}

	\begin{problist}
		\prob For the following matrices, find their null space, column space, and
		row space.
		\begin{enumerate}
			\item $M_{1}=\mat{1&2&1\\3&1&-2\\8&6&-2}$.

			\item $M_{2}=\mat{0&2&1\\3&2&5}$.

			\item $M_{3}=\mat{1&2\\3&1\\4&0}$.

			\item $M_{4}=\mat{1&-2&0&-1\\3&5&-1&0\\2&3&-2&0\\0&0&0&1}$.
		\end{enumerate}

		% Q1 Solutions

		\begin{solution}

			\begin{enumerate}
				\item $M_{1} = \mat{1&2&1\\3&1&-2\\8&6&-2}$;\, $\rref(M_{1})=\mat{1&0&-1\\0&1&1\\0&0&0}$.
				
					$\Null(M_{1})=\Set*{\vec{x} \in \R^3: M_1 \vec{x}=\vec 0}$;
					therefore, we need to solve $\mat{M_1|\vec0}$.
					\begin{align*}
						\Null(M_{1}) &= \Set*{\vec x \in \R^3:\vec x = s\mat{1\\-1\\1} \; \text{for some } s\in\R}\\
						             &= \Span\Set*{\mat{1\\-1\\1}}
					\end{align*}
					\begin{align*}
						\Col(M_{1}) &= \Span\Set*{\mat{1\\3\\8},\mat{2\\1\\6},\mat{1\\-2\\-2}}\\
						        &= \Span\Set*{\mat{1\\3\\8},\mat{2\\1\\6}}
					\end{align*}
					\begin{align*}
						\Row(M_{1}) &= \Span\Set*{\mat{1\\2\\1},\mat{3\\1\\-2},\mat{8\\6\\-2}}\\
						        &= \Span\Set*{\mat{1\\2\\1},\mat{3\\1\\-2}}\\
								&= \Span\Set*{\mat{1\\0\\-1},\mat{0\\1\\1}}
					\end{align*}

				\item $\rref(M_{2})=\mat{1&0&4/3\\0&1&1/2}$.
					$$\Null(M_{2})=\Span\Set*{\mat{4/3\\1/2\\-1}}$$
					$$\Col(M_{2})=\Span\Set*{\mat{0\\3},\mat{2\\2}}=\R^{2}$$
					$$\Row(M_{2})=\Span\Set*{\mat{0\\2\\1},\mat{3\\2\\5}}$$

				\item $\rref(M_{3})=\mat{1&0\\0&1\\0&0}$.
					$$\Null(M_{3})=\Set*{\mat{0\\0}}$$
					$$\Col(M_{3})=\Span\Set*{\mat{1\\3\\4},\mat{2\\1\\0}}$$
					$$\Row(M_{3})=\Span\Set*{\mat{1\\2},\mat{3\\1}}=\R^{2}$$

				\item $\rref(M_{4})=I_{4\times4}$.
					$$\Null(M_{4})=\Set*{\mat{0\\0\\0\\0}}$$
					$$\Col(M_{4})=\R^{4}$$
					$$\Row(M_{4})=\R^{4}$$
			\end{enumerate}
		\end{solution}

		\prob Let $\mathcal P$ be the plane given by $3x+4y+5z=0$, and let $T:\R^{3}\to\R^{3}$
		be projection onto $\mathcal P$.
		\begin{enumerate}
			\item Find $\Range(T)$ and $\Rank(T)$.

			\item Find $\Null(T)$ and $\Nullity(T)$.
		\end{enumerate}

		% Q2 Solutions

		\begin{solution}

			\begin{enumerate}
				\item $\Range(T)=\mathcal{P}$ For any vector $\vec x \in
					\mathcal{P}$; $T(\vec x) = \vec x$ So,
					$\vec x\in \text{Image}(T) \implies \mathcal{P}\subseteq
					\Range(T)$

					Let $\vec{y}\in$ Range $(T) \Rightarrow$ There is
					an input $\vec{a}\in \mathbb{R}^{3}$ such that
					$T(\vec{a})=\vec{y}$. By definition of projection,
					the closest point to $\vec{a}$ in $\mathcal{P}$ is
					$\vec{y}$. In particular, $\vec{y}$ is a point in
					$\mathcal{P}$. $\Rightarrow \vec{y}\in P$

					Thus Range $(T) \subseteq \mathcal{P},$ and hence $\mathcal{P}=$
					Range $(T)$

					$\mathcal{P}$ is a plane through origin in $\mathbb{R}^{3}$;
					So it is a 2 dimensional subspace of
					$\mathbb{R}^{3}$. dim $\mathcal{P}=2$

					$\implies \Dim \Range(T) = \Rank(T)=2$

				\item By the rank-nullity theorem, $\Rank(T)+\Nullity(T) =
					\Dim(\text{domain of }T)$
					\[
						\implies 2+\Nullity(T)=3 \implies \Nullity(T)=1
					\]


					Notice that the vector $\vec n = \mat{3\\4\\5}$ is
					normal to P.

					Thus, the projection of $\vec n$ onto P is $\vec 0$;
					i.e. $T(\vec n) =\vec 0$

					So, $\vec n\in \Null(T)$. Since $\Nullity(T)=1, \;\Null(T)$
					is a 1 dimensional subspace.

					Hence, $\Null(T) = \Span\Set*{\mat{3\\4\\5}}$
			\end{enumerate}
		\end{solution}

		\prob Find the range and null space of the following linear
		transformations.
		\begin{enumerate}
			\item $\mathcal P:\R^{2}\to\R^{2}$, where $\mathcal P$ is
				projection on to the line $y=x$.

			\item Let $\theta\in \R$ and let $\mathcal R:\R^{2}\to\R^{2}$ to be
				the transformation which rotates all vectors by counter-clockwise
				by $\theta$ radians.

			\item $\mathcal F:\R^{2}\to\R^{2}$, where $\mathcal F$ reflects
				over the $x$-axis.

			\item $\mathcal M:\R^{3}\to\R^{3}$ where $\mathcal M$ is the matrix
				transformation given by $\mat{1&2&3\\4&5&6\\7&8&9}$.

			\item $\mathcal Q:\R^{3}\to\R^{1}$ defined by $\mathcal Q\mat{x\\y\\z}=x+z$.
		\end{enumerate}

		% Q3 Solutions

		\begin{solution}

			\begin{enumerate}
				\item Similar to Problem 2.(a)
					$\Range(P) = \Set*{\text{line given by the
					equation } y=x}= \Span\Set*{\mat{1\\1}}$
					$\Null(P) = \Set*{\text{line given by the equation }
					y=-x}= \Span\Set*{\mat{1\\-1}}$

				\item Notice that if we have a non-zero vector, after rotation
					the vector still remains non-zero as rotation does
					not change magnitude of vectors. Thus, $\Null(R)=\Set*{\mat{0\\0}}$.
					From the rank-nullity theorem, $\Rank(R) = 2$
					$\implies \Range(R) = \R^{2}$

				\item So, $F\mat{x\\y}= \mat{x\\-y}$. Thus,
					$F\mat{x\\y}= \mat{0\\0}\implies x=0, \; y=0$
					Thus, $\Null(F)=\Set*{\mat{0\\0}}$. From the rank-nullity
					theorem, $\Rank(F) = 2$ $\implies \Range(F) = \R^{2}$

				\item Let M be the matrix of the transformation $\mathcal{M}$.
					$\Rref(M)= \mat{1&0&-1\\0&1&2\\0&0&0}$ So,
					$\Range(M)=\Col(M)=\Span\Set*{\mat{1\\4\\7},\mat{2\\5\\8}}$
					$\Null(M)=\Span\Set*{\mat{1\\-2\\1}}$

				\item Notice, $\Range(Q)\subseteq \R^{1}$ For any $t\in\R^{1},
					\; Q\mat{t\\0\\0}=t+0=t \implies t \in \Range(Q)$ Hence,
					$\Range(Q) = \R^{1}$ For $\vec x \in \R^{3}, \; Q(\vec
					x) = \vec 0 \implies Q\mat{x\\y\\z}=0 \implies x+z=0$
					Thus, $\Null(Q)$ is the plane given by the equation
					$x+z=0$ Alternatively, $\Null(Q) = \Span\Set*{\mat{1\\0\\-1},\mat{0\\1\\0}}$
			\end{enumerate}
		\end{solution}

		\prob
		\begin{enumerate}
			\item Let $\mathcal T$ be the transformation induced by the matrix
				$\mat{7&5\\-2&-2}$, and $\vec v=3\xhat-3\yhat$. Compute $\mathcal
				T\vec v$ and $[\mathcal T\vec v]_{\mathcal E}$.

			\item Let $\mathcal T$ be the transformation induced by the matrix
				$\mat{3&7&5\\1&-2&-2}$, and $\vec v=2\xhat+0\yhat+4\zhat$.
				Compute $\mathcal T\vec v$ and $[\mathcal T\vec v]_{\mathcal
				E}$.
		\end{enumerate}

		% Q4 Solutions

		\begin{solution}

			\begin{enumerate}
				\item $[\vec v]_{\varepsilon} = \mat{3\\-3}\;$ So, $[T\vec
					v]_{\varepsilon} = \mat{7&5\\-2&-2}\mat{3\\-3}=\mat{6\\0}$

					Thus $T\vec v =6\vec e_{1}.$

				\item $[\vec v]_{\varepsilon} = \mat{2\\0\\4}\;$ So, $[T\vec
					v]_{\varepsilon} = \mat{3&7&5\\1&-2&-2}\mat{2\\0\\4}=\mat{26\\-6}$

					Thus $T\vec v =26\vec e_{1}-6\vec e_{2}\;$ [Here $\vec
					e_{1}, \vec e_{2}\in \R^{2}$]
			\end{enumerate}
		\end{solution}

		\prob For each statement below, determine whether it is true or false.
		Justify your answer.
		\begin{enumerate}
			\item Let $A$ be an arbitrary matrix. Then $\Col(A)=\Col(A^{T})$.

			\item Let $T:\R^{m}\to\R^{n}$ be a transformation (not necessarily
				linear). If $\Null(T)=\Set{\vec x\in\R^m \given T(\vec x)=\vec
				0}$ is a subspace, then $T$ is linear.

			\item Let $T:\R^{m}\to\R^{n}$ be a linear transformation. Then $\Nullity(T)
				\geq n$.

			\item Let $T:\R^{m}\to\R^{n}$ be a linear transformation induced by
				a matrix $M$. If $\Rank(T) = n$, then $\Nullity(M) = 0$.
		\end{enumerate}


		\begin{solution}

			\begin{enumerate}
				\item False. $\Range(A)$ and $\Range(A^{T})$ need not even
					be in the same space.

					For example, take $A=\mat{1&0\\0&1\\0&0}, A^{T}=\mat{1&0&0\\0&1&0}$
					Then $\Range(A)=\Set*{xy-\text{plane in }\R^3}$,
					whereas $\Range(A^{T})=\R^{2}$

				\item False. Consider, $T:\R^{2}\to\R^{2}$ given by
					\[
						T\mat{x\\y}=\mat{x^2\\y}
					\]
					 Then, $\Null(T)= \Set*{\mat{0\\0}}$ which is a subspace.
					But $T$ is not linear.

				\item False. From rank-nullity theorem,
					\begin{align*}
						\Nullity(T) & \leq \Dim(\text{ domain of }T) & =m \quad \text{ (in this case) }
					\end{align*}So, for any $n>m$ this is false. Consider
					for example
					\begin{align*}
						 & T:\R^{1}\to\R^{2}\;\text{given by} & T(x)=\mat{x\\0};\; \Nullity(T)=0
					\end{align*}

				\item False. This is false whenever $m>n$ (follows from rank-nullity
					theorem) Take for example: $T:\R^{3}\to\R^{2}$ induced
					by the matrix $A=\mat{1&0&0\\0&1&0}$;\\ $\Rank(T)=\Dim(\Col
					A)=2$ $\Nullity(T) = \Nullity(A)=1$
			\end{enumerate}
		\end{solution}

		\prob Give an example of a $3\times4$ matrix $M$ with the specified rank, or explain why one cannot exist.
		\begin{enumerate}
			\item $\Rank(M) = 0$
			\item $\Rank(M) = 1$
			\item $\Rank(M) = 3$
			\item $\Rank(M) = 4$
		\end{enumerate}

		\begin{solution}
			\begin{enumerate}
				\item $\mat{0&0&0&0\\0&0&0&0\\0&0&0&0}$

				\item $\mat{1&0&0&0\\0&0&0&0\\0&0&0&0}$

				\item $\mat{1&0&0&0\\0&1&0&0\\0&0&1&0}$

				\item A $3\times4$ matrix with rank 4 cannot exist. Examples of justifications
					are: 1) The induced transformation maps from $R^{3}$ to $R^{4}$. By
					the rank-nullity theorem, the dimension of the range (the rank) cannot
					be greater than the dimension of the domain. 2) A matrix with rank 4
					has 4 pivots. A $3\times4$ matrix can have at most 3 pivots, which is
					less than 4
			\end{enumerate}
		\end{solution}

		\prob Let $\mathcal{P}:\R^3\to\R^3$ be the projection onto the $xy$-plane.
		\begin{enumerate}
			\item Find a matrix $M_\mathcal{P}$ for the transformation.
			\item Find the range of $\mathcal{P}$.
			\item Find the column space of $M_\mathcal{P}$. Are there any similarities to your answer in the previous part?
			\item Find the null space of $\mathcal{P}$ and $M_\mathcal{P}$. Are there similarities between the null space of a linear transformation and its associated matrix?
		\end{enumerate}

		\begin{solution}
			\begin{enumerate}
				\item $\mat{1&0&0\\0&1&0\\0&0&0}$
				\item The range of $\mathcal{P}$ is the $xy$-plane.
				\item The column space of $M_\mathcal{P}$ is $\Span\Set*{\mat{1\\0\\0},\mat{0\\1\\0}}$ which is the $xy$-plane. The column space of a matrix is the same as the range of its induced transformation.
				\item The null space of $\mathcal{P}$ and $M_\mathcal{P}$ is $\Span\Set*{\mat{0\\0\\1}}$ which is the $z$-axis. The null space of a matrix is the same as the null space of its induced transformation.
			\end{enumerate}
		\end{solution}

	\end{problist}
\end{exercises}
