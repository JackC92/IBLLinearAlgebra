
Linear transformations transform vectors, but they also change \emph{sets}.

\begin{center}
	\begin{tikzpicture}
		\begin{axis}[
		    anchor=origin,
		    name=plot1,
		    disabledatascaling,
		    xmin=-1,xmax=2,
		    ymin=0,ymax=3,
			xtick={-2,...,4},
			ytick={-2,...,4},
			xticklabels={,,},
			yticklabels={,,},
		    x=1cm,y=1cm,
		    grid=both,
		    grid style={line width=.1pt, draw=gray!10},
		    %major grid style={line width=.2pt,draw=gray!50},
		    axis lines=middle,
		    minor tick num=0,
		    enlargelimits={abs=0.5},
		    axis line style={latex-latex},
		    ticklabel style={font=\tiny,fill=white},
		    xlabel style={at={(ticklabel* cs:1)},anchor=north west},
		    ylabel style={at={(ticklabel* cs:1)},anchor=south west}
		]

		\begin{scope}[cm={1,0,0,1,(0,0)}, myorange]
			%\node[transform shape, black] at (.5,.5) {$C_2$};
			%\fill[opacity=.3] (0,0) rectangle (1,1);
			\draw[fill, opacity=.3] plot [smooth cycle] coordinates {(0,0) (1,0.1) (2,0.3) (2,1.4) (1.5,2.5) (0.8,2.5) (0.3,1.2) (-0.2,0.6)};
			\draw plot [smooth cycle] coordinates {(0,0) (1,0.1) (2,0.3) (2,1.4) (1.5,2.5) (0.8,2.5) (0.3,1.2) (-0.2,0.6)};
			\draw[black!70, very thick, ->] (0,0) -- (1,0);
			\draw[black!70, very thick, densely dotted, ->] (0,0) -- (0,1);
		\end{scope}
			\coordinate (A) at (1.25,3.25);
		\end{axis}
		\begin{axis}[
			name=plot2,
			at={($(plot1.east) + (1cm, 0)$)},
			anchor=west,
		    disabledatascaling,
		    xmin=-1,xmax=4,
		    ymin=0,ymax=3,
			xtick={-2,...,4},
			ytick={-2,...,4},
			xticklabels={,,},
			yticklabels={,,},
		    x=1cm,y=1cm,
		    grid=both,
		    grid style={line width=.1pt, draw=gray!10},
		    %major grid style={line width=.2pt,draw=gray!50},
		    axis lines=middle,
		    minor tick num=0,
		    enlargelimits={abs=0.5},
		    axis line style={latex-latex},
		    ticklabel style={font=\tiny,fill=white},
		    xlabel style={at={(ticklabel* cs:1)},anchor=north west},
		    ylabel style={at={(ticklabel* cs:1)},anchor=south west}
		]

		\begin{scope}[cm={2,1,-1,.5,(0,0)}, BlueGreen]
			%\node[transform shape, black] at (.5,.5) {$C_2$};
			%\node[black] at (.5,.5) {$\mathcal T(C_2)$};
			\draw[fill, opacity=.3] plot [smooth cycle] coordinates {(0,0) (1,0.1) (2,0.3) (2,1.4) (1.5,2.5) (0.8,2.5) (0.3,1.2) (-0.2,0.6)};
			\draw plot [smooth cycle] coordinates {(0,0) (1,0.1) (2,0.3) (2,1.4) (1.5,2.5) (0.8,2.5) (0.3,1.2) (-0.2,0.6)};
			\draw[black!70, very thick, ->] (0,0) -- (1,0);
			\draw[black!70, very thick, densely dotted, ->] (0,0) -- (0,1);
		\end{scope}
			\coordinate (B) at (-.25,3.25);
		\end{axis}

		\draw[thick] (A) edge[bend left, ->] node[midway,above] {Linear Transformation} (B) ;
	\end{tikzpicture}
\end{center}

It turns out to be particularly useful to track by how much a linear transformation
changes area/volume. This number (which is associated with a linear transformation with
the same domain and codomain) is called the \emph{determinant}\footnote{ This number is \emph{almost} the determinant.
The only difference is that the determinant might have a $\pm$ in front.}.

\Heading{Volumes}\index{Volume}

In this module, most examples will be in $\R^2$ because they're easier to draw.
The definitions given will extend to $\R^n$ for any $n$, however we need to establish some conventions
to properly express these ideas in English. In English, we say that a two-dimensional figure has an
\emph{area} and a three-and-up dimensional figure has a \emph{volume}. In this section, \emph{we
will use the term volume to also mean area} where appropriate.

To measure how volume changes, we need to compare input volumes and output volumes. 
The easiest volume to compute is that of the \emph{unit $n$-cube}, which has a special notation.

\SavedDefinitionRender{Unitncube}

$C_2$ should look familiar as the unit square in $\R^2$ with lower-left corner at the origin.

\begin{center}
	\begin{tikzpicture}
		\begin{axis}[
		    anchor=origin,
		    disabledatascaling,
		    xmin=-1,xmax=2,
		    ymin=0,ymax=1,
			xtick={-2,...,4},
			ytick={-2,...,4},
		    x=1.5cm,y=1.5cm,
		    grid=both,
		    grid style={line width=.1pt, draw=gray!10},
		    %major grid style={line width=.2pt,draw=gray!50},
		    axis lines=middle,
		    minor tick num=0,
		    enlargelimits={abs=0.5},
		    axis line style={latex-latex},
		    ticklabel style={font=\tiny,fill=white},
		    xlabel style={at={(ticklabel* cs:1)},anchor=north west},
		    ylabel style={at={(ticklabel* cs:1)},anchor=south west}
		]

		\begin{scope}[cm={1,0,0,1,(0,0)}, mypink]
			\node[transform shape, black] at (.5,.5) {$C_2$};
			\fill[opacity=.3] (0,0) rectangle (1,1);
			\draw[very thick, ->] (0,0) -- (1,0);
			\draw[very thick, densely dotted, ->] (0,0) -- (0,1);
		\end{scope}
		\end{axis}
	\end{tikzpicture}
\end{center}

$C_n$ always has volume $1$\footnote{ The fact that the volume of $C_n$ is $1$ is actually by definition.}, and by
analyzing the image of $C_n$ under a linear transformation, we can see by
how much a given transformation changes volume.

\begin{example}
	Let $\mathcal T:\R^2\to\R^2$ be defined by $\mathcal T\mat{x\\y}=\matc{2x-y\\x+\tfrac{1}{2}y}$.
	Find the volume of $\mathcal T(C_2)$.

	Recall that $C_2$ is the unit square in $\R^2$ with sides given by $\xhat = \mat{1\\0}$ 
	and $\yhat = \mat{0\\1}$. Applying the linear transformation $\mathcal T$ to $\xhat$ and $\yhat$, we obtain
	\[
	    \mathcal T(\xhat)=\mat{2\\1} \qquad\text{and}\qquad \mathcal T(\yhat)=\mat{-1\\\frac{1}{2}}.
	\]
	Plotting $\mat{2\\1}$ and $\mat{-1\\\frac{1}{2}}$, we see $\mathcal T(C_2)$ is a parallelogram 
	with base $\sqrt{5}$ and height $\frac{2\sqrt{5}}{5}$.
	
\begin{center}
	\begin{tikzpicture}
		\begin{axis}[
		    anchor=origin,
		    name=plot1,
		    disabledatascaling,
		    xmin=-1,xmax=2,
		    ymin=0,ymax=1,
			xtick={-2,...,4},
			ytick={-2,...,4},
		    x=1.5cm,y=1.5cm,
		    grid=both,
		    grid style={line width=.1pt, draw=gray!10},
		    %major grid style={line width=.2pt,draw=gray!50},
		    axis lines=middle,
		    minor tick num=0,
		    enlargelimits={abs=0.5},
		    axis line style={latex-latex},
		    ticklabel style={font=\tiny,fill=\currentbackgroundcolor},
		    xlabel style={at={(ticklabel* cs:1)},anchor=north west},
		    ylabel style={at={(ticklabel* cs:1)},anchor=south west}
		]

		\begin{scope}[cm={1,0,0,1,(0,0)}, mypink]
			\node[transform shape, black] at (.5,.5) {$C_2$};
			\fill[opacity=.3] (0,0) rectangle (1,1);
			\draw[very thick, ->] (0,0) -- (1,0);
			\draw[very thick, densely dotted, ->] (0,0) -- (0,1);
		\end{scope}
			\coordinate (A) at (1.25,1.25);
		\end{axis}
		\begin{axis}[
			name=plot2,
			at={($(plot1.east) + (1cm, 0)$)},
			anchor=west,
		    disabledatascaling,
		    xmin=-1,xmax=2,
		    ymin=0,ymax=1,
			xtick={-2,...,4},
			ytick={-2,...,4},
		    x=1.5cm,y=1.5cm,
		    grid=both,
		    grid style={line width=.1pt, draw=gray!10},
		    %major grid style={line width=.2pt,draw=gray!50},
		    axis lines=middle,
		    minor tick num=0,
		    enlargelimits={abs=0.5},
		    axis line style={latex-latex},
		    ticklabel style={font=\tiny,fill=\currentbackgroundcolor},
		    xlabel style={at={(ticklabel* cs:1)},anchor=north west},
		    ylabel style={at={(ticklabel* cs:1)},anchor=south west}
		]

		\begin{scope}[cm={2,1,-1,.5,(0,0)}, mygreen]
			%\node[transform shape, black] at (.5,.5) {$C_2$};
			\node[black] at (.5,.5) {$\mathcal T(C_2)$};
			\fill[opacity=.3] (0,0) rectangle (1,1);
			\draw[very thick, ->] (0,0) -- (1,0);
			\draw[very thick, densely dotted, ->] (0,0) -- (0,1);
		\end{scope}
			\coordinate (B) at (-.25,1.25);
		\end{axis}

		\draw[thick] (A) edge[bend left, ->] node[midway,above] {$\mathcal T$} (B) ;
	\end{tikzpicture}
\end{center}
	
	Therefore, the volume of $\mathcal T(C_2)$ is 2.
\end{example}

Let $\Vol(X)$ stand for the volume of the set $X$.
Given a linear transformation $\mathcal S:\R^n\to\R^n$, we can define a number
\[
	\VolChange(\mathcal S)=\frac{\Vol(\mathcal S(C_n))}{\Vol(C_n)}=\frac{\Vol(\mathcal S(C_n))}{1}=\Vol(\mathcal S(C_n)).
\]

A priori, $\VolChange(\mathcal S)$ only describes how $\mathcal S$ changes the volume of $C_n$. However, because $\mathcal S$
is a linear transformation, $\VolChange(\mathcal S)$ actually describes how $\mathcal S$ changes the volume of any figure.

\begin{theorem}
	Let $\mathcal T:\R^n\to\R^n$ be a linear transformation and let $X\subseteq \R^n$ be a subset
	with volume $\alpha$. Then the volume of $\mathcal T(X)$ is $\alpha\!\cdot\!\VolChange(\mathcal T)$.
\end{theorem}

A full proof of the above theorem requires calculus and limits, but the linear algebra ideas are based on the following
theorems.

\begin{theorem}
	Suppose $\mathcal T:\R^n\to\R^n$ is a linear transformation, $X\subseteq \R^n$ is a subset, and the
	volume of $\mathcal T(X)$ is $\alpha$.
	Then for any $\vec p\in \R^n$, the volume of $\mathcal T(X+\Set{\vec p})$ is $\alpha$.
\end{theorem}
\begin{proof}
	Fix $\mathcal T:\R^n\to\R^n$, $X\subseteq \R^n$, and $\vec p\in \R^n$. Combining linearity with
	the definition of set addition, we see
	\[
		\mathcal T(X+\Set{\vec p}) = \mathcal T(X)+\mathcal T(\Set{\vec p}) = \mathcal T(X)+\Set{\mathcal T(\vec p)}
	\]
	and so $\mathcal T(X+\Set{\vec p})$ is just a translation of $\mathcal T(X)$. Since translations don't change
	volume, $\mathcal T(X+\Set{\vec p})$ and $\mathcal T(X)$ must have the same volume.
\end{proof}

\begin{theorem}
	Fix $k$ and let $B_n$ be $C_n$ scaled to have side lengths $\frac{1}{k}$ and let $\mathcal T:\R^n\to\R^n$ be a linear
	transformation. Then
	\[
		\VolChange(\mathcal T) = \frac{\Vol(\mathcal T(B_n))}{\Vol(\mathcal B_n)}.
	\]
\end{theorem}

Rather than giving a formal proof of the above theorem, let's make a motivating picture.

\begin{center}
	\begin{tikzpicture}[
			%x=2cm,y=2cm,
		    x=1.5cm,y=1.5cm,
			]

	\begin{scope}[	spy using outlines={circle, lens={scale=5}, size=3.25cm, connect spies}]
			\coordinate (A) at (.3,.15*4);
			\coordinate (B) at (.4,.1*5);
			\fill[myorange,opacity=.5] (A) rectangle (B);
		\begin{axis}[
		    anchor=origin,
		    name=plot1,
		    disabledatascaling,
		    xmin=0,xmax=1,
		    ymin=0,ymax=1,
			xtick={-4,...,4},
			ytick={-2,...,4},
		    x=1.5cm,y=1.5cm,
		    %x=2cm,y=2cm,
		    grid=both,
		    grid style={line width=.1pt, draw=gray!10},
		    %major grid style={line width=.2pt,draw=gray!50},
		    axis lines=middle,
		    minor tick num=0,
		    enlargelimits={abs=0.5},
		    axis line style={latex-latex},
		    ticklabel style={font=\tiny,fill=white},
		    xlabel style={at={(ticklabel* cs:1)},anchor=north west},
		    ylabel style={at={(ticklabel* cs:1)},anchor=south west}
		]
		\end{axis}


			\begin{scope}[mypink, very thin]
				\foreach \x in {0,.1,...,1.05} {
					\draw ($(\x,0)$) -- ($(\x,1)$);
				}
				\foreach \y in {0,.1,...,1.05} {
					\draw ({0,\y}) -- ({1,\y});
				}
			\end{scope}


			\spy [black!80, dashed] on ($.5*(A)+.5*(B)$)
		     in node (a) at (2.3,1);


	\end{scope}
	\coordinate (L) at ($(a.center) - (.25,.25)$);
	\coordinate (R) at ($(a.center) - (-.25,.25)$);
	\coordinate (T) at ($(a.center) + (.25,.25)$);
	\node[right, mypink] at ($.5*(R)+.5*(T)$) {$1/k$};
	\node[below, mypink] at ($.5*(R)+.5*(L)$) {$1/k$};
	\node[black] at (a.center) {$B_2$};

	\begin{scope}[	spy using outlines={circle, lens={scale=5}, size=3.25cm, connect spies}, cm={1,0,0,1,(4.5,0)}]
		\begin{axis}[
		    anchor=origin,
		    name=plot2,
		    disabledatascaling,
		    xmin=0,xmax=1,
		    ymin=0,ymax=1,
			xtick={-4,...,4},
			ytick={-2,...,4},
		    x=1.5cm,y=1.5cm,
		    grid=both,
		    grid style={line width=.1pt, draw=gray!10},
		    %major grid style={line width=.2pt,draw=gray!50},
		    axis lines=middle,
		    minor tick num=0,
		    enlargelimits={abs=0.5},
		    axis line style={latex-latex},
		    ticklabel style={font=\tiny,fill=white},
		    xlabel style={at={(ticklabel* cs:1)},anchor=north west},
		    ylabel style={at={(ticklabel* cs:1)},anchor=south west}
		]
		\end{axis}



			\begin{scope}[mygreen, very thin,cm={1.5,.2,.5,1,(0,0)}]
				\coordinate (A) at (.3,.15*4);
				\coordinate (B) at (.4,.1*5);
				\fill[myorange,opacity=.5] (A) rectangle (B);
				\foreach \x in {0,.1,...,1.05} {
					\draw ($(\x,0)$) -- ($(\x,1)$);
				}
				\foreach \y in {0,.1,...,1.05} {
					\draw ({0,\y}) -- ({1,\y});
				}
			\end{scope}


			\spy [black!80, dashed] on ($.5*(A)+.5*(B)$)
		     in node (a) at (3.3,1);


	\end{scope}
	\coordinate (L) at ($(a.center) - (.25,.25)$);
	\coordinate (R) at ($(a.center) - (-.25,.25)$);
	\coordinate (T) at ($(a.center) + (.25,.25)$);
	\node[black] at (a.center) {$\mathcal T(B_2)$};

	\draw[thick] (plot1.south) edge[bend right, ->] node[midway, above] {$\mathcal T$} (plot2.south);
	\node[yshift=.5cm] (bn) at (plot1.north) {$k^2$ copies of $B_2$} node[above, yshift=-5pt] at (bn.north) {$C_2$ contains};
	\node[yshift=.5cm] (tbn) at (plot2.north) {$k^2$ copies of $\mathcal T(B_2)$} node[above, yshift=-5pt] at (tbn.north) {$\mathcal T(C_2)$ contains};
	\end{tikzpicture}
\end{center}

The argument now goes: there are $k^n$ copies of $B_n$ in $C_n$ and $k^n$ copies of $\mathcal T(B_n)$ in
$T(C_n)$. Thus,
\[
	\VolChange(\mathcal T)
	=\frac{\Vol(\mathcal T(C_n))}{\Vol(\mathcal C_n)}=\frac{k^n\Vol(\mathcal T(B_n))}{k^n\Vol(\mathcal B_n)}
	=\frac{\Vol(\mathcal T(B_n))}{\Vol(\mathcal B_n)}.
\]

Now we can finally show that for a linear transformation $\mathcal T:\R^n\to\R^n$,
the number ``$\VolChange(\mathcal T)$'' actually corresponds to how much $\mathcal T$ changes the volume
of any figure by.

The argument goes as follows: for a figure $X\subseteq \R^n$, we can fill it with shrunken and
translated copies, $B_n$, of $C_n$. The same number of copies of $\mathcal T(B_n)$ fit inside $\mathcal T(X)$
as do $B_n$'s fit inside $X$. Therefore, the change in volume between $\mathcal T(X)$ and $X$ must be the same
as the change in volume between $\mathcal T(B_n)$ and $B_n$, which is $\VolChange(\mathcal T)$.

\begin{center}
	\begin{tikzpicture}[
			%x=2cm,y=2cm,
		    x=1cm,y=1cm,
			]

	\begin{scope}[	spy using outlines={circle, lens={scale=5}, size=3.25cm, connect spies}]
		\begin{axis}[
		    anchor=origin,
		    name=plot1,
		    disabledatascaling,
		    xmin=0,xmax=1,
		    ymin=0,ymax=1,
			xtick={-4,...,4},
			ytick={-2,...,4},
		    x=1.5cm,y=1.5cm,
		    %x=2cm,y=2cm,
		    grid=both,
		    grid style={line width=.1pt, draw=gray!10},
		    %major grid style={line width=.2pt,draw=gray!50},
		    axis lines=middle,
		    minor tick num=0,
		    enlargelimits={abs=0.5},
		    axis line style={latex-latex},
		    ticklabel style={font=\tiny,fill=white},
		    xlabel style={at={(ticklabel* cs:1)},anchor=north west},
		    ylabel style={at={(ticklabel* cs:1)},anchor=south west}
		]
		\end{axis}


		\begin{scope}[cm={1,0,0,1,(0,0)}, myorange]
			\clip plot [smooth cycle] coordinates {(0,0) (1,0.1) (2,0.3) (2,1.4) (1.5,2.5) (0.8,2.5) (0.3,1.2) (-0.2,0.6)};
			\coordinate (A) at (.35,.15*4);
			\coordinate (B) at (.5,.15*5);
			\fill[myorange,opacity=.5] (A) rectangle (B);
			\begin{scope}[mypink, very thin]
				\draw plot [smooth cycle] coordinates {(0,0) (1,0.1) (2,0.3) (2,1.4) (1.5,2.5) (0.8,2.5) (0.3,1.2) (-0.2,0.6)};
				\foreach \x in {-1,-.85,...,2.05} {
					\draw ($(\x,0)$) -- ($(\x,3)$);
				}
				\foreach \y in {0,.15,...,3.05} {
					\draw ({-1,\y}) -- ({3,\y});
				}
			\end{scope}
		\end{scope}



			\spy [black!80, dashed] on ($.5*(A)+.5*(B)$)
		     in node (a) at (-2.5,1.5);


	\end{scope}
	\coordinate (L) at ($(a.center) - (.25,.25)$);
	\coordinate (R) at ($(a.center) - (-.25,.25)$);
	\coordinate (T) at ($(a.center) + (.25,.25)$);
	\node[black] at (a.center) {$B_2$};

	\begin{scope}[	spy using outlines={circle, lens={scale=8}, size=3.25cm, connect spies}, cm={1,0,0,1,(3.5,0)}]
		\begin{axis}[
		    anchor=origin,
		    name=plot2,
		    disabledatascaling,
		    xmin=0,xmax=1,
		    ymin=0,ymax=1,
			xtick={-4,...,4},
			ytick={-2,...,4},
		    x=1.5cm,y=1.5cm,
		    grid=both,
		    grid style={line width=.1pt, draw=gray!10},
		    %major grid style={line width=.2pt,draw=gray!50},
		    axis lines=middle,
		    minor tick num=0,
		    enlargelimits={abs=0.5},
		    axis line style={latex-latex},
		    ticklabel style={font=\tiny,fill=white},
		    xlabel style={at={(ticklabel* cs:1)},anchor=north west},
		    ylabel style={at={(ticklabel* cs:1)},anchor=south west}
		]
		\end{axis}



			\begin{scope}[mygreen, very thin,cm={1.5,.2,.5,1,(0,0)}]
				\coordinate (A) at (.3,.15*4);
				\coordinate (B) at (.4,.1*5);
				\fill[myorange,opacity=.5] (A) rectangle (B);
				\clip plot [smooth cycle] coordinates {(0,0) (1,0.1) (2,0.3) (2,1.4) (1.5,2.5) (0.8,2.5) (0.3,1.2) (-0.2,0.6)};
				\begin{scope}[mygreen, very thin]
					\draw plot [smooth cycle] coordinates {(0,0) (1,0.1) (2,0.3) (2,1.4) (1.5,2.5) (0.8,2.5) (0.3,1.2) (-0.2,0.6)};
					\foreach \x in {-1,-.9,...,2.05} {
						\draw ($(\x,0)$) -- ($(\x,3)$);
					}
					\foreach \y in {0,.1,...,3.05} {
						\draw ({-1,\y}) -- ({3,\y});
					}
				\end{scope}
			\end{scope}
		

			\spy [black!80, dashed] on ($.5*(A)+.5*(B)$)
		     in node (a) at (5.7,1.5);


	\end{scope}
	\coordinate (L) at ($(a.center) - (.25,.25)$);
	\coordinate (R) at ($(a.center) - (-.25,.25)$);
	\coordinate (T) at ($(a.center) + (.25,.25)$);
	\node[black] at (a.center) {$\mathcal T(B_2)$};

	\draw[thick] ($(plot1.north)+(0,.75cm)$) edge[bend left, ->] 
		node[midway, below] {$\mathcal T$}
		node[midway, above] {Number of pieces doesn't change}
	($(plot2.north)+(0,.75cm)$);
	\node[mypink] at (plot1.south) {$X$};
	\node[mygreen] at (plot2.south) {$\mathcal T(X)$};
	%\node[yshift=.5cm] (bn) at (plot1.north) {$k^2$ copies of $B_2$} node[above, yshift=-5pt] at (bn.north) {$C_2$ contains};
	%\node[yshift=.5cm] (tbn) at (plot2.north) {$k^2$ copies of $\mathcal T(B_2)$} node[above, yshift=-5pt] at (tbn.north) {$\mathcal T(C_2)$ contains};
	\end{tikzpicture}
\end{center}

\Heading{The Determinant}

The determinant of a linear transformation $\mathcal T:\R^n\to\R^n$ is \emph{almost} the same
as $\VolChange(\mathcal T)$, but with one twist: \emph{orientation}.

\SavedDefinitionRender{Determinant}

We need to understand what the term \emph{oriented volume} means.
We've previously defined the orientation of a basis, and we can use the orientation
of a basis to define whether a linear transformation is \emph{orientation preserving}
or \emph{orientation reversing}.

\SavedDefinitionRender{OrientationPreservingLinearTransformation}

\begin{center}
	\begin{tikzpicture}
		\begin{axis}[
		    anchor=origin,
		    name=plot1,
		    disabledatascaling,
		    xmin=-1,xmax=2,
		    ymin=0,ymax=1,
			xtick={-2,...,4},
			ytick={-2,...,4},
			xticklabels={,,},
			yticklabels={,,},
		    x=1.5cm,y=1.5cm,
		    grid=both,
		    grid style={line width=.1pt, draw=gray!10},
		    %major grid style={line width=.2pt,draw=gray!50},
		    axis lines=middle,
		    minor tick num=0,
		    enlargelimits={abs=0.5},
		    axis line style={latex-latex},
		    ticklabel style={font=\tiny,fill=white},
		    xlabel style={at={(ticklabel* cs:1)},anchor=north west},
		    ylabel style={at={(ticklabel* cs:1)},anchor=south west}
		]

		\begin{scope}[cm={1,0,0,1,(0,0)}, mypink]
			%\node[transform shape, black] at (.5,.5) {$C_2$};
			\fill[opacity=.3] (0,0) rectangle (1,1);
			\draw[very thick, ->] (0,0) -- (1,0) node[midway, below] {$\xhat$};
			\draw[very thick, densely dotted, ->] (0,0) -- (0,1) node[midway, left] {$\yhat$};
		\end{scope}
			\coordinate (A) at (1.25,1.25);
		\end{axis}
		\begin{axis}[
			name=plot2,
			at={($(plot1.east) + (1cm, 0)$)},
			anchor=west,
		    disabledatascaling,
		    xmin=-1,xmax=2,
		    ymin=0,ymax=1,
			xtick={-2,...,4},
			ytick={-2,...,4},
			xticklabels={,,},
			yticklabels={,,},
		    x=1.5cm,y=1.5cm,
		    grid=both,
		    grid style={line width=.1pt, draw=gray!10},
		    %major grid style={line width=.2pt,draw=gray!50},
		    axis lines=middle,
		    minor tick num=0,
		    enlargelimits={abs=0.5},
		    axis line style={latex-latex},
		    ticklabel style={font=\tiny,fill=white},
		    xlabel style={at={(ticklabel* cs:1)},anchor=north west},
		    ylabel style={at={(ticklabel* cs:1)},anchor=south west}
		]

		\begin{scope}[cm={2,1,-1,.5,(0,0)}, mygreen]
			%\node[transform shape, black] at (.5,.5) {$C_2$};
			%\node[black] at (.5,.5) {$\mathcal T(C_2)$};
			\fill[opacity=.3] (0,0) rectangle (1,1);
			\draw[very thick, ->] (0,0) -- (1,0) node[midway, below right] {$\mathcal T(\xhat)$};;
			\draw[very thick, densely dotted, ->] (0,0) -- (0,1) node[midway, below left, yshift=4pt] {$\mathcal T(\yhat)$};
		\end{scope}
			\coordinate (B) at (-.25,1.25);
		\end{axis}

		\draw[thick] (A) edge[bend left, ->] node[midway,above] {$\mathcal T$} (B) ;
		\node at ($(plot2.north)+(0,.5cm)$) {Orientation Preserving};
	\end{tikzpicture}

	\begin{tikzpicture}
		\begin{axis}[
		    anchor=origin,
		    name=plot1,
		    disabledatascaling,
		    xmin=-1,xmax=2,
		    ymin=0,ymax=1,
			xtick={-2,...,4},
			ytick={-2,...,4},
			xticklabels={,,},
			yticklabels={,,},
		    x=1.5cm,y=1.5cm,
		    grid=both,
		    grid style={line width=.1pt, draw=gray!10},
		    %major grid style={line width=.2pt,draw=gray!50},
		    axis lines=middle,
		    minor tick num=0,
		    enlargelimits={abs=0.5},
		    axis line style={latex-latex},
		    ticklabel style={font=\tiny,fill=white},
		    xlabel style={at={(ticklabel* cs:1)},anchor=north west},
		    ylabel style={at={(ticklabel* cs:1)},anchor=south west}
		]

		\begin{scope}[cm={1,0,0,1,(0,0)}, mypink]
			%\node[transform shape, black] at (.5,.5) {$C_2$};
			\fill[opacity=.3] (0,0) rectangle (1,1);
			\draw[very thick, ->] (0,0) -- (1,0) node[midway, below] {$\xhat$};
			\draw[very thick, densely dotted, ->] (0,0) -- (0,1) node[midway, left] {$\yhat$};
		\end{scope}
			\coordinate (A) at (1.25,1.25);
		\end{axis}
		\begin{axis}[
			name=plot2,
			at={($(plot1.east) + (1cm, 0)$)},
			anchor=west,
		    disabledatascaling,
		    xmin=-1,xmax=2,
		    ymin=0,ymax=1,
			xtick={-2,...,4},
			ytick={-2,...,4},
			xticklabels={,,},
			yticklabels={,,},
		    x=1.5cm,y=1.5cm,
		    grid=both,
		    grid style={line width=.1pt, draw=gray!10},
		    %major grid style={line width=.2pt,draw=gray!50},
		    axis lines=middle,
		    minor tick num=0,
		    enlargelimits={abs=0.5},
		    axis line style={latex-latex},
		    ticklabel style={font=\tiny,fill=white},
		    xlabel style={at={(ticklabel* cs:1)},anchor=north west},
		    ylabel style={at={(ticklabel* cs:1)},anchor=south west}
		]

		\begin{scope}[cm={-1,.5,2,1,(0,0)}, Blue]
			%\node[transform shape, black] at (.5,.5) {$C_2$};
			%\node[black] at (.5,.5) {$\mathcal S(C_2)$};
			\fill[opacity=.3] (0,0) rectangle (1,1);
			\draw[very thick, ->] (0,0) -- (1,0) node[midway, below left, yshift=4pt] {$\mathcal S(\xhat)$};;
			\draw[very thick, densely dotted, ->] (0,0) -- (0,1) node[midway, below right] {$\mathcal S(\yhat)$};
		\end{scope}
			\coordinate (B) at (-.25,1.25);
		\end{axis}

		\draw[thick] (A) edge[bend left, ->] node[midway,above] {$\mathcal S$} (B) ;
		\node at ($(plot2.north)+(0,.5cm)$) {Orientation Reversing};
	\end{tikzpicture}
\end{center}

In the figure above, $\mathcal T$ is orientation preserving and $\mathcal S$ is orientation reversing.

For an arbitrary linear transformation $\mathcal Q:\R^n\to\R^n$ and a set $X\subseteq\R^n$, we
define the \emph{oriented volume} of $\mathcal Q(X)$ to be $+\Vol\mathcal Q(X)$ if $\mathcal Q$ is
orientation preserving and $-\Vol\mathcal Q(X)$ if $\mathcal Q$ is orientation reversing.


\begin{example}
	Let $\mathcal T:\R^2\to\R^2$ be defined by $\mathcal T\mat{x\\y}=\matc{2x+y\\-x+\tfrac{1}{2}y}$.
	Find $\det(\mathcal T)$.

	This is the same $\mathcal T$ as from the previous example where we computed
	$\Vol\mathcal T(C_2)=2$. Since $\mathcal T$ is orientation preserving, we conclude that $\det(\mathcal T)=2$.
\end{example}
\begin{example}
	Let $\mathcal S:\R^2\to\R^2$ be defined by $\mathcal S\mat{x\\y}=\mat{-x+y\\x+y}$.
	Find $\det(\mathcal S)$.

	By drawing a picture, we see that $\mathcal S(C_2)$ is a square and $\Vol\mathcal S(C_2)=2$.
	However, $\mathcal S(\xhat) = \mat{-1\\1}$ and $\mathcal S(\yhat) = \mat{1\\1}$
	form a negatively oriented basis, and so $\mathcal S$ is orientation reversing. Therefore,
	$\det(\mathcal S) = - \Vol\mathcal S(C_2) = -2$.
\end{example}
\begin{example}
	Let $\mathcal P:\R^2\to\R^2$ be projection onto the line with equation $x+2y=4$. Find $\det(\mathcal P)$.

	Because $\mathcal P$ projects everything to a line, we know $\mathcal P(C_2)$ must be a line segment and
	therefore has volume zero. Thus $\det(\mathcal P)=0$.
\end{example}


\Heading{Determinants of Composition}\index{Linear transformation!composition of}

Volume changes are naturally multiplicative. If a linear transformation $\mathcal T$ changes volume by a factor
of $\alpha$ and $\mathcal S$ changes volume by a factor of $\beta$, then $\mathcal S\circ \mathcal T$ changes
volume by a factor of $\beta\alpha$. Thus, determinants must also be multiplicative\footnote{ To fully argue this, we need
to show that the composition of two orientation-reversing transformations is orientation preserving.}.

\begin{center}
	\begin{tikzpicture}
		\begin{axis}[
		    anchor=origin,
		    name=plot1,
		    disabledatascaling,
		    xmin=-1,xmax=2,
		    ymin=0,ymax=3,
			xtick={-2,...,4},
			ytick={-2,...,4},
			xticklabels={,,},
			yticklabels={,,},
		    x=.7cm,y=.7cm,
		    grid=both,
		    grid style={line width=.1pt, draw=gray!10},
		    %major grid style={line width=.2pt,draw=gray!50},
		    axis lines=middle,
		    minor tick num=0,
		    enlargelimits={abs=0.5},
		    axis line style={latex-latex},
		    ticklabel style={font=\tiny,fill=white},
		    xlabel style={at={(ticklabel* cs:1)},anchor=north west},
		    ylabel style={at={(ticklabel* cs:1)},anchor=south west}
		]

		\begin{scope}[cm={1,0,0,1,(0,0)}, myorange]
			%\node[transform shape, black] at (.5,.5) {$C_2$};
			%\fill[opacity=.3] (0,0) rectangle (1,1);
			\draw[fill, opacity=.3] plot [smooth cycle] coordinates {(0,0) (1,0.1) (2,0.3) (2,1.4) (1.5,2.5) (0.8,2.5) (0.3,1.2) (-0.2,0.6)};
			\draw plot [smooth cycle] coordinates {(0,0) (1,0.1) (2,0.3) (2,1.4) (1.5,2.5) (0.8,2.5) (0.3,1.2) (-0.2,0.6)};
			\draw[black!70, very thick, ->] (0,0) -- (1,0);
			\draw[black!70, very thick, densely dotted, ->] (0,0) -- (0,1);
		\end{scope}
			\coordinate (A) at (1.25,3.25);
			\coordinate (X) at (1.25,-.25);
		\end{axis}
		\begin{axis}[
			name=plot2,
			at={($(plot1.east) + (1cm, 0)$)},
			anchor=west,
		    disabledatascaling,
		    xmin=-1,xmax=4,
		    ymin=0,ymax=3,
			xtick={-2,...,4},
			ytick={-2,...,4},
			xticklabels={,,},
			yticklabels={,,},
		    x=.7cm,y=.7cm,
		    grid=both,
		    grid style={line width=.1pt, draw=gray!10},
		    %major grid style={line width=.2pt,draw=gray!50},
		    axis lines=middle,
		    minor tick num=0,
		    enlargelimits={abs=0.5},
		    axis line style={latex-latex},
		    ticklabel style={font=\tiny,fill=white},
		    xlabel style={at={(ticklabel* cs:1)},anchor=north west},
		    ylabel style={at={(ticklabel* cs:1)},anchor=south west}
		]

		\begin{scope}[cm={2,1,-1,.5,(0,0)}, BlueGreen]
			%\node[transform shape, black] at (.5,.5) {$C_2$};
			%\node[black] at (.5,.5) {$\mathcal T(C_2)$};
			\draw[fill, opacity=.3] plot [smooth cycle] coordinates {(0,0) (1,0.1) (2,0.3) (2,1.4) (1.5,2.5) (0.8,2.5) (0.3,1.2) (-0.2,0.6)};
			\draw plot [smooth cycle] coordinates {(0,0) (1,0.1) (2,0.3) (2,1.4) (1.5,2.5) (0.8,2.5) (0.3,1.2) (-0.2,0.6)};
			\draw[black!70, very thick, ->] (0,0) -- (1,0);
			\draw[black!70, very thick, densely dotted, ->] (0,0) -- (0,1);
		\end{scope}
			\coordinate (B) at (-.25,3.25);
			\coordinate (C) at (1.5,3.25);
		\end{axis}
		
		\begin{axis}[
			name=plot3,
			at={($(plot2.south east) + (1cm, 0)$)},
			anchor=south west,
		    disabledatascaling,
		    xmin=-1,xmax=6,
		    ymin=0,ymax=5,
			xtick={-2,...,6},
			ytick={-2,...,6},
			xticklabels={,,},
			yticklabels={,,},
		    x=.7cm,y=.7cm,
		    grid=both,
		    grid style={line width=.1pt, draw=gray!10},
		    %major grid style={line width=.2pt,draw=gray!50},
		    axis lines=middle,
		    minor tick num=0,
		    enlargelimits={abs=0.5},
		    axis line style={latex-latex},
		    ticklabel style={font=\tiny,fill=white},
		    xlabel style={at={(ticklabel* cs:1)},anchor=north west},
		    ylabel style={at={(ticklabel* cs:1)},anchor=south west}
		]

		\begin{scope}[cm={3, 0, -1, 2,(0,0)}, mypink]
			%\node[transform shape, black] at (.5,.5) {$C_2$};
			%\node[black] at (.5,.5) {$\mathcal T(C_2)$};
			\draw[fill, opacity=.3] plot [smooth cycle] coordinates {(0,0) (1,0.1) (2,0.3) (2,1.4) (1.5,2.5) (0.8,2.5) (0.3,1.2) (-0.2,0.6)};
			\draw plot [smooth cycle] coordinates {(0,0) (1,0.1) (2,0.3) (2,1.4) (1.5,2.5) (0.8,2.5) (0.3,1.2) (-0.2,0.6)};
			\draw[black!70, very thick, ->] (0,0) -- (1,0);
			\draw[black!70, very thick, densely dotted, ->] (0,0) -- (0,1);
		\end{scope}
			\coordinate (D) at (-.75,3.25);
			\coordinate (Y) at (1.25,-.25);
		\end{axis}

		\draw[thick] (A) edge[bend left, ->] node[midway,above] {$2\times\text{Volume}$} node[midway,below] {$\mathcal T$} (B) ;
		\draw[thick] (C) edge[bend left, ->] node[midway,above] {$3\times\text{Volume}$} node[midway,below] {$\mathcal S$}  (D) ;
		\draw[thick] (X) edge[bend right=20, ->] node[midway,above] {$6\times\text{Volume}$} node[midway,below] {$\mathcal S\circ \mathcal T$}  (Y) ;
	\end{tikzpicture}
\end{center}

\begin{theorem}
	Let $\mathcal T:\R^n\to\R^n$ and $\mathcal S:\R^n\to\R^n$ be linear transformations. Then
	\[
		\det(\mathcal S\circ \mathcal T)=\det(\mathcal S)\det(\mathcal T).
	\]
\end{theorem}

This means that we can compute the determinant of a complicated transformation by breaking it up
into simpler ones and computing the determinant of each piece. 

\Heading{Determinants of Matrices}\index{Matrix!elementary matrix}

The determinant of a matrix is defined as the determinant of its induced transformation.
That means, the determinant is multiplicative with respect to matrix multiplication (because
it's multiplicative with respect to function composition).

\begin{theorem}
	Let $A$ and $B$ be $n\times n$ matrices. Then
	\[
		\det(AB)=\det(A)\det(B).
	\]
\end{theorem}

We will derive an algorithm for finding the determinant of a matrix by considering the determinant of
elementary matrices. But first, consider the following theorem.

\begin{theorem}[Volume Theorem I]\index{Volume!theorem I}
	For a square matrix $M$, $\det(M)$ is the oriented volume of the parallelepiped\footnote{ A parallelepiped is the
	$n$-dimensional analog of a parallelogram.} given by the column
	vectors.
\end{theorem}
\begin{proof}
	Let $M$ be an $n\times n$ matrix and let $\mathcal T_M$ be its induced transformation. We know the sides of
	$\mathcal T_{M}(C_n)$ are given by $\Set{\mathcal T_M(\vec e_1),\ldots,\mathcal T_M(\vec e_n)}$. And, by definition,
	\[
		[\mathcal T_M(\vec e_i)]_{\mathcal E} = M[\vec e_i]_{\mathcal E} = \text{ $i$th column of $M$.}
	\]
	Therefore $\mathcal T_{M}(C_n)$ is the parallelepiped whose sides are given by the columns of $M$.
\end{proof}

This means we can think about the determinant of a matrix by considering its columns. Now we are ready to consider
the determinants of the elementary matrices!

\bigskip
There are three types of elementary matrices corresponding to the three elementary row operations. For each
one, we need to understand how the induced transformation changes volume.

{\bfseries Multiply a row by a non-zero constant $\alpha$.} Let $E_m$ be such an elementary matrix. 
Scaling one row of $I$ is equivalent to scaling one column of $I$, and so the columns of $E_m$ specify a parallelepiped
that is scaled by $\alpha$ in one direction.

For example, if
\[
	E_m=\mat{1&0&0\\0&1&0\\0&0&\alpha}\qquad\text{then}\qquad \Set{\vec e_1,\vec e_2,\vec e_3}\mapsto\Set{\vec e_1,\vec e_2,\alpha\vec e_3}.
\]

Thus $\det(E_m)=\alpha$.

{\bfseries Swap two rows.} Let $E_s$ be such an elementary matrix. Swapping two rows of $I$ is equivalent to swapping
two columns of $I$, so $E_s$ is $I$ with two columns swapped. This reverses the orientation of the basis given by the columns.

For example, if
\[
	E_s=\mat{0&1&0\\1&0&0\\0&0&1}\qquad\text{then}\qquad \Set{\vec e_1,\vec e_2,\vec e_3}\mapsto\Set{\vec e_2,\vec e_1,\vec e_3}.
\]

Thus $\det(E_s)=-1$.

{\bfseries Add a multiple of one row to another.} Let $E_a$ be such an elementary matrix. The columns of $E_a$ are the same as the columns
of $I$ except that one column where $\vec e_i$ is replaced with $\vec e_i+\alpha\vec e_j$. This has the effect of \emph{shearing} $C_n$ in
the $\vec e_j$ direction.


\begin{center}
	\usetikzlibrary{patterns,decorations.pathreplacing}
	\begin{tikzpicture}
		\coordinate (A) at (2,0);
		\coordinate (B) at (0,2);
		\coordinate (O) at (0,0);

		%\draw [mypink,fill] (A) circle (1.5pt) node [right] {initial point};
		%\draw [mypink,fill] (B) circle (1.5pt) node [left] {terminal point};
		\draw[fill,gray!30!white] (O) -- +(B) -- +($(A)+(B)$) -- +(A) -- (O);
		\draw[->,thick,myred!60!white] (0,0) -- +(A) node [midway,below] {$\vec e_j$};
		\draw[->,thick,mypink] (0,0) -- +(B) node [midway,above left] {$\vec e_i$};
		\draw[black,dashed,<->] ($.1*(A)$) -- ($.1*(A)+(0,2)$) node [midway,right] {$\text{height}=1$};
	\end{tikzpicture}
	\hspace{1cm}
	\begin{tikzpicture}
		\coordinate (A) at (2,0);
		\coordinate (B) at (0,2);
		\coordinate (C) at ($(B)+2*(A)$);
		\coordinate (O) at (0,0);

		%\draw [mypink,fill] (A) circle (1.5pt) node [right] {initial point};
		%\draw [mypink,fill] (B) circle (1.5pt) node [left] {terminal point};
		\draw[fill,gray!30!white] (O) -- +(C) -- +($(A)+(C)$) -- +(A) -- (O);
		\draw[->,thick,myred!60!white] (0,0) -- +(A) node [midway,below] {$\vec e_j$};
		\draw[->,thick,gray,dashed] (0,0) -- +(B);% node [midway,above left] {$\vec b$};
		\draw[->,thick,gray,dashed] (B) -- +($2*(A)$);% node [midway,above] {$2\vec a$};
		\draw[->,thick,mypink] (0,0) -- +(C) node [midway,above left] {$\vec e_i+\alpha\vec e_j$};
		\draw[black,dashed,<->] ($2.3*(A)$) -- ($2.3*(A)+(0,2)$) node [midway,right] {$\text{height}=1$};
	\end{tikzpicture}
\end{center}

Since $C_n$ is sheared in a direction parallel to one of its other sides, its volume is not changed. Thus $\det(E_a)=1$.

\begin{emphbox}[Takeaway]
	The determinants of elementary matrices are all easy to compute and the determinant of the
	most-used type of elementary matrix is $1$.
\end{emphbox}

Now, by decomposing a matrix into the product of elementary matrices, 
we can use the multiplicative property of the determinant (and the formulas for the determinants
of the different types of elementary matrices) to compute the determinant of an invertible
matrix.

\begin{example}
	Use elementary matrices to find the determinant of $A=\mat{1&2\\3&4}$.

	We can row-reduce $A$ with the following steps.
	\[
	    \mat{1&2\\3&4}\to \mat{1&2\\0&-2}\to \mat{1&2\\0&1}\to \mat{1&0\\0&1}.
	\]
	The elementary matrices corresponding to these steps are
	\[
		E_1=\mat{1&0\\-3&1} \qquad E_2=\mat{1&0\\0&-\frac{1}{2}}\qquad\text{and}\qquad E_3=\mat{1&-2\\0&1}, 
	\]
	and so $E_3 E_2 E_1 A = I$. Therefore
	\[
	    A=E_1^{-1}E_2^{-1}E_3^{-1}I=E_1^{-1}E_2^{-1}E_3^{-1}=
	    \mat{1&0\\3&1}\mat{1&0\\0&-2}\mat{1&2\\0&1}.
	\]
	Using the fact that the determinant is multiplicative, we get
	\begin{align*}
	    \det(A)&=\det\left(\mat{1&0\\3&1}\mat{1&0\\0&-2}\mat{1&2\\0&1}\right)\\
	           &=\det\left(\mat{1&0\\3&1}\right)\det\left(\mat{1&0\\0&-2}\right)\det\left(\mat{1&2\\0&1}\right)\\
		   &=(1)(-2)(1) = -2.
	\end{align*}
\end{example}

\Heading{Determinants and Invertibility}

We can use elementary matrices to compute the determinant of any invertible matrix by
decomposing it into the product of elementary matrices. But, what about non-invertible matrices?

Let $M$ be an $n\times n$ matrix that is \emph{not} invertible. Then, we must have $\Nullity(M)>0$
and $\Dim(\Col(M))=\Rank(M)<n$. Geometrically, this means there is at least one line of vectors, $\Null(M)$, that gets
collapsed to $\vec 0$, and the column space of $M$ must be ``flattened'' (i.e., it has lost a dimension). 
Therefore, the volume of the parallelepiped given by the columns of $M$ must be zero, and so $\det(M)=0$.

Based on this argument, we have the following theorem.
\begin{theorem}
	Let $A$ be an $n\times n$ matrix. $A$ is invertible\index{Matrix!invertible} if and only if $\det(A)\neq 0$.
\end{theorem}
\begin{proof}
	If $A$ is invertible, $A=E_1\cdots E_k$, where $E_1,\ldots,E_k$ are elementary matrices,
	and so
	\[
		\det(A)=\det(E_1\cdots E_k) = \det(E_1)\cdots \det(E_k).
	\]
	All elementary matrices have non-zero determinants,
	and so $\det(A)\neq 0$.

	Conversely, if $A$ is not invertible, $\Rank(A)<n$, which means the parallelepiped 
	given by the columns of $A$ is ``flattened'' and has zero volume.
\end{proof}

We now have another way to tell if a matrix is invertible! But, for an invertible matrix $A$,
how do $\det(A)$ and $\det(A^{-1})$ relate?  Well, by definition
\[
	AA^{-1}=I,
\]
and so
\[
	\det(AA^{-1})=\det(A)\det(A^{-1})=
	\det(I)=1,
\]
which gives
\[
	\det(A^{-1})=\frac{1}{\det(A)}.
\]

\Heading{Determinants and Transposes}

Somewhat mysteriously, we have the following theorem.
\begin{theorem}[Volume Theorem II]\index{Volume!theorem II}
	The determinant of a square matrix $A$ is equal to the oriented volume of the parallelepiped
	given by the rows of $A$.
\end{theorem}

Volume theorem II can be concisely stated as $\det(A)=\det(A^T)$, and joins other
strange transpose-related facts (like $\Rank(A)=\Rank(A^T)$).

We can prove volume theorem II using elementary matrices.
\begin{proof}
	Suppose $A$ is not invertible. Then, neither is $A^T$ and so $\det(A)=\det(A^T)=0$.

	Suppose $A$ is invertible and $A=E_1\cdots E_k$ where $E_1,\ldots, E_k$ are elementary
	matrices. We then have
	\[
		A^T = E_k^T\cdots E_1^T,
	\]
	which follows from the fact that the transpose reverses the order of matrix multiplication
	(i.e., $(XY)^T=Y^TX^T$).
	However, for each $E_i$, we may observe that $E_i^T$ is another elementary matrix
	of the same type and with the same determinant. Therefore,
	\begin{align*}
		\det(A^T) = \det(E_k^T\cdots E_1^T)&=\det(E_k^T)\cdots \det(E_i^T)\\ 
		&=
		\det(E_k)\cdots \det(E_1)\\ &= \det(E_1)\cdots \det(E_k) = \det(E_1\cdots E_k)=\det(A).
	\end{align*}
	The key observations for this proof are that (i) $\det(E_i^T)=\det(E_i)$ and (ii) since the 
	$\det(E_i)$'s are just scalars, the order in which they are multiplied doesn't matter.
\end{proof}
