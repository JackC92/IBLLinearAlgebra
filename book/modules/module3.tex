Let $\vec u=\mat{1\\1}$ and $\vec v=\mat{1\\-2}$. Can the vectors $\vec w=\mat{2\\5}$ be obtained
as a linear combination of $\vec u$ and $\vec v$?

By drawing a picture, the answer appears to be \emph{yes}.

\begin{center}
	\begin{tikzpicture}
		\begin{axis}[
		    anchor=origin,
		    disabledatascaling,
		    xmin=-1,xmax=3,
		    ymin=-2,ymax=5,
			xtick={-2,...,3},
		    x=1cm,y=1cm,
		    grid=both,
		    grid style={line width=.1pt, draw=gray!10},
		    %major grid style={line width=.2pt,draw=gray!50},
		    axis lines=middle,
		    minor tick num=0,
		    enlargelimits={abs=0.5},
		    axis line style={latex-latex},
		    ticklabel style={font=\tiny,fill=white},
		    xlabel style={at={(ticklabel* cs:1)},anchor=north west},
		    ylabel style={at={(ticklabel* cs:1)},anchor=south west}
		]

			\draw [myorange, thick, ->] (0,0) -- (1,1) node[midway, below] {$\vec u$};
			\draw [mypink, thick, ->] (0,0) -- (1,-2) node[midway, below left] {$\vec v$};
			\draw [mygreen, thick, ->] (0,0) -- (2,5) node[midway, above left] {$\vec w$};
			%\draw [black!70!white, thick,dashed, ->, xshift=-.15cm] (0,0) -- (0,1) node[midway, left] {$\yhat$};
		\end{axis}
	\end{tikzpicture}
	~~~~
	\begin{tikzpicture}
		\begin{axis}[
		    anchor=origin,
		    disabledatascaling,
		    xmin=-1,xmax=3,
		    ymin=-2,ymax=5,
			xtick={-2,...,3},
		    x=1cm,y=1cm,
		    grid=both,
		    grid style={line width=.1pt, draw=gray!10},
		    %major grid style={line width=.2pt,draw=gray!50},
		    axis lines=middle,
		    minor tick num=0,
		    enlargelimits={abs=0.5},
		    axis line style={latex-latex},
		    ticklabel style={font=\tiny,fill=white},
		    xlabel style={at={(ticklabel* cs:1)},anchor=north west},
		    ylabel style={at={(ticklabel* cs:1)},anchor=south west}
		]

			\draw [black!70!white, thick, ->] (0,0) -- (1,1);
			\draw [black!70!white, thick, ->] (1,1) -- (2,2) node[midway, below right] {$3\vec u$};
			\draw [black!70!white, thick, ->] (2,2) -- (3,3);
			\draw [black!70!white, thick, ->] (3,3) -- (2,5) node[midway, below left] {$-\vec v$};
			\draw [mygreen, thick, ->] (0,0) -- (2,5) node[midway, above left] {$\vec w$};
			%\draw [black!70!white, thick,dashed, ->, xshift=-.15cm] (0,0) -- (0,1) node[midway, left] {$\yhat$};
		\end{axis}
	\end{tikzpicture}
\end{center}

Algebraically, we can use the definition of a \emph{linear combination} to set up a system of equations.
We know $\vec w$ can be expressed as a linear combination of $\vec u$ and $\vec v$ if and only if 
the vector equation
\[
	\vec w = \mat{2\\5}=\alpha\mat{1\\1}+\beta\mat{1\\-2}=\alpha \vec u+\beta \vec v
\]
has a solution. By inspection, we see $\alpha=3$ and $\beta=-1$ solve this equation.

After initial success, we might ask the following:
\emph{what are all the locations in $\R^2$ that can be obtained
as a linear combination of $\vec u$ and $\vec v$?} Geometrically, it appears
any location can be reached. To verify this algebraically, consider the vector equation
\begin{equation}
	\label{EQSPAN1}
	\vec x=\mat{x\\y} = \alpha\mat{1\\1}+\beta\mat{1\\-2} = \alpha\vec u+\beta\vec v.
\end{equation}
Here $\vec x$ represents an arbitrary point in $\R^2$. If equation \eqref{EQSPAN1} always
has a solution\footnote{ The official terminology would be to say that
the equations are always \emph{consistent}.}, any vector in $\R^2$ can be obtained as a linear combination of $\vec u$ and $\vec v$.

We can solve this equation for $\alpha$ and $\beta$ by considering the equations arising from the
first and second coordinates. Namely,
	\[
		\sysdelim..\systeme*{x=\alpha+\beta, y=\alpha-2\beta}
	\]
Subtracting the second equation from the first, we get $x-y=3\beta$ and so $\beta=(x-y)/3$. Plugging 
$\beta$ into the first equation and solving, we get $\alpha=(2x+y)/3$. Thus, equation \eqref{EQSPAN1}
\emph{always} has the solution
\begin{align*}
	\alpha &= \tfrac{1}{3}(2x+y)\\
	\beta &= \tfrac{1}{3}(x-y).
\end{align*}

There is a formal term for the set of vectors that can be obtained as linear combinations
of others: \emph{span}\index{Span}.

\SavedDefinitionRender{Span}

We just showed above that $\Span\Set*{\mat{1\\1},\mat{1\\-2}}=\R^2$.

\begin{example}
	Let $\vec u=\mat{-1\\2}$ and $\vec v=\mat{1\\-2}$. Find $\Span\Set{\vec u,\vec v}$.

	By the definition of span,
	\[
		\Span\Set{\vec u,\vec v}=
		\Set{\vec x\given \vec x=\alpha\vec u+\beta\vec v\text{ for some }\alpha,\beta\in\R}.
	\]
	We need to determine for which $x$ and $y$ the vector equation $\mat{x\\y} = \alpha\mat{-1\\2}+\beta\mat{1\\-2}$ is consistent.
	
	From the first and second coordinates, we get the system
	\[
		\sysdelim..\systeme*{x=-\alpha+\beta, y=2\alpha-2\beta}.
	\]
	Adding 2 times the first equation to the second, we get $2x+y=0$ and so $y=-2x$.
	Therefore, if $\mat{x\\y}$ make the above system consistent, we must have
	\[
		\mat{x\\y}=\mat{t\\-2t}=t\vec v
	\]
	for some $t$.
	Thus,
	\[
		\Span\Set{\vec u,\vec v}=
		\Set{\vec x\given \vec x=t\vec v\text{ for some }t}=\Span\Set{\vec v},
	\]
	which is a line through the origin with direction $\vec v$.
\end{example}

\begin{example}
	Let $\vec a=\mat{1\\2\\1}$, $\vec b=\mat{0\\1\\0}$, and $\vec c=\mat{1\\1\\2}$. Show that
	$\R^3=\Span\Set{\vec a,\vec b,\vec c}$.

	If the equation 
	\[
		\vec x=\mat{x\\y\\z} = \alpha_1\mat{1\\2\\1}+\alpha_2\mat{0\\1\\0}+\alpha_3\mat{1\\1\\2}
		 = \alpha_1\vec a+\alpha_2\vec b+\alpha_3\vec c
	\]
	is always consistent, then any vector in $\R^3$ can be obtained as a linear combination of
	$\vec a, \vec b$, and $\vec c$.
	
	Reading off the coordinates, we get the system
	\[
		\sysdelim..\systeme*{x=\alpha_1+\alpha_3, y=2\alpha_1+\alpha_2+\alpha_3, z=\alpha_1+2\alpha_3}.
	\]
	Solving this system, we see
	\[
		\begin{aligned}
			\alpha_1&=2x-z\\
			\alpha_2&=-3x+y+z\\
			\alpha_3&=-x+z
		\end{aligned}
	\]
	is always a solution (no matter the values of $x$, $y$, and $z$). Therefore,
	$\Span\Set{\vec a,\vec b,\vec c}=\R^3$.
\end{example}

\Heading{Representing Lines \& Planes as Spans}

If spans remind you of vector forms of lines and planes, your intuition is keen.
Consider the line $\ell$ given in vector form by
\[
	\vec x=t\vec d+\vec 0.
\]
The line $\ell$ passes through the origin, and if we unravel its definition, we see
\[
	\ell=\Set{\vec x\given \vec x=t\vec d+\vec 0\text{ for some }t\in \R}
	=\Set{\vec x\given \vec x=t\vec d\text{ for some }t\in \R}=\Span\Set{\vec d}.
\]

Similarly, if $\mathcal P$ is a plane given in vector form by
\[
	\vec x=t\vec d_1+s\vec d_2+\vec 0,
\]
then
\[
	\mathcal P=
	\Set{\vec x\given \vec x=t\vec d_1+s\vec d_2\text{ for some }t,s\in \R}=\Span\Set{\vec d_1,\vec d_2}.
\]

If the ``$\vec p$'' in our vector form is $\vec 0$, then that vector form actually defines
a span. This means (if you accept that every line/plane through the origin has a
vector form) that every line/plane through the origin can be written as a span. Conversely,
if $X=\Span\Set{\vec v_1,\ldots,\vec v_n}$ is a span, we know $\vec 0=0\vec v_1+\cdots+0\vec v_n\in X$,
and so every span passes through the origin.


As it turns out, spans exactly describe points, lines, planes, and volumes\footnote{
 We use the word \emph{volume} to indicate the higher-dimensional analogue of a plane.} through the origin.

\begin{example}
	The line $\ell_1\subseteq\R^2$ is described by the equation $x+2y=0$ and the line 
	$\ell_2\subseteq \R^2$ is described by the equation $4x-2y=6$.
	If possible, describe $\ell_1$ and $\ell_2$ using spans.

	We can express $\ell_1$ in vector form by
	\[
		\vec x=t\mat{2\\-1}+\vec 0,
	\]
	and so 
	\[
		\ell_1=\Span\Set*{\mat{2\\-1}}.
	\]

	However, $\ell_2$ does not pass through $\vec 0$, and so $\ell_2$ cannot be written as a span.
\end{example}

\begin{emphbox}[Takeaway]
	Lines and planes through the origin, and only lines and planes through the origin, can be expressed as spans.
\end{emphbox}

\Heading{Set Addition}
We're going to work around the fact that only objects which pass through the origin can be written as
spans, but first let's take a detour and learn about \emph{set addition}.

\SavedDefinitionRender{SetAddition}

Set sums are very different than regular sums despite using the same symbol, ``$+$''\footnote{
For example, $A+\Set{}=\Set{}$, which might seem counterintuitive for an ``addition'' operation.
}.
 However, they are very useful.
Let $C=\Set{\vec x\in\R^2\given \norm{\vec x}=1}$ be the unit circle centered at the origin, and consider
the sets
\[
	X=C+\Set{\yhat}\qquad Y=C+\Set{3\xhat,\yhat}\qquad Z=C+\Set{\vec 0, \xhat,\yhat}.
\]
Rewriting, we see $X=\Set{\vec x+\yhat\given \norm{\vec x}=1}$ is just $C$ translated
by $\yhat$. Similarly, $Y=\Set{\vec x+\vec v\given \norm{\vec x}=1\text{ and }\vec v=3\xhat\text{
	or }\vec v=\yhat}=(C+\Set{3\xhat})\cup (C+\Set{\yhat})$, and so $Y$ is the union
of two translated copies of $C$\footnote{ If you want to stretch your mind, consider what $C+C$
is as a set.}. Finally, $Z$ is the union of three translated copies of $C$.

\begin{center}
	\begin{tikzpicture}
		\begin{axis}[
		    anchor=origin,
		    disabledatascaling,
		    xmin=-1,xmax=1,
		    ymin=-1,ymax=2,
			xtick={-2,...,4},
		    x=1cm,y=1cm,
		    grid=both,
		    grid style={line width=.1pt, draw=gray!10},
		    %major grid style={line width=.2pt,draw=gray!50},
		    axis lines=middle,
		    minor tick num=0,
		    enlargelimits={abs=0.5},
		    axis line style={latex-latex},
		    ticklabel style={font=\tiny,fill=white},
		    xlabel style={at={(ticklabel* cs:1)},anchor=north west},
		    ylabel style={at={(ticklabel* cs:1)},anchor=south west}
		]

			\draw[myorange, thick] (0,1) circle (1cm) node[right] {$X$};
		\end{axis}
	\end{tikzpicture}
	~~~~
	\begin{tikzpicture}
		\begin{axis}[
		    anchor=origin,
		    disabledatascaling,
		    xmin=-1,xmax=4,
		    ymin=-1,ymax=2,
			xtick={-2,...,4},
		    x=1cm,y=1cm,
		    grid=both,
		    grid style={line width=.1pt, draw=gray!10},
		    %major grid style={line width=.2pt,draw=gray!50},
		    axis lines=middle,
		    minor tick num=0,
		    enlargelimits={abs=0.5},
		    axis line style={latex-latex},
		    ticklabel style={font=\tiny,fill=white},
		    xlabel style={at={(ticklabel* cs:1)},anchor=north west},
		    ylabel style={at={(ticklabel* cs:1)},anchor=south west}
		]

			\draw[mypink, thick] (0,1) circle (1cm) (3,0) circle (1cm) node[above right] {$Y$};
		\end{axis}
	\end{tikzpicture}
	~~~~
	\begin{tikzpicture}
		\begin{axis}[
		    anchor=origin,
		    disabledatascaling,
		    xmin=-1,xmax=2,
		    ymin=-1,ymax=2,
			xtick={-2,...,4},
		    x=1cm,y=1cm,
		    grid=both,
		    grid style={line width=.1pt, draw=gray!10},
		    %major grid style={line width=.2pt,draw=gray!50},
		    axis lines=middle,
		    minor tick num=0,
		    enlargelimits={abs=0.5},
		    axis line style={latex-latex},
		    ticklabel style={font=\tiny,fill=white},
		    xlabel style={at={(ticklabel* cs:1)},anchor=north west},
		    ylabel style={at={(ticklabel* cs:1)},anchor=south west}
		]

			\draw[mygreen, thick] (0,0) circle (1cm) (0,1) circle (1cm) (1,0) circle (1cm) node[above right] {$Z$};
		\end{axis}
	\end{tikzpicture}
\end{center}

\Heading{Translated Spans}

Set addition allows us to easily create parallel lines and planes by translation. For example,
consider the lines $\ell_1$ and $\ell_2$ given in vector form as $\vec x=t\vec d$ and $\vec x=t\vec d+\vec p$,
respectively, where  $\vec d=\mat{2\\1}$ and $\vec p=\mat{-1\\1}$.  These lines
differ from each other by a translation. That is, every point in $\ell_2$ can be obtained by adding $\vec p$ to
a corresponding point in $\ell_1$. Using the idea of set addition, we can express this relationship
by the equation
\[
	\ell_2 = \ell_1+\Set{\vec p}.
\]

\begin{center}
	\begin{tikzpicture}
		\begin{axis}[
		    anchor=origin,
		    disabledatascaling,
		    xmin=-1,xmax=4,
		    ymin=-1,ymax=3,
			xtick={-2,...,4},
		    x=1cm,y=1cm,
		    grid=both,
		    grid style={line width=.1pt, draw=gray!10},
		    %major grid style={line width=.2pt,draw=gray!50},
		    axis lines=middle,
		    minor tick num=0,
		    enlargelimits={abs=0.5},
		    axis line style={latex-latex},
		    ticklabel style={font=\tiny,fill=white},
		    xlabel style={at={(ticklabel* cs:1)},anchor=north west},
		    ylabel style={at={(ticklabel* cs:1)},anchor=south west}
		]

			\draw[myorange,very thick] (-2,-1) -- (6,3) node[midway, below right] {$\ell_1$};
			\draw[mygreen, thick] (-3,0) -- (5,4) node[near end, above left] {$\ell_2=\ell_1+\Set{\vec p}$};
			\draw[black!50!white, thick,dotted, ->] (0,0) -- +(-1,1) node[midway, below left, yshift=2pt] {$\vec p$};
		\end{axis}
	\end{tikzpicture}
	~~~~
	\begin{tikzpicture}
		\begin{axis}[
		    anchor=origin,
		    disabledatascaling,
		    xmin=-1,xmax=4,
		    ymin=-1,ymax=3,
			xtick={-2,...,4},
		    x=1cm,y=1cm,
		    grid=both,
		    grid style={line width=.1pt, draw=gray!10},
		    %major grid style={line width=.2pt,draw=gray!50},
		    axis lines=middle,
		    minor tick num=0,
		    enlargelimits={abs=0.5},
		    axis line style={latex-latex},
		    ticklabel style={font=\tiny,fill=white},
		    xlabel style={at={(ticklabel* cs:1)},anchor=north west},
		    ylabel style={at={(ticklabel* cs:1)},anchor=south west}
		]

			\draw[myorange, very thick] (-2,-1) -- (6,3) node[pos=.15, below right] {$\ell_1$};
			\draw[black, thick, dashed, ->] (0,0) -- (3,1.5) node[midway, below right] {$\tfrac{3}{2}\vec d\in\ell_1$};
			\draw[mygreen, thick] (-3,0) -- (5,4);% node[near end, above left] {$\ell_1+\Set{\vec p}$};
			\draw[black!50!white, thick,dotted, ->] (0,0) -- +(-1,1);
			\draw[black!50!white, thick,dotted, ->] (-1,-1/2) -- +(-1,1);
			\draw[black!50!white, thick,dotted, ->] (1,1/2) -- +(-1,1);
			\draw[black!50!white, thick,dotted, ->] (2,1) -- +(-1,1);
			\fill[black!60!white] (2,2.5) circle (1.5pt);
			\draw[black, thick,dashed, ->] (3,3/2) -- +(-1,1) node[midway, above right] {$\vec p$} 
			node[above left]{$\tfrac{3}{2}\vec d+\vec p\in\ell_2$};
			\draw[black!50!white, thick,dotted, ->] (4,2) -- +(-1,1);
		\end{axis}
	\end{tikzpicture}
\end{center}

Note: it would be incorrect to write ``$\ell_2=\ell_1+\vec p$''. Because $\ell_1$
is a set and $\vec p$ is not a set, ``$\ell_1+\vec p$'' does not make mathematical sense.

\begin{example}
	Recall $\ell_2\subseteq\R^2$ is the line described by the equation $4x-2y=6$.
	Describe $\ell_2$ as a translated span.

	We can express $\ell_2$ in vector form with the equation
	\[
		\vec x=t\mat{1\\2}+\mat{1\\-1}.
	\]
	Therefore,
	\[
		\ell_2=\Span\Set*{\mat{1\\2}}+\Set*{\mat{1\\-1}}.
	\]
\end{example}

We can now see translated spans provide an alternative notation to vector form 
for specifying lines and planes. If $Q$ is described in vector form by
\[
	\vec x=t\vec d_1+s\vec d_2+\vec p,
\]
then
\[
	Q=\Span\Set{\vec d_1,\vec d_2}+\Set{\vec p}.
\]



\begin{emphbox}[Takeaway]
	All lines and planes, whether through the origin or not, can be expressed as translated
	spans.
\end{emphbox}

\Heading{Linear Independence \& Linear Dependence}

Let
\[
	\vec u=\mat{1\\0\\0}\qquad\vec v=\mat{0\\1\\0}\qquad \vec w=\mat{1\\1\\0}.
\]
Since $\vec w=\vec u+\vec v$, we know that $\vec w\in\Span\Set{\vec u,\vec v}$. Geometrically,
this is also clear because $\Span\Set{\vec u,\vec v}$ is the $xy$-plane in $\R^3$ and 
$\vec w$ lies on that plane.

What about $\Span\Set{\vec u,\vec v,\vec w}$? Intuitively, since $\vec w$ is already
a linear combination of $\vec u$ and $\vec v$, we can't get anywhere \emph{new} by
taking linear combinations of $\vec u$, $\vec v$, \emph{and} $\vec w$ compared to linear combinations
of just $\vec u$ and $\vec v$. So $\Span\Set{\vec u,\vec v}=\Span\Set{\vec u,\vec v,\vec w}$.

Can we prove this from the definitions? Yes! Suppose $\vec r\in \Span\Set{\vec u,\vec v,\vec w}$.
By definition,
\[
	\vec r=\alpha\vec u+\beta\vec v+\gamma\vec w
\] for some $\alpha,\beta,\gamma\in\R$. Since $\vec w=\vec u+\vec v$, we see
\[
	\vec r=\alpha\vec u+\beta\vec v+\gamma(\vec u+\vec v) 
	= (\alpha+\gamma)\vec u+(\beta+\gamma)\vec v\in\Span\Set{\vec u,\vec v}.
\]
Thus, $\Span\Set{\vec u,\vec v,\vec w}\subseteq \Span\Set{\vec u,\vec v}$. Conversely, if $\vec s\in\Span\Set{\vec u,\vec v}$,
by definition,
\[
	\vec s=a\vec u+b\vec v=a\vec u+b\vec v+0\vec w
\]
for some $a,b\in \R$,
and so $\vec s\in\Span\Set{\vec u,\vec v,\vec w}$. Thus $\Span\Set{\vec u,\vec v}\subseteq\Span\Set{\vec u,\vec v,\vec w}$.
We conclude $\Span\Set{\vec u,\vec v}=\Span\Set{\vec u,\vec v,\vec w}$.

In this case, $\vec w$ was a redundant vector---it wasn't needed for the span. When a set contains
a redundant vector, we call the set \emph{linearly dependent}.

\SavedDefinitionRender{LinearlyDependentIndependentGeometric}

We will also refer to sets of vectors (for example $\Set{\vec v_1,\ldots,\vec v_n}$) as being linearly
independent or linearly dependent. For technical reasons, we didn't state the definition in terms
of sets\footnote{ The issue is, every element of a set is unique. Clearly, the vectors $\vec v$ and $\vec v$
are linearly dependent, but $\Set{\vec v,\vec v}=\Set{\vec v}$, and so $\Set{\vec v,\vec v}$ is technically
a linearly independent set. This issue would be resolved by talking about \emph{multisets} instead of sets,
but it isn't worth the hassle.}.

The geometric definition of linear dependence says that the vectors $\vec v_1,\ldots,\vec v_n$ are linearly dependent
if you can remove at least one vector without changing the span. In other words, $\vec v_1,\ldots,\vec v_n$ 
are linearly dependent \emph{if there is a redundant vector}.

\begin{example}
	\label{EXLINDEP}
	Let $\vec a=\mat{1\\2}$, $\vec b=\mat{2\\3}$, $\vec c=\mat{4\\6}$, and $\vec d=\mat{4\\5}$. Determine whether
	$\Set{\vec a,\vec b,\vec c,\vec d}$ is linearly independent or linearly dependent.

	By inspection, we see $\vec c=2\vec b$. Therefore,
	\[
		\Span\Set{\vec a,\vec b,\vec c,\vec d}=\Span\Set{\vec a,\vec b,\vec d},
	\]
	and so $\Set{\vec a,\vec b,\vec c,\vec d}$ is linearly dependent.
\end{example}

\begin{example}
	The planes $\mathcal P$ and $\mathcal Q$ are given in vector form by
	\[
		\overbrace{\vec x=t\mat{1\\2\\1}+s\mat{2\\2\\1}}^{\mathcal P}
		\qquad\text{and}\qquad
		\overbrace{\vec x=t\mat{3\\4\\2}+s\mat{2\\2\\1}}^{\mathcal Q}.
	\]
	Determine if $\mathcal P$ and $\mathcal Q$ are the same plane.

	Let $\vec a_1=\mat{1\\2\\1}$ and $\vec a_2=\mat{2\\2\\1}$ be the direction vectors
	for $\mathcal P$ and let
	$\vec b_1=\mat{3\\4\\2}$ and $\vec b_2=\mat{2\\2\\1}$ be the direction vectors
	for $\mathcal Q$. We know that $\mathcal P=\mathcal Q$ if every direction vector for $\mathcal Q$
	is a linear combination of the direction vectors for $\mathcal P$. In other words, 
	$\mathcal P=\mathcal Q$ if
	\[
		\Set{\vec a_1,\vec a_2,\vec b_1}\qquad\text{and}\qquad \Set{\vec a_1,\vec a_2,\vec b_2}
	\]
	are both linearly dependent sets. Since $\vec a_2=\vec b_2$, clearly $\Set{\vec a_1,\vec a_2,\vec b_2}$
	is linearly dependent. 

	We will now check whether $\Set{\vec a_1,\vec a_2,\vec b_1}$ is linearly dependent. Since $\Set{\vec a_1,\vec a_2}$
	is a linearly independent set, we only need to check if $\vec b_1$ can be written as a linear combination of $\vec a_1$ and $\vec a_2$.
	After solving the system
	\[
		\mat{3\\4\\2}=x\mat{1\\2\\1}+y\mat{2\\2\\1},
	\]
	we see that $\vec b_1=\vec a_1+\vec a_2$, and so $\Set{\vec a_1,\vec a_2,\vec b_1}$ is linearly dependent. Therefore,
	$\mathcal P=\mathcal Q$.
\end{example}

We can also think of linear independence/dependence from an algebraic perspective.
Suppose the vectors $\vec u$, $\vec v$, and $\vec w$ satisfy 
\begin{equation}
	\label{EQLIND}
	\vec w=\vec u+\vec v.
\end{equation}
The set
$\Set{\vec u,\vec v,\vec w}$ is linearly dependent since $\vec w\in\Span\Set{\vec u,\vec v}$,
but equation \eqref{EQLIND} can be rearranged
to get
\begin{equation}
	\label{EQLIND2}
	\vec 0=\vec u+\vec v-\vec w.
\end{equation}
Here we have expressed $\vec 0$ as a linear combination of $\vec u$, $\vec v$, and $\vec w$.
By itself, this is nothing special. After all, we know $\vec 0=0\vec u+0\vec v+0\vec w$
is a linear combination of $\vec u$, $\vec v$, and $\vec w$. However, the right side of equation \eqref{EQLIND2} has non-zero
coefficients, which makes the linear combination \emph{non-trivial}.

\SavedDefinitionRender{TrivialLinearCombination}

We can always write $\vec 0$ as a linear combination of vectors if we let all the coefficients
be zero, but it turns out we can only write $\vec 0$ as a \emph{non-trivial} linear combination
of vectors if those vectors are linearly dependent. This is the inspiration for another definition
of linear independence/dependence.

\SavedDefinitionRender{LinearlyDependentIndependentAlgebraic}

The idea of a ``redundant vector'' coming from the geometric definition of linear dependence
is easy to visualize, but it can be hard to prove things with---checking for linear independence
with the geometric definition involves verifying for every vector that it is not in the span of the others.
The algebraic definition on the other hand is less obvious, but the reasoning is easier.
You only need to analyze solutions to one equation!

\begin{example}
	Let $\vec u=\mat{1\\2}$, $\vec v=\mat{2\\3}$, and $\vec w=\mat{4\\5}$. Use the algebraic definition
	of linear independence to determine whether
	$\Set{\vec u,\vec v,\vec w}$ is linearly independent or dependent.

	We need to determine if there is a non-trivial solution to 
	\[
		x\vec u+y\vec v+z\vec w=\vec 0.
	\]
	This vector equation is equivalent to the system of equations
	\[
		\systeme{x+2y+4z=0,2x+3y+5z=0}.
	\]
	Solving this system using row reduction, we see the complete solution set can be expressed as
	\[
		\mat{x\\y\\z}=t\mat{2\\-3\\1}.
	\]
	In particular, $(x,y,z)=(2,-3,1)$ is a non-trivial solution to this system. Therefore $\Set{\vec u,\vec v,\vec w}$
	is linearly dependent.
\end{example}


\begin{theorem}
	The geometric and algebraic definitions of linear independence are equivalent.
\end{theorem}
\begin{proof}
	To show the two definitions are equivalent, we need to show that geometric~$\implies$~algebraic
	and algebraic~$\implies$~geometric.

	\medskip
	\noindent
	(geometric~$\implies$~algebraic) Suppose $\vec v_1,\ldots,\vec v_n$ are linearly dependent by the 
	geometric definition. That means that for some $i$, we have
	\[
		\vec v_i \in \Span\Set{\vec v_1,\ldots,\vec v_{i-1},\vec v_{i+1},\ldots,\vec v_n}.
	\]
	Fix such an $i$. Then, by the definition of span we know
	\[
		\vec v_i=\alpha_1\vec v_1+\cdots \alpha_{i-1}\vec v_{i-1}+\alpha_{i+1}\vec v_{i+1}+\cdots +\alpha_n\vec v_n,
	\]
	and so
	\[
		\vec 0=\alpha_1\vec v_1+\cdots \alpha_{i-1}\vec v_{i-1}-\vec v_i+\alpha_{i+1}\vec v_{i+1}+\cdots +\alpha_n\vec v_n.
	\]
	This must be a non-trivial linear combination because the coefficient of $\vec v_i$ is $-1\neq 0$. Therefore, 
	$\vec v_1,\ldots,\vec v_n$ is linearly dependent by the algebraic definition.
	
	\medskip
	\noindent
	(algebraic~$\implies$~geometric) Suppose $\vec v_1,\ldots,\vec v_n$ are linearly dependent by the 
	algebraic definition. That means there exist $\alpha_1,\ldots,\alpha_n$, not all zero, so that
	\[
		\vec 0=\alpha_1\vec v_1+\cdots +\alpha_n\vec v_n.
	\]
	Fix $i$ so that $\alpha_i\neq 0$ (why do we know there is such an $i$?). Rearranging we get
	\[
		-\alpha_i\vec v_i=\alpha_1\vec v_1+\cdots \alpha_{i-1}\vec v_{i-1}+\alpha_{i+1}\vec v_{i+1}+\cdots +\alpha_n\vec v_n,
	\]
	and since $\alpha_i\neq 0$, we can multiply both sides by $\frac{-1}{\alpha_i}$ to get
	\[
		\vec v_i=\tfrac{-\alpha_1}{\alpha_i}\vec v_1+\cdots +\tfrac{-\alpha_{i-1}}{\alpha_i}\vec v_{i-1}
	+\tfrac{-\alpha_{i+1}}{\alpha_i}\vec v_{i+1}+\cdots +\tfrac{-\alpha_n}{\alpha_i}\vec v_n.
	\]
	This shows that
	\[
		\vec v_i \in \Span\Set{\vec v_1,\ldots,\vec v_{i-1},\vec v_{i+1},\ldots,\vec v_n},
	\]
	and so $\vec v_1,\ldots,\vec v_n$ is linearly dependent by the geometric definition.
\end{proof}

\Heading{Linear Independence and Unique Solutions}

The algebraic definition of linear independence can teach us something about
solutions to systems of equations. 

Recall the linearly dependent vectors 
\[
	\vec u=\mat{1\\0\\0}\qquad \vec v=\mat{0\\1\\0}\qquad \vec w=\mat{1\\1\\0}
\]
which satisfy the non-trivial relationship $\vec u+\vec v-\vec w=\vec 0$. Since $\vec u+\vec v-\vec w=\vec 0$
is a non-trivial relationship giving $\vec 0$, we can use it to generate others. For example,
\begin{alignat*}{2}
	17(\vec u+\vec v-\vec w)\ &=\ 17\vec u+17\vec v-17\vec w\ &=\ 17\vec 0\ &=\ \vec 0\\
	-3(\vec u+\vec v-\vec w)\ &=\ -3\vec u-3\vec v+3\vec w\ &=\ -3\vec 0\ &=\ \vec 0\\
	&\hspace{5em}\vdots&&
\end{alignat*}
are all different non-trivial linear combinations that give $\vec 0$. In other words, if 
the equation $\alpha\vec u+\beta \vec v+\gamma \vec w=\vec 0$ has a non-trivial solution,
it has \emph{infinitely many} non-trivial solutions. Conversely, if  the equation
$\alpha\vec u+\beta \vec v+\gamma \vec w=\vec 0$ has infinitely many solutions, one of them
has to be non-trivial!

Equations where one side is $\vec 0$ show up often and are called \emph{homogeneous} equations.
\SavedDefinitionRender{HomogeneousSystem}

This insight links linear independence and homogeneous systems together, and is encapsulated
in the following theorem.

\begin{theorem}
	The vectors $\vec v_1$, \ldots, $\vec v_n$ are linearly independent if and only if
	the homogeneous equation
	\[
		\alpha_1\vec v_1+\cdots +\alpha_n\vec v_n=\vec 0
	\]
	has a unique solution.
\end{theorem}

This theorem has a practical application: suppose you wanted to decide if the vectors $\vec a$, $\vec b$, and
$\vec c$ were linearly independent. You could (i) find a non-trivial solution to $x\vec a+y\vec b+z\vec c=\vec 0$, 
or (ii) merely show that $x\vec a+y\vec b+z\vec c=\vec 0$ has more than one solution. Sometimes one is easier than
the other.

\Heading{Linear Independence and Vector Form}

The equation
\[
	\vec x=t_1\vec d_1+t_2\vec d_2
\]
represents a plane in vector form whenever $\vec d_1$ and $\vec d_2$ are non-zero, non-parallel vectors.
In other words, 
	$\vec x=t_1\vec d_1+t_2\vec d_2$
	represents a plane whenever $\Set{\vec d_1,\vec d_2}$ is linearly independent.

Does this reasoning work for lines too? The equation
\[
	\vec x=t\vec d
\]
represents a line in vector form precisely when $\vec d\neq \vec 0$. And $\Set{\vec d}$ is linearly independent
exactly when $\vec d\neq 0$.

This reasoning generalizes to volumes. The equation
\[
	\vec x=t_1\vec d_1+t_2\vec d_2+t_3\vec d_3
\]
represents a \emph{volume} in vector form exactly when $\Set{\vec d_1,\vec d_2,\vec d_3}$ is linearly independent.
To see this, suppose $\Set{\vec d_1,\vec d_2,\vec d_3}$ were linearly dependent. That means one or
more vectors could be removed from $ \Set{\vec d_1,\vec d_2,\vec d_3}$ without changing its span.
Therefore, if $\Set{\vec d_1,\vec d_2,\vec d_3}$ is linearly dependent $\vec x=t_1\vec d_1+t_2\vec d_2+t_3\vec d_3$
at best represents a plane (though it could be a line or a point).

We now have a way of testing the validity of a vector-form representation of a line/plane/volume. Just check
whether the chosen direction vectors are linearly independent!
\begin{emphbox}[Takeaway]
	When writing an object in vector form, the direction vectors must always be linearly independent.
\end{emphbox}
