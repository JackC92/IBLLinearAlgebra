\begin{exercises}
	% Topics
	% formal + intuitive definition of subspace
	% relationship between subspace and spans
	% prove a set is a subspace
	% basis + dimension of a subspace
	\begin{problist}

		\prob For each of the following sets, prove whether or not they are a subpace.
		\begin{enumerate}
			\item  $\mathcal T \subseteq \R^2$, where $\mathcal T$ is the complete solution to $3x-y=0$.
			\item  $\mathcal U \subseteq \R^2$, where $\mathcal U$ is the complete solution to $\frac{1}{2}x-6y=0$.
			\item  $\mathcal V \subseteq \R^2$, where $\mathcal V$ is the complete solution to $x-5y-1=0$.
			\item  $\mathcal X \subseteq \R^3$, where $\mathcal X$ is the complete solution to
			$5x-\pi y + (\ln 2)z=0$.
			\item $\mathcal Q \subseteq \R^n$, where $\mathcal Q$ is the complete solution to
			$a_1x_1+a_2x_2+...+a_{n-1}x_{n-1}+a_nx_n=0$ where $a_1,...,a_n \in \R$.
		\end{enumerate}
		\begin{solution}
			\begin{enumerate}
				\item $\mathcal T$ is a subspace: it is non-empty since $\mat{0\\0}\in\mathcal T$. 
				\begin{enumerate}
					\item Let $\mat{x_1\\y_1},\mat{x_2\\y_2}\in \mathcal T$. 
						Then $3x_1-y_1=0=3x_2-y_2$. Hence $3(x_1+x_2)-(y_1+y_2)=(3x_1-y_1)+(3x_2-y_1)=0+0=0$ 
						and $\mat{x_1\\y_1}+\mat{x_2\\y_2}\in\mathcal T$.
					\item Let $\mat{x\\y}\in\mathcal T$. Then $3x-y=0$. Hence for all $\alpha\in\R$, 
						$3\alpha x-\alpha y=\alpha(3x-y)=\alpha\cdot 0=0$ and $\alpha\mat{x\\y}\in\mathcal T$.
				\end{enumerate}

				\item $\mathcal U$ is a subspace: it is non-empty since $\mat{0\\0}\in\mathcal U$. 
				\begin{enumerate}
					\item Let $\mat{x_1\\y_1},\mat{x_2\\y_2}\in \mathcal U$. Then $\frac{1}{2}x_1-6y_1=0=\frac{1}{2}x_2-6y_2$.
						Hence $\frac{1}{2}(x_1+x_2)-6(y_1+y_2)=(\frac{1}{2}x_1-6y_1)+(\frac{1}{2}x_2-6y_2)=0+0=0$
						and $\mat{x_1\\y_1}+\mat{x_2\\y_2}\in\mathcal U$.
					\item Let $\mat{x\\y}\in\mathcal U$. Then $\frac{1}{2}x-6y=0$. Hence for all 
						$\alpha\in\R$, $\frac{1}{2}\alpha x-6\alpha y=\alpha(\frac{1}{2}x-6y)=\alpha\cdot 0=0$ 
						and $\alpha\mat{x\\y}\in\mathcal U$.
				\end{enumerate}

				\item $\mathcal V$ is not a subspace, since for example $\mat{6\\1}\in\mathcal V$, but $0\cdot\mat{6\\1}=\vec 0\not\in\mathcal V$.

				\item $\mathcal X$ is a subspace: it is non-empty since $\mat{0\\0\\0}\in\mathcal X$. 
				\begin{enumerate}
					\item Let $\mat{x_1\\y_1\\z_1},\mat{x_2\\y_2\\z_2}\in \mathcal X$. Then 
						$5x_1-\pi y_1+(\ln 2)z_1=0=5x_2-\pi y_2+(\ln 2)z_2$. Hence
						$5(x_1+x_2)-\pi(y_1+y_2)+(\ln 2)(z_1+z_2)=(5x_1-\pi y_1+(\ln 2)z_1)+(5x_2-\pi y_2+(\ln 2)z_2)=0+0=0$ and $\mat{x_1\\y_1\\z_1}+\mat{x_2\\y_2\\z_2}\in\mathcal X$.

					\item Let $\mat{x\\y\\z}\in\mathcal X$. Then $5x-\pi y+(\ln 2)z=0$. Hence for all
						$\alpha\in\R$, $5\alpha x-\pi\alpha y+(\ln 2)\alpha z=\alpha(5x-\pi y+(\ln 2)z)=\alpha\cdot 0=0$ 
						and $\alpha\mat{x\\y\\z}\in\mathcal X$.
				\end{enumerate}

				\item $\mathcal Q$ is a subspace: it is non-empty since $\vec 0\in\mathcal Q$. 
				\begin{enumerate}
					\item Let $\mat{x_1\\\vdots\\x_n},\mat{x'_1\\\vdots\\x'_n}\in \mathcal Q$. Then 
						$a_1x_1+\ldots+a_nx_n=0=a_1x'_1+\ldots+a_nx'_n$. Hence 
						$a_1(x_1+x'_1)+\ldots+a_n(x_n+x'_n)=(a_1x_1+\ldots+a_nx_n)+(a_1x'_1+\ldots+a_nx'_n)=0+0=0$ 
						and $\mat{x_1\\\vdots\\x_n}+\mat{x'_1\\\vdots\\x'_n}\in\mathcal Q$.

					\item Let $\mat{x_1\\\vdots\\x_n}\in\mathcal Q$. Then $a_1 x_1+\ldots+a_nx_n=0$. Hence for any scalar 
						$\alpha\in\R$, $a_1\alpha x_1+\ldots+a_n\alpha x_n=\alpha(a_1x_1+\ldots+a_nx_n)=\alpha\cdot 0=0$ 
						and $\alpha\mat{x_1\\\vdots\\x_n}\in\mathcal Q$.
				\end{enumerate}
			\end{enumerate}
		\end{solution}

		\prob For each of the following sets, prove whether or not it is a subpace.
		\begin{enumerate}
			\item $\mathcal A \subseteq \R^2$, where $\mathcal A$ is the  line specified in vector form by
			$\vec x = t\mat{5\\-7}+\mat{1\\2}$.
			\item $\mathcal B \subseteq \R^2$, where $\mathcal B$ is  the line specified  in vector form by
			$\vec x = t\mat{-3\\4}$.
			\item $\mathcal C \subseteq \R^3$, where $\mathcal C$ is  the line specified  in vector form by
			$\vec x = t\mat{1\\0\\5}$.
			\item $\mathcal D \subseteq \R^3$, where $\mathcal D$ is  the line specified  in vector form by
			$\vec x = t\mat{2\\3\\4}+s\mat{10\\20\\131}+\mat{0\\0\\6}$.
			\item $\mathcal E \subseteq \R^3$, where $\mathcal E$ is  the line specified  in vector form by
			$\vec x = t\mat{5\\7\\1}+s\mat{2\\-2\\1}$.
		\end{enumerate}
		\begin{solution}
			\begin{enumerate}
				\item $\mathcal A$ is not a subspace, since for example $\mat{1\\2}\in\mathcal A$, 
					but $0\cdot\mat{1\\2}=\vec 0\not\in\mathcal A$.

				\item $\mathcal B$ is a subspace: it is non-empty since $\mat{-3\\4}\in\mathcal B$. 
				\begin{enumerate}
					\item Let $\vec u,\vec v\in \mathcal B$. Then $\vec u=t_1\mat{-3\\4}$ and $\vec v=t_2\mat{-3\\4}$ 
						for some $t_1,t_2\in\R$. But then $\vec u+\vec v=(t_1+t_2)\mat{-3\\4}\in \mathcal B$.
					\item Let $\vec u=t\mat{-3\\4}\in\mathcal B$. For any scalar $\alpha\in\R$, we have
						$\alpha\vec u=(\alpha t)\mat{-3\\4}\in\mathcal B$.
				\end{enumerate}

				\item $\mathcal C$ is a subspace: it is non-empty since $\mat{1\\0\\5}\in\mathcal C$. 
				\begin{enumerate}
					\item Let $\vec u,\vec v\in \mathcal C$. Then $\vec u=t_1\mat{1\\0\\5}$ and $\vec v=t_2\mat{1\\0\\5}$ 
						for some $t_1,t_2\in\R$. But then $\vec u+\vec v=(t_1+t_2)\mat{1\\0\\5}\in \mathcal C$.

					\item Let $\vec u=t\mat{1\\0\\5}\in\mathcal C$. For any scalar $\alpha\in\R$, we have 
						$\alpha\vec u=(\alpha t)\mat{1\\0\\5}\in\mathcal C$.
				\end{enumerate}

				\item $\mathcal D$ is not a subspace, since for example $\mat{0\\0\\6}\in\mathcal D$, but 
					$0\cdot\mat{0\\0\\6}=\vec 0\not\in\mathcal D$. ($\mat{0\\0\\-6}$ cannot be written 
					as a linear combination of $\mat{2\\3\\4}$ and $\mat{10\\20\\131}$.)

				\item $\mathcal E$ is a subspace: it is non-empty since $\mat{0\\0\\0}\in\mathcal E$. 
				\begin{enumerate}
					\item Let $\vec u,\vec v\in \mathcal E$. Then $\vec u=t_1\mat{5\\7\\1}+s_1\mat{2\\-2\\1}$ 
						and $\vec v=t_2\mat{5\\7\\1}+s_2\mat{2\\-2\\1}$ for some $t_1,t_2,s_1,s_2\in\R$. 
						But then $\vec u+\vec v=(t_1+t_2)\mat{5\\7\\1}+(s_1+s_2)\mat{2\\-2\\1}\in \mathcal E$.
					\item Let $\vec u=t\mat{5\\7\\1}+s\mat{2\\-2\\1}\in\mathcal E$. For any scalar $\alpha\in\R$,
						we have $\alpha\vec u=(\alpha t)\mat{5\\7\\1}+(\alpha s)\mat{2\\-2\\1}\in\mathcal E$.
				\end{enumerate}
			\end{enumerate}

		\end{solution}		

		\prob Use the definition of subspace to prove each span below is a subspace.
		\begin{enumerate}
			\item $\Span\Set*{\mat{0\\1}, \mat{1\\2}}$
			\item $\Span\Set*{\mat{1\\1\\1}, \mat{1\\0\\0}, \mat{2\\0\\0}}$
		\end{enumerate}
		\begin{solution}
			\begin{enumerate}
				\item Let $\mathcal A$ be $\Span\Set*{\mat{0\\1},\mat{1\\2}}$. $\mathcal A$ is a subspace: 
					it is non-empty since $\vec 0\in\mathcal A$. 
				\begin{enumerate}
					\item Let $\vec u,\vec v\in \mathcal A$. Then $\vec u=t_1\mat{0\\1}+s_1\mat{1\\2}$ and 
						$\vec v=t_2\mat{0\\1}+s_2\mat{1\\2}$ for some $t_1,t_2,s_1,s_2\in\R$. But 
						then $\vec u+\vec v=(t_1+t_2)\mat{0\\1}+(s_1+s_2)\mat{1\\2}\in \mathcal A$.
					\item Let $\vec u=t\mat{0\\1}+s\mat{1\\2}\in\mathcal A$. For any scalar $\alpha\in\R$,
						we have $\alpha\vec u=(\alpha t)\mat{0\\1}+(\alpha s)\mat{1\\2}\in\mathcal A$.
				\end{enumerate}

				\item Let $\mathcal B$ be $\Span\Set*{\mat{1\\1\\1},\mat{1\\0\\0},\mat{2\\0\\0}}$. Observe 
					that since $\mat{2\\0\\0}=2\mat{1\\0\\0}$, this is the same thing as saying 
					$\mathcal B=\Span\Set*{\mat{1\\1\\1},\mat{1\\0\\0}}$.

					$\mathcal B$ is a subspace: it is non-empty since $\vec 0\in\mathcal B$. 
				\begin{enumerate}
					\item Let $\vec u,\vec v\in \mathcal B$. Then $\vec u=t_1\mat{1\\1\\1}+s_1\mat{1\\0\\0}$ 
						and $\vec v=t_2\mat{1\\1\\1}+s_2\mat{1\\0\\0}$ for some $t_1,t_2,s_1,s_2\in\R$. 
						But then $\vec u+\vec v=(t_1+t_2)\mat{1\\1\\1}+(s_1+s_2)\mat{1\\0\\0}\in \mathcal B$.
					\item Let $\vec u=t\mat{1\\1\\1}+s\mat{1\\0\\0}\in\mathcal B$. For any scalar $\alpha\in\R$, 
						we have $\alpha\vec u=(\alpha t)\mat{1\\1\\1}+(\alpha s)\mat{1\\0\\0}\in\mathcal B$.
				\end{enumerate}
			\end{enumerate}
		\end{solution}
		
		\prob
		A non-empty subset $\mathcal V \subseteq \R^n$ is called a subspace if
		for all $\vec u, \vec v \in \mathcal V$ and all scalars $k$ we have
			(i) $\vec u + \vec v \in \mathcal V$ and
			(ii) $k\vec u \in \mathcal V$.
			For each set below, list which of property (i), property (ii), or non-emptiness fails.
			Justify your answer.
		\begin{enumerate}
			\item $\Set*{(x,y,z)\given x+y+z=4}$
			\item $\Set*{}$
			\item $\Set*{(x,y)\given x=y^2}$
			\item $\Set*{(x_1,x_2)\given x_1 \geq 0}$
			\item $\Set*{(x,y)\given x^2+y^2=0}$
		\end{enumerate}
		\begin{solution}
			\begin{enumerate}
				\item Property (i) fails, since $\mat{4\\0\\0}$ is in the set, but $\mat{4\\0\\0}+\mat{4\\0\\0}=\mat{8\\0\\0}$ is not.

				Property (ii) fails, since $\mat{4\\0\\0}$ is in the set, but $0\cdot \mat{4\\0\\0}=\vec 0$ is not.

				\item Non-emptiness fails, since it is the empty set.

				\item Property (i) fails, since $\mat{1\\1}$ is in the set ($1=1^2$), but $\mat{1\\1}+\mat{1\\1}=\mat{2\\2}$ is not ($2\neq2^2$).

					Property (ii) fails, since $\mat{1\\1}$ is in the set, but $2\mat{1\\1}=\mat{2\\2}$ is not.

				\item Property (ii) fails, since $\mat{1\\0}$ is in the set, but $-\mat{1\\0}=\mat{-1\\0}$ is not.

				\item None of the properties fail. The set is $\Set{\vec 0}$, the trivial subspace.
			\end{enumerate}
		\end{solution}

		\prob For each subset below, determine whether or not it is a subspace.
		If it is a subspace, find (i) its dimension and (ii) a basis for it.
		\begin{enumerate}
			\item $\Span\Set*{\mat{2\\3}, \mat{-4\\-6}, \matc{1\\3/2}}$
			\item $\Span\Set*{\mat{1\\0\\-2}, \mat{0\\2\\5}, \mat{1\\2\\3}}$
			\item The plane given in vector form by
			\[
				\vec x = t\mat{6\\1\\1}+s\mat{6\\0\\6}
			\]
			\item The line given in vector form by
			\[
				\vec x = t\mat{2\\2\\3}
			\]
			\item The complete solution to
			\[
				\mat{1\\-2\\3} \cdot \left(\mat{x\\y\\z}-\mat{2\\2\\\frac{2}{3}}\right)=0
			\]
			\item The complete solution to
			\[
				\mat{1\\3\\3\\7} \cdot \left(\mat{x\\y\\z\\w}-\mat{0\\0\\0\\0}\right)=0
			\]
		\end{enumerate}
		\begin{solution}
			\begin{enumerate}
				\item Yes, it is a subspace. Its dimension is $1$ and a basis for it is $\Set*{\mat{2\\3}}$.

				\item Yes, it is a subspace. Its dimension is $2$ and a basis for it is $\Set*{\mat{1\\0\\-2},\mat{0\\2\\5}}$.

				\item Yes, it is a subspace. Its dimension is $2$ and a basis for it is $\Set*{\mat{6\\1\\1},\mat{6\\0\\6}}$.

				\item Yes, it is a subspace. Its dimension is $1$ and a basis for it is $\Set*{\mat{2\\2\\3}}$.

				\item Yes, it is a subspace. Its dimension is $2$ and a basis for it is $\Set*{\mat{6\\6\\2},\mat{2\\1\\0}}$.

				\item Yes, it is a subspace. Its dimension is $3$ and a basis for it is $\Set*{\mat{3\\-1\\0\\0},\mat{3\\0\\-1\\0},\mat{7\\0\\0\\-1}}$.
			\end{enumerate}
		\end{solution}

		\prob Which of the following are bases for $\R^3$?
		\begin{enumerate}
			\item $\Set*{\mat{2\\6\\1}, \mat{4\\2\\1}, \mat{6\\8\\2}}$
			\item $\Set*{\mat{1\\0\\1}, \mat{1\\0\\0}, \mat{0\\-1\\0}, \mat{-2\\1\\1}}$
			\item $\Set*{\mat{2\\3\\5}, \mat{5\\-4\\2}}$
			\item $\Set*{\mat{2\\5\\-6}, \mat{4\\11\\-12}, \mat{0\\0\\-3}}$
		\end{enumerate}
		\begin{solution}
			\begin{enumerate}
				\item Not a basis, as it is not linearly independent: the third vector is equal to the sum of the other two.

				\item Not a basis, since four vectors in $\R^3$ cannot be linearly independent.

				\item Not a basis, as two vectors cannot span all of $\R^3$. You need at least three vectors.

				\item It is a basis.
			\end{enumerate}
		\end{solution}


		\prob  For each statement, determine if it is true or false.
		Justify your answer by referring to a definition or a theorem.
		\begin{enumerate}
			\item All spans are subspaces.
			\item All subspaces can be expressed as spans.
			\item All translated spans are subspaces.
			\item The empty set is a subspace.
			\item The set $\Set*{\mat{1\\2}, \mat{2\\3}}$ is a subspace.
		\end{enumerate}
		\begin{solution}
			\begin{enumerate}
				\item True by the Subspace-Span Theorem.

				\item True by the Subspace-Span Theorem.

				\item False. Translated spans cannot all be expressed as spans (since they do not need to contain $\vec 0$, but spans do), but the Subspace-Span Theorem says that all subspaces can be expressed as spans.

				\item False. By the definition of a subspace, they cannot be the empty set.

				\item False. By the definition of a subspace, since $\mat{1\\2}$ and $\mat{2\\3}$ are in the set, $\mat{1\\2}+\mat{2\\3}=\mat{3\\5}$ should also be in the set. But it isn't.
\end{enumerate}
		\end{solution}

		\prob Give two examples of subspaces of $\R^4$ that are (i) 1 dimensional, (ii)
		3 dimensional. Can you give an example of a subspace that is 0 dimensional? 
		\label{PROB-GIVE-EXAMPLES}
		\begin{solution}
			\begin{enumerate}[label=(\roman*)]
				\item $\Span\Set*{\mat{1\\2\\3\\4}}$ and the line with vector form $\vec x=t\mat{2\\2\\3\\0}$.

				\item $\Span\Set*{\mat{1\\0\\0\\0},\mat{0\\1\\0\\0},\mat{0\\0\\1\\0}}$ and the volume with equation $x_1+x_2+x_3+x_4=0$.
			\end{enumerate}

			The set $\Set{\vec 0}$ is the only 0-dimensional subspace.
		\end{solution}

		\prob Let $\mathcal G\subseteq \R^n$ be a subspace.
		Give upper and lower bounds for the dimension of $\mathcal G$.
		\begin{solution}
			As we've seen in problem \ref{PROB-GIVE-EXAMPLES}, the dimension of $\mathcal G$ can be as small as $0$. This is our lower bound.

			As for the upper bound, we know that $n+1$ vectors in $\R^n$ cannot be linearly independent. So a basis for $\mathcal G$ can have at most $n$ vectors. Hence $n$ is an upper bound for the dimension of $\mathcal G$.
		\end{solution}
	\end{problist}
\end{exercises}
