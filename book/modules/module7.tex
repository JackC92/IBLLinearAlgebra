Matrix-vector multiplication gives a compact way to represent systems of linear equations.

Consider the system
\begin{equation}
	\label{EQREGSYS}
	\sysdelim\{.
		\systeme{
			x+2y-2z=-15,
			2x+y-5z=-21,
			x-4y+z=18
		},
\end{equation}
which is equivalent to the vector equation
\[
	\mat{x+2y-2z\\
           2x+y-5z\\
	     x-4y+z}=\mat{-15\\-21\\18}.
\]
We can rewrite \eqref{EQREGSYS} using matrix-vector multiplication:
\[
	\underbrace{\mat{1&2&-2\\2&1&-5\\1&-4&1}}_{A}\mat{x\\y\\z}=\mat{-15\\-21\\18}.
\]
The matrix $A$ on the left is called the \emph{coefficient matrix}\index{Matrix!coefficient matrix}\index[definitions]{Matrix!coefficient matrix}\index{System of linear equations!coefficient matrix}\index[definitions]{System of linear equations!coefficient matrix} because it
is made up of the coefficients from equation \eqref{EQREGSYS}.

\medskip
By using coefficient matrices, every system of linear equations can be rewritten as a single matrix equation
of the form
\[
	A\vec x=\vec b
\]
where $A$ is a coefficient matrix, $\vec x$ is a column vector of variables, and $\vec b$
is a column vector of constants.

\begin{example}
	Consider the one equation system
	\begin{equation}
		\label{EQSYS1}
		\systeme{x-4y+z=5}
	\end{equation}
	and the two-equation system
	\begin{equation}
		\label{EQSYS2}
		\systeme{x-4y+z=5, y-z=9}.
	\end{equation}
	Rewrite each system as a single matrix equation.

	We can rewrite \eqref{EQSYS1} as
	\[
        \mat{1&-4&1}\mat{x\\y\\z}=\mat{5}.
    \]
    Multiplying out to verify, we see,
    \[
        \mat{1&-4&1}\mat{x\\y\\z}=\mat{x-4y+z}=\mat{5},
    \]
    which is indeed equivalent to \eqref{EQSYS1}.

	Similarly, we can rewrite \eqref{EQSYS2} as
	\[
       \mat{1&-4&1\\0&1&-1}\mat{x\\y\\z}=\mat{5\\9}.
    \]
    Multiplying out to verify, we see,
    \[
       \mat{1&-4&1\\0&1&-1}\mat{x\\y\\z}=\mat{x-4y+z\\0x+y-z}=\mat{5\\9},
    \]
    which is equivalent to \eqref{EQSYS2}.
\end{example}

\Heading{Interpretations of Matrix Equations}

The solution set to a system of linear equations, like
\begin{equation}
	\label{EQREGSYSb}
	\sysdelim\{.
		\systeme{
			x+2y-2z=-15,
			2x+y-5z=-21,
			x-4y+z=18
		},
\end{equation}
can be interpreted as the intersection of three planes (or hyperplanes if there were more variables).
Each equation (each row) specifies a plane, and the solution set is the intersection of all of these planes.
Rewriting a system of equations in matrix form gives two additional ways to interpret  the solution set.

\Heading{The \emph{Column} Picture}\index{Matrix equations!the column picture}

Using the column interpretation of matrix-vector multiplication, we see that system \eqref{EQREGSYSb} is equivalent to
\[
	\mat{1&2&-2\\2&1&-5\\1&-4&1}\mat{x\\y\\z}=
	     x\mat{1\\2\\1}+y\mat{2\\1\\-4}+z\mat{-2\\-5\\1}
	     =\mat{-15\\-21\\18}.
\]
We now see that asking, ``\emph{What coefficients
allow $\mat{1\\2\\1}$, $\mat{2\\1\\-4}$, and $\mat{-2\\-5\\1}$ to form $\mat{-15\\-21\\18}$ as a linear combination?}'' is
equivalent to asking, ``\emph{What are the solutions to system \eqref{EQREGSYSb}?}''
Here,
 $\mat{1\\2\\1}$, $\mat{2\\1\\-4}$, and $\mat{-2\\-5\\1}$ are the columns of the coefficient matrix.


\Heading{The \emph{Row} Picture}\index{Matrix equations!the row picture}
The row interpretation gives another perspective. Let $\vec r_1$, $\vec r_2$, and $\vec r_3$ be the rows of the coefficient matrix
for system \eqref{EQREGSYSb}. Then, system \eqref{EQREGSYSb} is equivalent to
\[
	\mat{1&2&-2\\2&1&-5\\1&-4&1}\mat{x\\y\\z}=
	\left[\begin{array}{c}\vec r_1\\\hline\vec r_2\\\hline\vec r_3\end{array}\right]\vec x=
	\mat{\vec r_1\cdot \vec x\\\vec r_2\cdot\vec x\\\vec r_3\cdot \vec x}
	     =\mat{-15\\-21\\18}.
\]
In other words, we can interpret solutions to system \eqref{EQREGSYSb} as vectors whose dot product
with $\vec r_1$ is $-15$, whose dot product with $\vec r_2$ is $-21$, and whose dot product with
$\vec r_3$ is $18$. Given that the dot product has a geometric interpretation, this perspective is powerful
(especially when the right side of the equation is all zeros!).

\Heading{Interpreting Homogeneous Systems}
Consider the homogeneous system/matrix equation
\begin{equation}
	\label{EQREGSYSc}
	\sysdelim..
		\systeme{
			x+2y-2z=0,
			2x+y-5z=0,
			x-4y+z=0
		}
		\qquad\iff\qquad
		\underbrace{\mat{1&2&-2\\2&1&-5\\1&-4&1}}_A\mat{x\\y\\z}=\mat{0\\0\\0}
		.
\end{equation}
Now, the column interpretation of system \eqref{EQREGSYSc} is: \emph{what linear combinations of the column vectors
of $A$ give $\vec 0$?} This directly relates to the question of whether the column vectors of $A$ are linearly independent.

Let $\vec r_1$, $\vec r_2$, and $\vec r_3$ be the rows of $A$. The row interpretation of system \eqref{EQREGSYSc} asks:
\emph{what vectors are simultaneously orthogonal to $\vec r_1$, $\vec r_2$, and $\vec r_3$?}

\begin{emphbox}[Takeaway]
	There are three ways to interpret solutions to a matrix equation $A\vec x=\vec b$: (i) the intersection of hyperplanes
	specified by the rows; (ii) what linear combinations of the columns of $A$ give $\vec b$; (iii) what vectors yield
	the entries of $\vec b$ when dot producted with the rows of $A$.
\end{emphbox}

\begin{example}
	Find all vectors orthogonal to $\vec a=\mat{1\\1\\1}$ and $\vec b=\mat{1\\2\\1}$.

	To find all vectors orthogonal to $\vec a$ and $\vec b$ we need to find vectors $\vec x$
	satisfying $\vec a\cdot \vec x=0$ and $\vec b\cdot \vec x=0$. This is equivalent to solving the matrix equation
	\[
		\left[\begin{array}{c}\vec a\\\hline\\[-1.5\medskipamount]\vec b\end{array}\right]\vec x
	=\mat{\vec a\cdot \vec x\\\vec b\cdot\vec x}
	=\underbrace{\mat{1&1&1\\1&2&1}}_A\mat{x\\y\\z}
	=\mat{0\\0}.
    \]
    By row reducing $A$, we get
    \[
        \Rref(A) = \mat{1&0&1\\0&1&0},
    \]
    and so the complete solution expressed in vector form is
    \[
        \vec x = t\mat{-1\\0\\1}.
    \]
\end{example}

The row picture is particularly applicable when trying to find normal vectors.

\begin{example}
	Let $\mathcal Q$ be the hyperplane specified in vector form by
	\[
		\vec x=t\mat{1\\1\\-1\\1}+s\mat{0\\1\\0\\1}+r\mat{2\\0\\0\\0}+\mat{1\\2\\3\\4}.
	\]
	Find a normal vector for $\mathcal Q$ and write $\mathcal Q$ in normal form.

	Like the above example, since normal vectors for $\mathcal Q$ need to be orthogonal to $\vec d_1=\mat{1\\1\\-1\\1}$, $\vec d_2=\mat{0\\1\\0\\1}$, and $\vec d_3=\mat{2\\0\\0\\0}$, we can find the normal vectors by solving
	\[
	\underbrace{\mat{1&1&-1&1\\0&1&0&1\\2&0&0&0}}_A\mat{x\\y\\z\\w}=\mat{0\\0\\0}.
    \]
    By row reducing $A$, we get
    \[
        \Rref(A) = \mat{1&0&0&0\\0&1&0&1\\0&0&1&0},
    \]
    and so we get that the complete solution expressed in vector form is
    \[
        \vec x = t\mat{0\\-1\\0\\1}.
	\]
    Therefore, any non-zero multiple of $\mat{0\\-1\\0\\1}$ is a normal vector for $\mathcal Q$. For example, $\vec n =\mat{0\\-1\\0\\1}$ is a normal vector for $\mathcal Q$, and $\mathcal Q$ can be written in normal form as
    \[
	    \mat{0\\-1\\0\\1}\cdot\left(\mat{x\\y\\z\\w}-\mat{1\\2\\3\\4}\right)=0.
    \]
\end{example}



