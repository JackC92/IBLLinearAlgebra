		\subsection*{Learning Objectives} Students need to be able to\ldots
		\begin{itemize}
			\item Turn geometric descriptions and pictures into equations/formulas/sets
				suitable for manipulation with mathematics.

			\item Be comfortable enough with set notation and operations to combine
				the operations in new ways.
		\end{itemize}


		\subsection*{Context} Students in class have gone over sets, set operations,
		vectors, linear combinations, vector form of lines and planes, and have just started
		span. Some sections may also have covered \emph{restricted} linear combinations, for example
		convex combinations. Sections have \emph{not} covered norm notation (i.e., $\|\vec x\|$) or
		lengths of vectors in general. However, they all know the Pythagorean theorem from high school.


		\subsection*{What to Do} This is the first tutorial of the term, and
		it is your chance to win the students over! This is a groupwork tutorial,
		but students may not be used to working in groups.

		\begin{itemize}
			\item Arranged for group work. Reorganize the desks and chairs
				(if possible) to facilitate groups of 3 or 4. Ask
				students to form groups of 3 or 4 with other students
				nearby. Don't allow larger groups.

			\item Begin the tutorial by introducing yourself (your name,
				your program of study, and your year). You might
				also want to give them some more personal information,
				such as where you are from or when you first started liking math.

			\item Introduce the structure and purpose of tutorials: students
				will be working to (1) better understand concepts
				from lecture, (2) practice tackling concepts that
				have not been explained in lecture, and (3) effectively
				communicate. They can expect to spend most of the
				tutorial working in small groups.

			\item Emphasize the importance of working with others when
				learning mathematics---they should be working with
				others in this tutorial \emph{and} outside of
				class.
		\end{itemize}

		This introduction should take no more than 5 minutes.

		Next, introduce the learning objectives for the day's tutorial. Explain
		that the goal of this tutorial. Their worksheet has the ``formal'' objectives
		stated and these instructions have the ``hidden'' objectives. Feel free
		to share with them the hidden objectives as well.

		Ask the students to pair up and
		start working on the problem list. Circulate around the room during
		this time and ask groups what they're thinking. They will be tempted
		to move quickly through the list without thoroughly checking their
		new answers---encourage them to think deeply.

		Problem 1 is a straightforward question, but students will struggle starting
		with part (d) and especially with (e) and (f). They may have forgotten about unions! Ask
		them to review the set operations that they know and be creative. When most people are on
		parts (e) and (f), go over parts (a)--(d). Then, let them continue working through number 2.
		If most of the class gets stuck at any point, draw the class's attention to the front
		of the room and work on the difficult part together.

		There are too many problems to finish in 50 minutes and \emph{you should not be going
		over the solution to every problem}. Solutions will be posted for the students. The goal
		of tutorial is for students to spend time \emph{doing} mathematics with an expert around
		to help them if they get stuck. Don't feel any time pressure, even if you only get through 1.5
		questions, that's okay!

		During the last 6 minutes of class, pick one problem (perhaps a few parts of one problem)
		that most groups have at least started, and do this problem as a wrapup. Seeing an expert do the
		problem is the student's reward for working so hard.

		Notes:
		\begin{itemize}
			\item Students won't have a good conceptualization of convex combinations which make 1(f) and 3(c).
				These problems can also be done by describing a line in vector form (i.e., $\vec x=t\vec d+\vec p$)
				and restricting the scalar to get points on the line segment.
			\item For 1(e), some students might write $\{\vec x:\vec x\text{ is a convex linear combination
				of }\vec p\text{ and }\vec q\}$. Other students might think that this description
				is ``mathy'' enough. This description is mathy enough, but we can also expand it
				by inserting the definition of \emph{convex linear combination} into the set.
			\item Problem 2 is more open-ended than they're used to. Some will get excited about this, and others will
				be turned off because it's not a ``plug and chug'' question. Emphasize to them that
				they will be hired for their creativity and problem-solving, and not their ability to
				answer precisely laid out questions, and this is what we're practicing!
			\item Students will be confused what part 2(d) means. If so, take some time at the front of the room
				to make a chalk drawing of a face and a reverse face.
				Remember, if you're writing on a chalkboard, black and white are already reversed!
		\end{itemize}