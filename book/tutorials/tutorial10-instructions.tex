\subsection*{Learning Objectives}
	Students need to be able to\ldots
	\begin{itemize}
		\item Define eigenvector and eigenvalue
		\item Have an intuitive geometric understanding of eigenvectors
		\item Use the definition of eigenvector to classify vectors as eigenvectors or not
		\item Apply eigenvectors to solve problems
	\end{itemize}

\subsection*{Context}
	Students have been introduced to determinants, characteristic polynomials, and the definition
		of eigenvectors/values in lecture. Some students may have started diagonalization in
		lecture.

	In this course we will only pursue real eigenvalues/vectors. Some students may know complex
		numbers from high school, but most will not.


\subsection*{What to Do}
	Start the tutorial by stating the day's learning objectives. Emphasize that eigenvectors/values
		are the capstone of this course and they take a while to get used to. Today we are getting used
		to them.

		Have students start in groups on \#1. They should be used to the style of part (a) by now,
		but they will probably struggle with part (b). Try to get them to produce something better than
		reading the mathematical definition aloud.

		\#2 is a computational question that follows straight from the definition. If you like,
		divide the class into three teams.
		One team will check for $A$, another for $B$, and the last for $C$. Write up (or have each
		team write up) a summary of their results. Then have everyone verify each team's result.

		Before finishing \#2, make sure to have a discussion about $\vec v_6$. Is it an eigenvector?
		Does it go to a multiple of itself?

		It is likely that \#1 and \#2 will take most of class time. If you do have extra time, continue as usual,
		letting students work in small groups on \#3 and then having a mini-discussion when half the groups
		have figured it out.

		6 minutes before the end of class, pick a suitable problem to do as a wrap-up.


\subsection*{Notes}
	\begin{itemize}
		\item For \#1(b), many students will write ``$\vec v$ is an eigenvector for $T$ if
			$\vec v$ is non-zero and $T$ of $\vec v$ is $\lambda$ times $\vec v$ for some scalar
			$\lambda$.'' This is reading the mathematical definition aloud and is not what we want.
			We want students to rephrase $T\vec v=\lambda \vec v$ in terms of a vector not changing
			direction.
		\item For \#2, some students will get confused between $\vec v\neq \vec 0$ and $\lambda \neq 0$.
			$C,\vec v_4$ has eigenvalue $0$ precisely so this confusion will come up. Make sure to
			have a conversation about this whether or not it comes up naturally.
		\item Encourage students to draw pictures for \#3. They will naturally try to start by writing
			down a matrix for $\mathcal P$ and $\mathcal R$, but that short-circuits thinking.
		\item
		When talking about eigenvalues of a transformation $T$
		not existing, always make sure to say ``$T$ has no real eigenvalues''. Students who
		know about complex numbers can have a deeper conversation with you about them after class.
		\item In \#4, some might find it so ``obvious'' that they don't know how to prove it. That's largely
			the point. They should be able to prove obvious things!
		\item \#5 culminates in diagonalization, but it asks students to make a judgement about whether
			a vector is ``approximated''. The fact that $\|\vec v_1\|,\|\vec v_2\|\leq 100$ is essential
			for this, but isn't the point of the problem. The point is that there is a unique, largest, positive
			eigenvalue.

	\end{itemize}