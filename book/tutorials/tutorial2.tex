\Heading{Problems}


\begin{enumerate}
	\item Write down a mathematically-precise definition of ``$\Span\{\vec u,\vec v,\vec w\}$''.
	\item Let $\vec a=\mat{1\\1\\0}$ and $\vec b=\mat{1\\0\\-1}$. Let $S=\Span\{\vec a,\vec b\}$.

		For each of the following, use the definition of span to justify whether they are in $S$.
		\[
			\vec x=\mat{2\\1\\-1}\qquad\vec y=\mat{2\\1\\0}\qquad\vec z=\mat{1\\1}\qquad\vec 0=\mat{0\\0\\0}
		\]

	\item Let $\mathcal P\subseteq \R^3$ be the plane with equation $x+y+z=0$.
		\begin{enumerate}
			\item Find a linearly independent set that spans $\mathcal P$.
			\item Find a linearly dependent set that spans $\mathcal P$.
		\end{enumerate}
	\item Correct the statement: \emph{The span of two vectors $\vec p,\vec q\in\R^3$ is a plane through $\vec 0$
		containing both $\vec p$ and $\vec q$.}
	\item In economics vectors are used to describe consumer preferences. Often times, it is assumed
		that these preference vectors cannot be multiplied by negative scalars. Thus, for an economist,
		``the set of all linear combinations'' might look different than the span.

		Define the \emph{positive span} of the vectors $\vec u$, $\vec v$, and $\vec w$ as the set of all
		linear combinations of $\vec u$, $\vec v$, and $\vec w$ with non-negative coefficients.

		\begin{enumerate}
			\item Write down the positive span of $\{\vec u,\vec v,\vec w\}$ using set-builder notation.
			\item Describe the positive span of $\mat{1\\0}$ and $\mat{0\\1}$. How is it different than
				the span of $\mat{1\\0}$ and $\mat{0\\1}$?
			\item Could the positive span of a linearly independent set ever be all of $\R^2$? What about the
				positive span of a linearly dependent set?
		\end{enumerate}
	
\end{enumerate}