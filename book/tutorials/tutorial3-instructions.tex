\Heading{Learning Objectives}
	Students need to be able to\ldots
	\begin{itemize}
		\item Apply the definition of orthogonality
		\item Interpret and apply a new definition, that of \emph{orthogonal sets}
		\item Set up a system of equations to produce orthogonal vectors
	\end{itemize}

\Heading{Context}
	Students will have covered dot products, orthogonality, normal form of lines and planes, and
		projections in the previous week's lectures\footnote{ Note that if you use the word \emph{projection},
		it has a specific definition in this class: $\Proj_{X}\vec a$ is the closest point in $X$
		to $\vec a$. In particular, $\Proj_{\vec b}\vec a$ is not a defined notation, since $\vec b$ is
		not a set. Instead we use the notation $\Comp_{\vec b}\vec a$.}.

\Heading{What to Do}
	Introduce the learning objectives for the day's tutorial. Explain that we will be extending
		the notion of orthogonality they know from lecture to sets. Further, explain that one of the skills
		we're developing in this class is to be able to read, understand, and apply a new definition, and that's
		what the first tutorial question is all about.

	
	After most groups have finished \#1, go over it as a class. There is no need to write a formal
		proof for part (d), though everyone should be able to give a convincing argument, even
		if it's not a full ``proof''. Students might be puzzled on how to explain (b),
		resorting to ``because they are\ldots''. Try to push them to refer to the definitions
		from which they can give a full argument (even if the argument seems silly).

	Continue as usual, walking around the room and asking
		questions while letting students work on the next problem and gathering them together
		for discussion when most groups have finished.

		7 minutes before class ends, pick a suitable problem to do as a wrap-up.



	
\Heading{Notes}
	\begin{itemize}
		\item Students will have a hard time explaining themselves for 1(d). Make sure you
			have some prompts in your back pocket.
		\item Students might get stuck on \#2. Remind them to think about the definitions and
			to start by writing them down---it's amazing how impossible a problem seems before
			you write down the definition\ldots.
		\item For \#3 we want an answer written as a proof. This will be hard for most students.
		\item \#4 is there as a challenge question. You probably won't be talking about it.

	\end{itemize}