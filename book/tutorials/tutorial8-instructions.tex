\subsection*{Learning Objectives}
	Students need to be able to\ldots
	\begin{itemize}
		\item Define rank (algebraically for matrices and geometrically for linear transformations)
		\item Produce examples of linear transformations
		\item Describe connection between matrix multiplication and composition of functions.
		\item Compute the image of a set of vectors under a transformation.
	\end{itemize}

\subsection*{Context}


\subsection*{What to Do}
  	\begin{itemize}
		\item Give the students a couple of minutes to think about
		problem 1. Then invite four of them up to the board. Have
		two define ``rank''
			of a matrix and two define ``rank'' of a linear
			transformation. Discuss any issues that their
			definitions may have. Make sure to end this
			portion with correct, clear definitions on
			the board. And make sure that the students
			understand them.

		\item Next, arrange the students in groups and give them
		10-12 minutes to work on problem 2 and 4. Be sure to
		walk around the room and try to get the students engaged
		and working. Once time is up, spend a few minutes at
		the board addressing any glaring issues you might've
		noticed while you were circling the room.

		\item Repeat the same process with either problem 3 or
		5. Ask the students to see if they prefer one problem
		over the other. Wrap-up the tutorial by discussing any
		issues you noticed while circling the room.
	\end{itemize}

\subsection*{Notes}
  	\begin{itemize}
		\item Be mindful of late-comers. During group work, try to approach them and make sure they are aware of what is going on, and that they are engaged and working in a group.
		\item Try to keep an organized board. A disorganized and chaotic board is just as annoying to the students as a jumbled mess on a test paper that you're trying to grade is to you.
	\end{itemize}