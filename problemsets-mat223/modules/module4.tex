Let $\vec a$ and $\vec b$ be vectors rooted at the same point and let
$\theta$ denote the \emph{smaller} of the
two angles between them (so $0\leq \theta \leq \pi$).
The \emph{dot product}\index{dot product} of $\vec a$ and $\vec b$ is defined to be
\[
	\vec a\cdot \vec b=\norm{\vec a}\norm{\vec b}\cos \theta.
\]
We will call this the \emph{geometric definition of the dot product}.

\begin{center}
	\newcommand{\tikzAngleOfLine}{\tikz@AngleOfLine}
	  \def\tikz@AngleOfLine(#1)(#2)#3{%
	  \pgfmathanglebetweenpoints{%
	    \pgfpointanchor{#1}{center}}{%
	    \pgfpointanchor{#2}{center}}
	  \pgfmathsetmacro{#3}{\pgfmathresult}%
	  }
	\newcommand{\tikzMarkAngle}[3]{
	\tikzAngleOfLine#1#2{\AngleStart}
	\tikzAngleOfLine#1#3{\AngleEnd}
	\draw #1+(\AngleStart:0.35cm) arc (\AngleStart:\AngleEnd:0.35cm);
	}
	\usetikzlibrary{patterns,decorations.pathreplacing}
	\begin{tikzpicture}
		\coordinate (A) at (2,1);
		\coordinate (B) at (.5,2);
		\coordinate (O) at (0,0);

		\draw[->,thick,myred!60!white] (0,0) -- +(A) node [midway,below right] {$\vec a$};
		\draw[->,thick,mypink] (0,0) -- +(B) node [midway,above left] {$\vec b$};
		\tikzMarkAngle{(O)}{(A)}{(B)}
		\node at ($(O)+(50:.65)$) {$\theta$};
	\end{tikzpicture}
	\hspace{1cm}
	\begin{tikzpicture}
		\coordinate (A) at (-2,-1);
		\coordinate (B) at (.5,2);
		\coordinate (O) at (0,0);

		\draw[->,thick,myred!60!white] (0,0) -- +(A) node [midway,below right] {$\vec a$};
		\draw[->,thick,mypink] (0,0) -- +(B) node [midway,above left] {$\vec b$};
		\tikzMarkAngle{(O)}{(B)}{(A)}
		\node at ($(O)+(140:.65)$) {$\theta$};
	\end{tikzpicture}
\end{center}

The dot product is also sometimes called the \emph{scalar product} because
the result is a scalar.

Algebraically, we can define the dot product in terms of coordinates:
\[
	\matc{a_1\\a_2\\\vdots\\a_n}\cdot \matc{b_1\\b_2\\\vdots\\b_n}
	=a_1b_1+a_2b_2+\cdots+a_nb_n.
\]
We will call this the \emph{algebraic definition of the dot product}.

By switching between algebraic and geometric definitions, we can use the dot
product to find quantities that are otherwise difficult to find.
\begin{example}
	Find the angle between the vectors $\vec v=(1,2,3)$ and $\vec w=(1,1,-2)$.

	From the algebraic definition of the dot product, we know
	\[
		\vec v\cdot \vec w = 1(1)+2(1)+3(-2) = -3.
	\]
	From the geometric definition, we know
	\[
		\vec v\cdot \vec w=\norm{\vec v}\norm{\vec w}\cos\theta
		=\sqrt{14}\sqrt{6}\cos\theta=2\sqrt{21}\cos\theta.
	\]
	Equating the two definitions of $\vec v\cdot \vec w$, we see
	\[
		\cos\theta = \frac{-3}{2\sqrt{21}}
	\]
	and so $\theta=\arccos\Big(\tfrac{-3}{2\sqrt{21}}\Big)$.
\end{example}

The dot product has several interesting properties. Since the angle between
$\vec a$ and itself is $0$, the geometric definition of the dot product tells us
\[
	\vec a\cdot \vec a=\Norm{\vec a}\Norm{\vec a}\cos 0 = \Norm{\vec a}^2.
\]
In other words,
\[
	\Norm{\vec a}=\sqrt{\vec a\cdot \vec a},
\]
and so dot products can be used to compute the length of vectors\footnote{ Oftentimes in
non-geometric settings, the dot product between two vectors is defined first and then
the length of $\vec a$ is actually \emph{defined} to be $\sqrt{\vec a\cdot \vec a}$.}.

From the algebraic definition of the dot product, we can deduce several distributive laws.
Namely, for any vectors $\vec a$, $\vec b$, and $\vec c$ and any scalar $k$ we have
\[
	(\vec a+\vec b)\cdot\vec c = \vec a\cdot \vec c+\vec b\cdot \vec c\qquad
	\vec a\cdot(\vec b+\vec c)=\vec a\cdot \vec b+\vec a\cdot \vec c
\]
\[
	(k\vec a)\cdot \vec b = k(\vec a\cdot \vec b)=\vec a\cdot (k\vec b)
\]
and
\[
	\vec a\cdot \vec b=\vec b\cdot \vec a.
\]

\Heading{Orthogonality}
Recall that for vectors $\vec a$ and $\vec b$, the relationship $\vec a\cdot \vec b=0$
can hold for two reasons: (i) either $\vec a=\vec 0$, $\vec b=\vec 0$, or both
or (ii) $\vec a$ and $\vec b$ meet at $90^{\circ}$.  Thus, the dot product
can be used to tell if two vectors are perpendicular.  There is some strangeness
with the zero vector here, but it turns out this strangeness simplifies our lives
mathematically.

\SavedDefinitionRender{Orthogonal}

The definition of orthogonal encapsulates both the idea of two vectors forming
a right angle and the idea of one of them being $\vec 0$.

\medskip
Before we continue, let's pin down exactly what we mean by
the \emph{direction}\index{direction}\index{positive direction} of a vector.
There are many ways we could define
this term, but we'll go with the following.

\SavedDefinitionRender{Direction}

The vector $2\xhat$ points in the direction of $\xhat$ since
$\frac{1}{2}(2\xhat)=\xhat$.  Since $\frac{1}{2}>0$, $2\xhat$ also points
in the positive direction of $\xhat$. In contrast,
$-\xhat$ points in the direction $\xhat$ but not the positive direction of $\xhat$.

When it comes to the relationship between two vectors, there are two extremes: they
point in the same direction, or they are orthogonal. The dot product can be used to
tell you which of these cases you're in, and more than that, it can tell you to
what extent one vector points in the direction of another (even if they don't point in
the same direction).

\begin{example}
	Let $\vec a=\mat{1\\2}$, $\vec b=\mat{3\\3}$, $\vec c=\mat{2\\1}$, and $\vec v=\mat{3\\4}$. Which vector out of
	$\vec a$, $\vec b$, and $\vec c$ has a direction closest to the direction of $\vec v$?

	We would like to know when $\theta$, the angle between a pair of the given vectors, is smallest.
	This is equivalent to finding when $\cos\theta$ is closest to 1 (since $\cos0=1$).
    	By equating the geometric and algebraic definitions of the dot product, we know
	    \[
        	\cos\theta = \frac{\vec p\cdot\vec q}{\norm{\vec p}\norm{\vec q}}.
	    \]

    Let $\alpha$, $\beta$, $\gamma$ between the vector $\vec v$ and
	the vectors $\vec a$, $\vec b$, $\vec c$, respectively. Computing, we find
	\begin{align*}
	    \cos \alpha &= \frac{3+8}{5\sqrt{5}}=\frac{11\sqrt{5}}{25} \approx 0.9838699101 \\
		\cos \beta &= \frac{9+12}{5\sqrt{18}} =\frac{7\sqrt{2}}{10} \approx 0.989949437\\
	    \cos \gamma &= \frac{6+4}{5\sqrt{5}}=\frac{2\sqrt{5}}{5} \approx 0.894427191.
	\end{align*}
	Since $\cos \beta$ is the closest to $1$, we know $\vec b$ has a direction closest to that of $\vec v$.
\end{example}


\Heading{Normal Form of Lines and Planes}

Let $\vec n=\mat{1\\2}$. If a vector $\vec v=\mat{v_1\\v_2}$ is orthogonal to $\vec n$, then
\[
	\vec n\cdot\vec v=v_1+2v_2=0,
\]
and so $v_1=-2v_2$. In other words, $\vec v$ is orthogonal to $\vec n$ exactly when $\vec v\in\Span\Set*{\mat{-2\\1}}$.
What have we learned? The set of all vectors orthogonal to $\vec n$ forms a line $\ell=\Span\Set*{\mat{-2\\1}}$. In
this case, we call $\vec n$ a \emph{normal vector} for $\ell$.

\begin{center}
	\begin{tikzpicture}
		\def\dotMarkRightAngle[size=#1](#2,#3,#4){%
		 \draw ($(#3)!#1!(#2)$) --
		       ($($(#3)!#1!(#2)$)!#1!90:(#2)$) --
		       ($(#3)!#1!(#4)$);
		 \path (#3) -- ($($(#3)!#1!(#2)$)!#1!90:(#2)$);
		}
		\begin{axis}[
		    anchor=origin,
		    name=plot1,
		    disabledatascaling,
		    xmin=-3,xmax=3,
		    ymin=-2,ymax=2,
			xtick={-4,...,4},
			ytick={-2,...,4},
		    x=1cm,y=1cm,
		    grid=both,
		    grid style={line width=.1pt, draw=gray!10},
		    %major grid style={line width=.2pt,draw=gray!50},
		    axis lines=middle,
		    minor tick num=0,
		    enlargelimits={abs=0.5},
		    axis line style={latex-latex},
		    ticklabel style={font=\tiny,fill=white},
		    xlabel style={at={(ticklabel* cs:1)},anchor=north west},
		    ylabel style={at={(ticklabel* cs:1)},anchor=south west}
		]
			\coordinate (A) at (1,2);
			\coordinate (O) at (0,0);
			\coordinate (B) at (2,-1);
			\dotMarkRightAngle[size=6pt](B,O,A);
			\draw[mypink, thick] ($-2*(-2,1)$) -- ($2*(-2,1)$);
			\draw[mygreen, thick, ->] (0,0) -- (1,2) node[midway, below right] {$\vec n$};
		\end{axis}
		\node[mypink, above right] at (2,-1.2) {$\ell=\Span\Set*{\mat{-2\\1}}$};
	\end{tikzpicture}
\end{center}

\SavedDefinitionRender{NormalVector}

In $\R^2$, normal vectors provide yet another way to describe lines, including lines which don't pass
through the origin.

Let $n=\mat{1\\2}$ as before, and fix $\vec p=\mat{1\\1}$. If we draw the set of all vectors
orthogonal to $\vec n$ but \emph{root all the vectors at $\vec p$}, again we get a line, but this
time the line passes through $\vec p$.

\begin{center}
	\begin{tikzpicture}
		\def\dotMarkRightAngle[size=#1](#2,#3,#4){%
		 \draw ($(#3)!#1!(#2)$) --
		       ($($(#3)!#1!(#2)$)!#1!90:(#2)$) --
		       ($(#3)!#1!(#4)$);
		 \path (#3) -- ($($(#3)!#1!(#2)$)!#1!90:(#2)$);
		}
		\begin{axis}[
		    anchor=origin,
		    name=plot1,
		    disabledatascaling,
		    xmin=-2,xmax=4,
		    ymin=-1,ymax=3,
			xtick={-4,...,4},
			ytick={-2,...,4},
		    x=1cm,y=1cm,
		    grid=both,
		    grid style={line width=.1pt, draw=gray!10},
		    %major grid style={line width=.2pt,draw=gray!50},
		    axis lines=middle,
		    minor tick num=0,
		    enlargelimits={abs=0.5},
		    axis line style={latex-latex},
		    ticklabel style={font=\tiny,fill=white},
		    xlabel style={at={(ticklabel* cs:1)},anchor=north west},
		    ylabel style={at={(ticklabel* cs:1)},anchor=south west}
		]
			\begin{scope}[cm={1,0,0,1,(1,1)}]
				\coordinate (A) at (1,2);
				\coordinate (O) at (0,0);
				\coordinate (B) at (2,-1);
				\dotMarkRightAngle[size=6pt](B,O,A);
				\draw[mypink, thick] ($-2*(-2,1)$) -- ($2*(-2,1)$);
				\draw[mygreen, thick, ->] (0,0) -- (1,2) node[midway, below right] {$\vec n$};
				\fill[myorange] (0,0) node[below left] {$\vec p$} circle (2pt);
			\end{scope}
		\end{axis}
		\node[mypink, above right] at (3,0) {$\ell_2=\Span\Set*{\mat{-2\\1}}+\Set*{\mat{1\\1}}$};
	\end{tikzpicture}
\end{center}

In fact, the line we get is $\ell_2=\Span\Set*{\mat{-2\\1}}+\Set*{\mat{1\\1}}=\ell+\Set{\vec p}$, which is just $\ell$ (the
parallel line through the origin) translated by $\vec p$.

Let's relate this to dot products and normal vectors. By definition, for every $\vec v\in \ell$, we have $\vec n\cdot \vec v=0$.
Since $\ell_2$ is a translate of $\ell$ by $\vec p$, we deduce the relationship that for every $\vec v\in \ell_2$,
\[
	\vec n\cdot(\vec v-\vec p) = 0.
\]
When a line is expressed as above, we say it is expressed in \emph{normal form}.

\SavedDefinitionRender{NormalFormofaLine}


\bigskip
What about in $\R^3$?
Fix a non-zero vector $\vec n\in\R^3$ and let $\mathcal Q\subseteq\R^3$ be the set of vectors orthogonal to $\vec n$.
$\mathcal Q$ is a plane through the origin, and again, we call $\vec n$ a \emph{normal vector}\index{normal vector} of the plane $\mathcal Q$.


\begin{center}
\begin{tikzpicture}
	\newcommand{\RightAngle}[4][5pt]{%
        \draw ($#3!#1!#2$)
        --($ #3!2!($($#3!#1!#2$)!.5!($#3!#1!#4$)$) $)
        --($#3!#1!#4$) ;
        }
    \begin{axis}[grid=major,view={20}{40},z buffer=sort,
	    width=12cm,
	    scale mode=scale uniformly,
	    zmin=-5,zmax=5,xmin=-10,xmax=10,ymin=-10,ymax=10,
	    xticklabels={,,}, yticklabels={,,}, zticklabels={,,},
	    xtick={-10,-5,...,10}, ytick={-10,-5,...,10}
	    ]
		\addplot3 [data cs=cart,surf,domain=-10:10,samples=2, opacity=0.5]
		{.25*x+.25*y};
		\coordinate (A) at (axis cs:-4,-4,-2);
		\coordinate (N) at (axis cs:-6,-6,6);
		\coordinate (B) at (axis cs:4,4,2);
		\coordinate (C) at (axis cs:-4,4,0);

		\draw [mypink,fill] (A) circle (1.5pt) node [below right] {$\vec 0$};
		\draw [->, thick] (A) -- (B) node [midway,below right] {$\vec d_1$};
		\draw [->, thick] (A) -- (C) node [midway,above left] {$\vec d_2$};
		\draw [->, thick] (A) -- (N) node [midway,left] {$\vec n$};
		\RightAngle{(N)}{(A)}{(C)}
		%\RightAngle{(N)}{(A)}{(B)}
    \end{axis}
  \end{tikzpicture}
\end{center}

In a similar way to the line, $\mathcal Q$ is the set of solutions to $\vec n\cdot \vec x=0$.
And, for any $\vec p\in\R^3$, the translated plane $\mathcal Q+\Set{\vec p}$ is the solution set to
\[
	\vec n\cdot (\vec x-\vec p)=0.
\]
Thus, we see planes in $\R^3$ have a normal form just like lines in $\R^2$ do.

\begin{example}
	Find vector form and normal form of the plane $\mathcal P$ passing
	through the points $A=(1,0,0)$, $B=(0,1,0)$ and $C=(0,0,1)$.

	To find vector form of $\mathcal P$, we need a point on the plane and
	two direction vectors.  We have three points on the plane, so we can
	obtain two direction vectors by subtracting these points in different ways.
	Let
	\[
		\vec d_1=\overrightarrow{AB} = \mat{-1\\1\\0}\qquad\vec d_2=\overrightarrow{AC}=
		\mat{-1\\0\\1}.
	\]
	Using the point $A$, we may now express $\mathcal P$ in vector form by
	\[
		\mat{x\\y\\z} = t\mat{-1\\1\\0}+s\mat{-1\\0\\1}+\mat{1\\0\\0}.
	\]

	To write $\mathcal P$ in normal form, we need to find a normal vector for $\mathcal P$.  By inspection,
	we see that $\vec n=(1,1,1)$ is a normal vector to $\mathcal P$.  (If we weren't
	so insightful, we could also solve the system $\vec n\cdot \vec d_1=0$ and $\vec n\cdot\vec d_2=0$ to find a
	normal vector.)  Now, we may express $\mathcal P$ in normal form as
	\[
		\mat{1\\1\\1}\cdot\left(\mat{x\\y\\z}-\mat{1\\0\\0}\right)=0.
	\]
\end{example}

In $\R^2$, only lines have a normal form, and in $\R^3$ only planes have a normal form. In general,
we call objects in $\R^n$ which have a normal form \emph{hyperplanes}.

\SavedDefinitionRender{Hyperplane}

Hyperplanes always have dimension one less than the space they're contained in. So, hyperplanes in
$\R^2$ are (one-dimensional) lines, hyperplanes in $\R^3$ are regular (two-dimensional) planes,
and hyperplanes in $\R^4$ are (three-dimensional) volumes.

\Heading{Hyperplanes and Linear Equations}

Suppose $\vec n,\vec p\in \R^3$ and $\vec n\neq \vec 0$. Then, solutions to
\[
		\vec n\cdot(\vec x-\vec p)=0
\]
define a plane $\mathcal P$. But, $\vec n\cdot (\vec x-\vec p)=0$
if and only if
\[
		\vec n\cdot\vec x = \vec n\cdot \vec p = \alpha.
\]
Since $\vec n$ and $\vec p$ are fixed, $\alpha$ is a constant. Expanding using
coordinates, we see
\[
		\vec n\cdot(\vec x-\vec p)=\vec n\cdot\vec x-\alpha=
		n_xx+n_yy+n_zz-\alpha=0
\]
and so $\mathcal P$ is the set of solutions to
\begin{equation}
	\label{EQScalarForm}
		n_xx+n_yy+n_zz=\alpha.
\end{equation}
Equation \eqref{EQScalarForm} is sometimes
called \emph{scalar form}\index{scalar form of a plane}
of a plane.  For us, it will not be important to distinguish between scalar and normal form, but
what is important is that we can use the row reduction algorithm to write the complete solution
to \eqref{EQScalarForm}, and this complete solution will necessarily be written in vector form.

\begin{example}
	Let $\mathcal Q\subseteq \R^3$ be the plane passing through $\vec p$ and with normal vector $\vec n$ where
	\[
		\vec p=\mat{1\\1\\0}\qquad\text{and}\qquad\vec n=\mat{1\\1\\1}.
	\]
	Write $\mathcal Q$ in vector form.

	We know $\mathcal Q$ is the set of solutions to $\vec n\cdot (\vec x-\vec p)=0$. In scalar form, this equation
	becomes
	\[
		\vec n\cdot (\vec x-\vec p)=\vec n\cdot \vec x-\vec n\cdot \vec p = x+y+z-2=0.
	\]
	Rearranging, we see $\mathcal Q$ is the set of all solutions to
	\[
		x+y+z=2.
	\]
	Using the row reduction algorithm to write the complete solution\footnote{ In some sense, this is overkill
	because the equation corresponds to the augmented matrix $\mat{1&1&1&\!\!\!\!|\!\!\!\!&2}$, which is already row reduced.}, we get
	\[
		\mat{x\\y\\z}=t\mat{-1\\1\\0}+s\mat{-1\\0\\1}+\mat{2\\0\\0}.
	\]
\end{example}
