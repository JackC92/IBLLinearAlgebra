\documentclass[11pt]{exam}
\usepackage{amsmath}
\usepackage{amssymb}
\usepackage{graphicx}
\usepackage{enumitem}
\usepackage{amsfonts}
\usepackage{amssymb}
\usepackage{xparse}
\usepackage{ifthen}
\usepackage{geometry}
\noprintanswers

\newcommand {\DS} [1] {${\displaystyle #1}$}
\newcommand{\answer}[1]{{\bf Answer:} \; #1}
\newcommand{\vv}{\vspace{.2cm}}
\newcommand{\vvv}{\vspace{6cm}}

\newcommand{\ul}{$\underline{\phantom{xxx}}$}
\newcommand{\ull}{\underline{\phantom{xxx}}}
\newcommand{\xh}{\hat{\bf x}}
\newcommand{\yh}{\hat{\bf y}}
\newcommand{\zh}{\hat{\bf z}}
\newcommand{\R}{\mathbb{R}}
\newcommand{\C}{\mathbb{C}}
\newcommand{\Z}{\mathbb{Z}}
\newcommand{\N}{\mathbb{N}}
\newcommand{\proj}{\mathrm{proj}}
\newcommand{\mat}[1]{\begin{bmatrix}#1\end{bmatrix}}
\newcommand{\floor}[1]{\lfloor #1 \rfloor}

\renewcommand{\span}{\mathrm{span}\,}
\newcommand{\rref}{\mathrm{rref}}
\newcommand{\rank}{\mathrm{rank}}
\newcommand{\nnul}{\mathrm{nullity}}

\pagestyle{empty}


%%%%%%%%%%%%%%%%%%%%%%%%%%%%%%%%%%%%%%%%%
%  Edit course information here
%%%%%%%%%%%%%%%%%%%%%%%%%%%%%%%%%%%%%%%%%

\newcommand{\mthCourse}{MATH 110}
\newcommand{\mthTerm}{Fall 2013}
\newcommand{\mthTutorialNumber}{10}
\newcommand{\mthDate}{November 20, 2013}


%%%%%%%%%%%%%%%%%%%%%%%%%%%%%%%%%%%%%%%%%
\topmargin -1in
\textheight 10in

\begin{document}


%%%%%%%%%%%%%%%%%%%%%%%%%%%%%%%%%%%%%%%%%%%%%%%
% Main Questions
%%%%%%%%%%%%%%%%%%%%%%%%%%%%%%%%%%%%%%%%%%%%%%%
{\large
	\begin{center}
		{\bf \mthCourse, \mthTerm}\\ 
		{\bf Tutorial \#\mthTutorialNumber}\\
		{\bf \mthDate}
	\end{center}
}

\section*{Today's main problems}
		$\mathcal V=\{\vec v_1,\vec v_2,\vec v_3\}$ where
		\[
			\vec v_1=\mat{0\\2\\0}\qquad
			\vec v_2=\mat{-4\\0\\3}\qquad
			\vec v_3=\mat{3\\0\\4}
		\]
\begin{enumerate}
	\item 
	\begin{enumerate}
		\item Show that $\mathcal V$ is an orthogonal basis.  Is it an
		orthonormal basis?

		\item Create an orthonormal basis $\mathcal B=\{\vec b_1,\vec b_2,\vec b_3\}$ 
		by ``fixing'' the vectors in $\mathcal V$ so they are orthonormal.

		\item Write the vector $\vec w=\mat{1\\1\\1}$ as a linear combination of
			vectors in the $\mathcal B$ basis.
	\end{enumerate}

\item Consider the three planes $\mathcal P_1=\span \{\vec v_1,\vec v_2\}$, 
	$\mathcal P_2=\span \{\vec v_2,\vec v_3\}$, $\mathcal P_3=\span \{\vec v_1,\vec v_3\}$.
	\begin{enumerate}
		\item Find the normal vectors of the planes $\mathcal P_1,
			\mathcal P_2,$ and $\mathcal P_3$.
		\item Find the projection of $\vec w=\mat{1\\1\\1}$ onto $\mathcal P_1,
			\mathcal P_2,$ and $\mathcal P_3$.
		\item Write $\vec w=\vec a+\vec b$ where $\vec a\in \mathcal P_1$ and
			$\vec b$ is perpendicular to $\mathcal P_1$.
	\end{enumerate}
	

\end{enumerate}
\subsection*{Further Questions}
\begin{enumerate}[resume]
	
	\item 
	Let $B=[\vec b_1|\vec b_2|\vec b_3]$ and $V=[\vec v_1|\vec v_2|\vec v_3]$
	where $\vec b_i$ and $\vec v_i$ are from problem 1.
	\begin{enumerate}
		\item Compute $B^{-1}$ (Hint, this is easy. No computation required!)
		\item Compute $V^{-1}$ (This requires a little thinking, but not
			much computation).
	\end{enumerate}
	\item $V$ is a subspace of $\R^9$ and $P$ is a matrix that projects vectors
		in $\R^9$ onto $V$.  Further, rank$(P)=3$.
		\begin{enumerate}
			\item How many vectors are in a basis for $V$?
			\item Let $\vec v\in \R^9$ and $\vec w = \vec v-P\vec v$. What
				is the angle between $\vec w$ and any vector in $V$?
			\item What is rank$(I-P)$?
		\end{enumerate}
\end{enumerate}




%%%%%%%%%%%%%%%%%%%%%%%%%%%%%%%%%%%%%%%%%%%%%%%
% Challenge questions
%%%%%%%%%%%%%%%%%%%%%%%%%%%%%%%%%%%%%%%%%%%%%%%
\newpage
{
	\begin{center}
		{\bf \mthCourse, \mthTerm}\\ 
		{\bf Tutorial \#\mthTutorialNumber}\\
		{\bf \mthDate}
	\end{center}
}

\section*{Challenge questions}

	Recall that the trace of a matrix is the sum of the diagonal entries
	and that trace$(A)=$trace$(B)$ if $A$ and $B$ are similar.
\begin{enumerate}[resume]
	\item Let $P$ be a projection matrix.  Show that trace$(P)$ is always an integer (Hint, think
	about what a projection matrix must be similar to).
	\item Show that in fact trace$(P)$ is the dimension of the subspace that $P$ projects onto.

\end{enumerate}



%%%%%%%%%%%%%%%%%%%%%%%%%%%%%%%%%%%%%%%%%%%%%%%
% TA instructions
%%%%%%%%%%%%%%%%%%%%%%%%%%%%%%%%%%%%%%%%%%%%%%%
\newpage
{\small
	\begin{center}
		{\bf \mthCourse, \mthTerm}\\ 
		{\bf Tutorial \#\mthTutorialNumber. Instructions for TAs}
	\end{center}
}

\subsection*{Objectives}


\subsection*{Hidden objectives}


\subsection*{Suggestions}

\subsection*{Wrapup}
	Choose a question that most of the class has started but not yet finished,
	or a question that people particularly struggled with.

\subsection*{Solutions}
\begin{enumerate}
	\item
\end{enumerate}
	

\end{document}
