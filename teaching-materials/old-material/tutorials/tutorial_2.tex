\documentclass[11pt]{exam}
\usepackage{amsmath}
\usepackage{amssymb}
\usepackage{graphicx}
\usepackage{enumitem}
\usepackage{amsfonts}
\usepackage{amssymb}
\usepackage{xparse}
\usepackage{ifthen}
\usepackage{geometry}
\noprintanswers

\newcommand {\DS} [1] {${\displaystyle #1}$}
\newcommand{\answer}[1]{{\bf Answer:} \; #1}
\newcommand{\vv}{\vspace{.2cm}}
\newcommand{\vvv}{\vspace{6cm}}

\newcommand{\ul}{$\underline{\phantom{xxx}}$}
\newcommand{\ull}{\underline{\phantom{xxx}}}
\newcommand{\xh}{\hat{\bf x}}
\newcommand{\yh}{\hat{\bf y}}
\newcommand{\zh}{\hat{\bf z}}
\newcommand{\R}{\mathbb{R}}
\newcommand{\C}{\mathbb{C}}
\newcommand{\Z}{\mathbb{Z}}
\newcommand{\N}{\mathbb{N}}
\newcommand{\proj}{\mathrm{proj}}
\newcommand{\mat}[1]{\begin{bmatrix}#1\end{bmatrix}}
\newcommand{\floor}[1]{\lfloor #1 \rfloor}

\pagestyle{empty}


%%%%%%%%%%%%%%%%%%%%%%%%%%%%%%%%%%%%%%%%%
%  Edit course information here
%%%%%%%%%%%%%%%%%%%%%%%%%%%%%%%%%%%%%%%%%

\newcommand{\mthCourse}{MATH 110}
\newcommand{\mthTerm}{Fall 2013}
\newcommand{\mthTutorialNumber}{2}
\newcommand{\mthDate}{September 18, 2013}


%%%%%%%%%%%%%%%%%%%%%%%%%%%%%%%%%%%%%%%%%


\begin{document}


%%%%%%%%%%%%%%%%%%%%%%%%%%%%%%%%%%%%%%%%%%%%%%%
% Main Questions
%%%%%%%%%%%%%%%%%%%%%%%%%%%%%%%%%%%%%%%%%%%%%%%
{\large
	\begin{center}
		{\bf \mthCourse, \mthTerm}\\ 
		{\bf Tutorial \#\mthTutorialNumber}\\
		{\bf \mthDate}
	\end{center}
}

\section*{Today's main problems}

The line $L$ is defined by the equation $y=-2x+3$.  
\begin{enumerate}
	\item Find a constant $k$ so that the equation
	\[
		kx+2y=6
	\]
	also describes $L$
	\item Write $L$ in vector form and write down a normal vector for $L$.
	\item Let $A$ be the point on $L$ closest to the origin and let $\vec v$
		be the vector from the origin to $A$. 
	Draw $L$, its normal vector, and $\vec v$. 
	Clearly label all right angles.
	\item Find the distance from $A$ to the origin and the coordinates of $A$.
\end{enumerate}


\section*{Further questions}

\begin{enumerate}[resume]
	\item  Find the distance from the point $p_1=(2,1,-3)$ to
	the plane with equation $3x-y+4z=1$.

	\item Find vector and parametric form of the line parallel to
	$\vec u=\mat{2\\-1\\0}$ that passes through the point $p=(1,-1,3)$.
	Does this line have a normal form?
\end{enumerate}

%%%%%%%%%%%%%%%%%%%%%%%%%%%%%%%%%%%%%%%%%%%%%%%
% Challenge questions
%%%%%%%%%%%%%%%%%%%%%%%%%%%%%%%%%%%%%%%%%%%%%%%
\newpage
{
	\begin{center}
		{\bf \mthCourse, \mthTerm}\\ 
		{\bf Tutorial \#\mthTutorialNumber}\\
		{\bf \mthDate}
	\end{center}
}

\section*{Challenge questions}

\begin{enumerate}[resume]

	\item What is the distance between the sphere of radius $2$ centered 
	at $p=(1,1,1)$ and the plane with normal vector $\vec n=\mat{1\\1\\0}$
	passing through the point $(-1,-1,-1)$?


	\item  Find the distance between the lines 
	\[
		L_1(t)=t\mat{1\\1\\-2}+\mat{3\\3\\-3}
	\] and \[
		L_2(t)=t\mat{-2\\1\\0}+\mat{0\\-1\\0}.
	\]
\end{enumerate}



%%%%%%%%%%%%%%%%%%%%%%%%%%%%%%%%%%%%%%%%%%%%%%%
% TA instructions
%%%%%%%%%%%%%%%%%%%%%%%%%%%%%%%%%%%%%%%%%%%%%%%
\newpage
{\small
	\begin{center}
		{\bf \mthCourse, \mthTerm}\\ 
		{\bf Tutorial \#\mthTutorialNumber. Instructions for TAs}
	\end{center}
}

\subsection*{Objectives}

	Normal vectors and direction vectors are confusing at first, but allow
	you to solve problems that used to be quite hard.  We'd like
	to become familiar with them.

\subsection*{Hidden objectives}
	
	There are formulas for all of these things in your textbook, but if you memorize
	them, you'll fail at more sophisticated problems.  We'd like to see the geometry
	of these objects and use our basic tools of vector addition, dot products, and
	projection to solve all of these problems.

\subsection*{Suggestions}
	The days main problem will probably take the whole class time.  For an intro,
	you could remind everyone of normal, vector, and parametric form, and emphasise
	that vector and parametric form are essentially the same thing but
	with different notation.

\subsection*{Wrapup}
	Choose a question that most of the class has started but not yet finished,
	or a question that people particularly struggled with.

\subsection*{Solutions}
\begin{enumerate}
	\item
\end{enumerate}
	

\end{document}
