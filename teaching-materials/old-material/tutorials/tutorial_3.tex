\documentclass[11pt]{exam}
\usepackage{amsmath}
\usepackage{amssymb}
\usepackage{graphicx}
\usepackage{enumitem}
\usepackage{amsfonts}
\usepackage{amssymb}
\usepackage{xparse}
\usepackage{ifthen}
\usepackage{geometry}
\noprintanswers

\newcommand {\DS} [1] {${\displaystyle #1}$}
\newcommand{\answer}[1]{{\bf Answer:} \; #1}
\newcommand{\vv}{\vspace{.2cm}}
\newcommand{\vvv}{\vspace{6cm}}

\newcommand{\ul}{$\underline{\phantom{xxx}}$}
\newcommand{\ull}{\underline{\phantom{xxx}}}
\newcommand{\xh}{\hat{\bf x}}
\newcommand{\yh}{\hat{\bf y}}
\newcommand{\zh}{\hat{\bf z}}
\newcommand{\R}{\mathbb{R}}
\newcommand{\C}{\mathbb{C}}
\newcommand{\Z}{\mathbb{Z}}
\newcommand{\N}{\mathbb{N}}
\newcommand{\proj}{\mathrm{proj}}
\newcommand{\mat}[1]{\begin{bmatrix}#1\end{bmatrix}}
\newcommand{\floor}[1]{\lfloor #1 \rfloor}

\pagestyle{empty}


%%%%%%%%%%%%%%%%%%%%%%%%%%%%%%%%%%%%%%%%%
%  Edit course information here
%%%%%%%%%%%%%%%%%%%%%%%%%%%%%%%%%%%%%%%%%

\newcommand{\mthCourse}{MATH 110}
\newcommand{\mthTerm}{Fall 2013}
\newcommand{\mthTutorialNumber}{3}
\newcommand{\mthDate}{September 25, 2013}


%%%%%%%%%%%%%%%%%%%%%%%%%%%%%%%%%%%%%%%%%


\begin{document}


%%%%%%%%%%%%%%%%%%%%%%%%%%%%%%%%%%%%%%%%%%%%%%%
% Main Questions
%%%%%%%%%%%%%%%%%%%%%%%%%%%%%%%%%%%%%%%%%%%%%%%
{\large
	\begin{center}
		{\bf \mthCourse, \mthTerm}\\ 
		{\bf Tutorial \#\mthTutorialNumber}\\
		{\bf \mthDate}
	\end{center}
}

\section*{Today's main problems}

\begin{enumerate}
	\item Consider the system
	\begin{align*}
		x-3y-5z &= 0\\
		y+z &= 3.
	\end{align*}
	\begin{enumerate}
		\item Row reduce the corresponding augmented matrix.
		\item Identify the solution as a point, line, or plane.
		\item Write the solution in vector form.
	\end{enumerate}
	\item Consider the system
	\begin{align*}
		x+hy &= 2\\
		4x+8y &= k,
	\end{align*}
	which has two unknown quantities: $h$, $k$.  Find values of $h$ and
	$k$ so that the system has
	\begin{enumerate}
		\item no solution
		\item a unique solution
		\item many solutions
	\end{enumerate}
\end{enumerate}


\section*{Further questions}

\begin{enumerate}[resume]
	\item  Consider the system
	\begin{align*}
		x-2y-z &=0\\
		-2x+4y+5z &=3\\
		3x-6y-6z &=2.
	\end{align*}
	\begin{enumerate}
		\item Row reduce the corresponding augmented matrix.
		\item Identify the solution as a point, line, or plane, if it exists.
		\item Write the solution in vector form, if it exists.
	\end{enumerate}
\end{enumerate}

%%%%%%%%%%%%%%%%%%%%%%%%%%%%%%%%%%%%%%%%%%%%%%%
% Challenge questions
%%%%%%%%%%%%%%%%%%%%%%%%%%%%%%%%%%%%%%%%%%%%%%%
\newpage
{
	\begin{center}
		{\bf \mthCourse, \mthTerm}\\ 
		{\bf Tutorial \#\mthTutorialNumber}\\
		{\bf \mthDate}
	\end{center}
}

\section*{Challenge questions}

\begin{enumerate}[resume]

	\item Sketch an example of a system in $\R^2$ that has no solution.
	\item Sketch an example of a system in $\R^3$ that has no solution.
	\item $P_1$, $P_2$, and $P_3$ are planes in $\R^3$.  The normal vectors
	for $P_1$ and $P_2$ are $\vec n_1=\mat{1\\1\\2}$ and $\vec n_2=\mat{-1\\1\\2}$,
	and the normal vector for $P_3$ is $\vec n_3$ which we won't specify
	exactly.
	Further, $P_1$ and $P_2$ contain
	the origin and $P_3$ passes through the point $(1,1,2)$ and does not contain the origin.
	\begin{enumerate}
		\item Is the point $p=(0,-4,2)$ in the intersection of $P_1$ and $P_2$?
		\item If $\vec n_3\cdot p = 2.7$, do $P_1$, $P_2$, and $P_3$ all intersect?
		\item If $\vec n_3\cdot p = 0$, do $P_1$, $P_2$, and $P_3$ all intersect?
	\end{enumerate}

\end{enumerate}



%%%%%%%%%%%%%%%%%%%%%%%%%%%%%%%%%%%%%%%%%%%%%%%
% TA instructions
%%%%%%%%%%%%%%%%%%%%%%%%%%%%%%%%%%%%%%%%%%%%%%%
\newpage
{\small
	\begin{center}
		{\bf \mthCourse, \mthTerm}\\ 
		{\bf Tutorial \#\mthTutorialNumber. Instructions for TAs}
	\end{center}
}

\subsection*{Objectives}

	There are some very computational tools that we lean on a lot
	in Linear Algebra.  One of these tools is solving systems using row 
	reduction.  Though not hard, the only way to become proficient with
	these tools is to use them, so let's practice.

	In particular, the idea of using free variables to write
	the solution to a system of equations can be confusing, so 
	we need extra practice with this.

\subsection*{Hidden objectives}
	

\subsection*{Suggestions}

\subsection*{Wrapup}
	Choose a question that most of the class has started but not yet finished,
	or a question that people particularly struggled with.

\subsection*{Solutions}
\begin{enumerate}
	\item
\end{enumerate}
	

\end{document}
