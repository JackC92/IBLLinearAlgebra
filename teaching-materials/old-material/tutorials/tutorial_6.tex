\documentclass[11pt]{exam}
\usepackage{amsmath}
\usepackage{amssymb}
\usepackage{graphicx}
\usepackage{enumitem}
\usepackage{amsfonts}
\usepackage{amssymb}
\usepackage{xparse}
\usepackage{ifthen}
\usepackage{geometry}
\noprintanswers

\newcommand {\DS} [1] {${\displaystyle #1}$}
\newcommand{\answer}[1]{{\bf Answer:} \; #1}
\newcommand{\vv}{\vspace{.2cm}}
\newcommand{\vvv}{\vspace{6cm}}

\newcommand{\ul}{$\underline{\phantom{xxx}}$}
\newcommand{\ull}{\underline{\phantom{xxx}}}
\newcommand{\xh}{\hat{\bf x}}
\newcommand{\yh}{\hat{\bf y}}
\newcommand{\zh}{\hat{\bf z}}
\newcommand{\R}{\mathbb{R}}
\newcommand{\C}{\mathbb{C}}
\newcommand{\Z}{\mathbb{Z}}
\newcommand{\N}{\mathbb{N}}
\newcommand{\proj}{\mathrm{proj}}
\newcommand{\mat}[1]{\begin{bmatrix}#1\end{bmatrix}}
\newcommand{\floor}[1]{\lfloor #1 \rfloor}

\pagestyle{empty}


%%%%%%%%%%%%%%%%%%%%%%%%%%%%%%%%%%%%%%%%%
%  Edit course information here
%%%%%%%%%%%%%%%%%%%%%%%%%%%%%%%%%%%%%%%%%

\newcommand{\mthCourse}{MATH 110}
\newcommand{\mthTerm}{Fall 2013}
\newcommand{\mthTutorialNumber}{6}
\newcommand{\mthDate}{October 16, 2013}


%%%%%%%%%%%%%%%%%%%%%%%%%%%%%%%%%%%%%%%%%
\topmargin -1in
\textheight 10in

\begin{document}


%%%%%%%%%%%%%%%%%%%%%%%%%%%%%%%%%%%%%%%%%%%%%%%
% Main Questions
%%%%%%%%%%%%%%%%%%%%%%%%%%%%%%%%%%%%%%%%%%%%%%%
{\large
	\begin{center}
		{\bf \mthCourse, \mthTerm}\\ 
		{\bf Tutorial \#\mthTutorialNumber}\\
		{\bf \mthDate}
	\end{center}
}

\section*{Today's main problems}

\begin{enumerate}
	\item Find the inverse of $A=\mat{1&2&3\\2&5&3\\1&0&8}$ if it exists
		or show that it is not invertible.
	\item Solve the system
		\[
			\begin{array}{rl}
				x+2y+3z &=3\\
				2x+5y+3z &=1\\
				x+8z &= 3
			\end{array}
		\]
	\item For an unknown matrix $B$, we know 
		\[
			AB=\mat{-6&12&20\\-5&21&37\\-18&16&23}.
		\]
		Find $B$.
	\item For an unknown matrix $C$, we know
		\[
			CA = \mat{-1 &-3& 0\\0&-3&8\\0&0&0}.
		\]
		Find $C$.

\end{enumerate}
\subsection*{Further Questions}
\begin{enumerate}[resume]
	\item The solution to $D\vec x=\mat{1\\0\\0}$ is $\mat{-1\\2\\1}$;
		the solution to $D\vec x=\mat{0\\1\\0}$ is $\mat{1\\0\\0}$;
		and 
		the solution to $D\vec x=\mat{0\\0\\1}$ is $\mat{-3\\7\\-1}$.
		Find $D$ and $D^{-1}$ if possible or explain why it cannot be done.
	\item A solution to $E\vec x=\mat{1\\0\\0}$ is $\mat{-1\\2\\1}$;
		a solution to $E\vec x=\mat{0\\1\\0}$ is $\mat{1\\0\\0}$;
		and 
		$E\vec x=\mat{0\\0\\1}$ has no solution.
		Find $E$ and $E^{-1}$ if possible or explain why it cannot be done.
\end{enumerate}




%%%%%%%%%%%%%%%%%%%%%%%%%%%%%%%%%%%%%%%%%%%%%%%
% Challenge questions
%%%%%%%%%%%%%%%%%%%%%%%%%%%%%%%%%%%%%%%%%%%%%%%
\newpage
{
	\begin{center}
		{\bf \mthCourse, \mthTerm}\\ 
		{\bf Tutorial \#\mthTutorialNumber}\\
		{\bf \mthDate}
	\end{center}
}

\section*{Challenge questions}

\begin{enumerate}[resume]
	\item For some unknown column vector $v$, $V=vv^T$.  Could $V$ ever be invertible?
	\item For some unknown row vector $w$, $W=ww^T$.  Could $W$ ever be invertible?
	\item Suppose that $A$ and $B$ are invertible symmetric matrices with
		$AB=BA$.  Show that $C=AB^{-1}$ is symmetric.

\end{enumerate}



%%%%%%%%%%%%%%%%%%%%%%%%%%%%%%%%%%%%%%%%%%%%%%%
% TA instructions
%%%%%%%%%%%%%%%%%%%%%%%%%%%%%%%%%%%%%%%%%%%%%%%
\newpage
{\small
	\begin{center}
		{\bf \mthCourse, \mthTerm}\\ 
		{\bf Tutorial \#\mthTutorialNumber. Instructions for TAs}
	\end{center}
}

\subsection*{Objectives}

	Practice computing matrix inverses and using them to solve matrix equations,
	with special care taken to remember that when solving matrix equations, multiplying
	on the left is different than multiplying on the right.

\subsection*{Hidden objectives}

	Matrix inverses are a nice way to consolidate all the information
	we know about solving a system of equations into one object.  Let's
	see how they relate to some things we've done before so that some time
	in the future it can `click.'

\subsection*{Suggestions}

\subsection*{Wrapup}
	Choose a question that most of the class has started but not yet finished,
	or a question that people particularly struggled with.

\subsection*{Solutions}
\begin{enumerate}
	\item
\end{enumerate}
	

\end{document}
