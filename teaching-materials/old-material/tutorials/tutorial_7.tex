\documentclass[11pt]{exam}
\usepackage{amsmath}
\usepackage{amssymb}
\usepackage{graphicx}
\usepackage{enumitem}
\usepackage{amsfonts}
\usepackage{amssymb}
\usepackage{xparse}
\usepackage{ifthen}
\usepackage{geometry}
\noprintanswers

\newcommand {\DS} [1] {${\displaystyle #1}$}
\newcommand{\answer}[1]{{\bf Answer:} \; #1}
\newcommand{\vv}{\vspace{.2cm}}
\newcommand{\vvv}{\vspace{6cm}}

\newcommand{\ul}{$\underline{\phantom{xxx}}$}
\newcommand{\ull}{\underline{\phantom{xxx}}}
\newcommand{\xh}{\hat{\bf x}}
\newcommand{\yh}{\hat{\bf y}}
\newcommand{\zh}{\hat{\bf z}}
\newcommand{\R}{\mathbb{R}}
\newcommand{\C}{\mathbb{C}}
\newcommand{\Z}{\mathbb{Z}}
\newcommand{\N}{\mathbb{N}}
\newcommand{\proj}{\mathrm{proj}}
\newcommand{\mat}[1]{\begin{bmatrix}#1\end{bmatrix}}
\newcommand{\floor}[1]{\lfloor #1 \rfloor}

\pagestyle{empty}


%%%%%%%%%%%%%%%%%%%%%%%%%%%%%%%%%%%%%%%%%
%  Edit course information here
%%%%%%%%%%%%%%%%%%%%%%%%%%%%%%%%%%%%%%%%%

\newcommand{\mthCourse}{MATH 110}
\newcommand{\mthTerm}{Fall 2013}
\newcommand{\mthTutorialNumber}{7}
\newcommand{\mthDate}{October 23, 2013}


%%%%%%%%%%%%%%%%%%%%%%%%%%%%%%%%%%%%%%%%%
\topmargin -1in
\textheight 10in

\begin{document}


%%%%%%%%%%%%%%%%%%%%%%%%%%%%%%%%%%%%%%%%%%%%%%%
% Main Questions
%%%%%%%%%%%%%%%%%%%%%%%%%%%%%%%%%%%%%%%%%%%%%%%
{\large
	\begin{center}
		{\bf \mthCourse, \mthTerm}\\ 
		{\bf Tutorial \#\mthTutorialNumber}\\
		{\bf \mthDate}
	\end{center}
}

\section*{Today's main problems}

\begin{enumerate}
	\item 
	\[
		S_1 = \left\{
			\mat{x_1\\x_2\\x_3}:x_1+3x_2-x_3=0
		\right\}
		\qquad
		S_2 = \left\{
			\mat{x_1\\x_2}:x_1x_2=0
		\right\}
	\]
	\[
		S_3 = \left\{
			\mat{x_1\\x_2\\x_3}:x_1+3x_2-x_3=0\text{ and }-x_1+x_2+2x_3=0
		\right\}
	\]\[
		S_4 = \text{A line in $\R^7$ through $\vec 0$ with direction $\vec d$}
	\]
	\begin{enumerate}
		\item Is $S_1$ a subspace of $\R^3$?
		\item Is $S_2$ a subspace of $\R^2$?
		\item Is $S_3$ a subspace of $\R^3$?
		\item Is $S_4$ a subspace of $\R^7$?
	\end{enumerate}
	
	\item Consider
		\[
			A=\mat{1&1&-1\\1&3&0\\3&-1&-5}\qquad \vec b=\mat{1\\2\\1}
			\qquad \vec v=\mat{7\\-1\\2}\qquad \vec w=\mat{1&-3&-3}
		\]
	\begin{enumerate}
		\item Determine whether or not $\vec b$ is in the column space of $A$.
		\item Determine whether or not $\vec w$ is in the row space of $A$.
		\item Determine whether or not $\vec v$ is in the null space of $A$.
		\item Find the dimension of the row space, column space, and null space.
	\end{enumerate}

\end{enumerate}
\subsection*{Further Questions}
\begin{enumerate}[resume]
	\item Give a basis for the row space, column space, and null space of $A$ where
		\[
			A=\mat{1&1&0&1\\0&1&-1&1\\0&1&-1&-1}
		\]
	\item Let $V=\{\vec v_1, \vec v_2, \vec v_3, \vec v_4, \vec v_5\}$
		where
		\[
			\vec v_1=\mat{1\\-1\\5\\2}\qquad
			\vec v_2=\mat{-2\\3\\1\\0}\qquad
			\vec v_3=\mat{4\\-5\\9\\4}\qquad
			\vec v_4=\mat{0\\4\\2\\-3}\qquad
			\vec v_5=\mat{-7\\18\\2\\-8}
		\]
		Find a subset of $V$ that is a
		basis for $\mathrm{span}( V)$ and express the remaining vectors as linear
		combinations of this basis.
\end{enumerate}




%%%%%%%%%%%%%%%%%%%%%%%%%%%%%%%%%%%%%%%%%%%%%%%
% Challenge questions
%%%%%%%%%%%%%%%%%%%%%%%%%%%%%%%%%%%%%%%%%%%%%%%
\newpage
{
	\begin{center}
		{\bf \mthCourse, \mthTerm}\\ 
		{\bf Tutorial \#\mthTutorialNumber}\\
		{\bf \mthDate}
	\end{center}
}

\section*{Challenge questions}

\begin{enumerate}[resume]
	\item Show that the intersection of any two subspaces is also a subspace.
	\item Completely classify all
		conditions under which the union of two subspaces a subspace.
	\item If $X$ is a subspace, $\proj_X\vec v$ is defined to be the vector
		in $X$ whose distance from $\vec v$ is smallest.
		Show that if $X=\mathrm{span}\{\vec d\}$, then $\proj_X \vec v$ agrees
		with $\proj_{\vec d}\vec v$.
	\item If $X=\mathrm{span}\{\vec u_1,\vec u_2\}$ is a plane, can you come up with an 
		algorithm for computing $\proj_X\vec v$?  (Note that $\vec u_1,
		\vec u_2$ may not be in $\R^3$.)

\end{enumerate}



%%%%%%%%%%%%%%%%%%%%%%%%%%%%%%%%%%%%%%%%%%%%%%%
% TA instructions
%%%%%%%%%%%%%%%%%%%%%%%%%%%%%%%%%%%%%%%%%%%%%%%
\newpage
{\small
	\begin{center}
		{\bf \mthCourse, \mthTerm}\\ 
		{\bf Tutorial \#\mthTutorialNumber. Instructions for TAs}
	\end{center}
}

\subsection*{Objectives}


\subsection*{Hidden objectives}


\subsection*{Suggestions}
	They may find set notation confusing, and some may think that
	$\{\vec v\}$ and $\mathrm{span} \{\vec v\}$ are two ways of saying the same thing.

	I would suggest briefly talking about how to interpret set notation
	with $S_1$ as an example.  ``A plane is a set of infinitely many vectors
	pointing from the origin to the surface if the plane,'' etc.

	It's a good idea to get the axioms of a subspace (not of an abstract
	vector space, just a subspace) written on the board at the beginning of
	class and remind them that if they want to show that something
	isn't a subspace, they only need to find a single violation
	of one of the axioms.  If something is a subspace, they need to show it
	works for all vectors, which will involve picking arbitrary vectors
	that satisfy the conditions specified in the set.  This is going to be
	very, very hard for them.

	This tutorial has quite a few problems on it and can be quite time
	consuming.  Remind them that they are not expected to get through both
	the main problems and the further problems in the tutorial time. 
	(In fact, it wouldn't be unreasonable if it took the full time just
	for question 1).

\subsection*{Wrapup}
	Choose a question that most of the class has started but not yet finished,
	or a question that people particularly struggled with.

\subsection*{Solutions}
\begin{enumerate}
	\item
\end{enumerate}
	

\end{document}
