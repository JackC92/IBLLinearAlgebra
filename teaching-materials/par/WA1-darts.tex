\documentclass{article}
\usepackage{amsmath}
\usepackage[dvipsnames]{xcolor}
\pagenumbering{gobble}
\usepackage{enumitem}

\usepackage[paperheight=6in, paperwidth=8in, margin=.3in]{geometry}

\begin{document}
	\Large\bf \color{NavyBlue} \textit{Recall that}
	\[
		\overbrace{\vec{x} =
		\begin{bmatrix}
			1\\
			0
		\end{bmatrix}t +
		\begin{bmatrix}
			0\\
			2
		\end{bmatrix}}^{\ell_1}, \qquad \overbrace{\vec{x} =
		\begin{bmatrix}
			1 \\
			-2
		\end{bmatrix}t +
		\begin{bmatrix}
			1\\
			3
		\end{bmatrix}}^{\ell_2}.
	\]


	\medskip


	\begin{enumerate}
		\color{Violet} \vspace{.2cm} \hrule

		\item Jack, you've made a mistake. The $t$'s can be different.
			You need to set them different when you solve, and then
			you'll see Tammy is correct.

			\vspace{.2cm} \hrule

		\item
			\[
				\ell_{1}\cap \ell_{2}\neq \{\}
			\]
			\[
				\iff
			\]
			\[
				\begin{bmatrix}
					1\\
					0
				\end{bmatrix}t +
				\begin{bmatrix}
					0\\
					2
				\end{bmatrix}=
				\begin{bmatrix}
					1 \\
					-2
				\end{bmatrix}s +
				\begin{bmatrix}
					1\\
					3
				\end{bmatrix}\qquad\implies\qquad t=\frac{1}{2}\quad
				s=\frac{3}{2}
			\]
			 So the lines intersect as Tammy said. \vspace{.2cm}
			\hrule

		\item The `$t$' that shows up in the vector form of a line is a dummy
			variable. To determine if two lines written in vector form
			intersect, you need to replace $t$ with a real variable.
			For example, you could solve
			\[
				\begin{bmatrix}
					1\\
					0
				\end{bmatrix}t +
				\begin{bmatrix}
					0\\
					2
				\end{bmatrix}=
				\begin{bmatrix}
					1 \\
					-2
				\end{bmatrix}s +
				\begin{bmatrix}
					1\\
					3
				\end{bmatrix}.
			\]
			 \vspace{.2cm} \hrule
	\end{enumerate}
\end{document}
