\documentclass[red]{tutorial}
\usepackage[no-math]{fontspec}
\usepackage{xpatch}
	\renewcommand{\ttdefault}{ul9}
	\xpatchcmd{\ttfamily}{\selectfont}{\fontencoding{T1}\selectfont}{}{}
	\DeclareTextCommand{\nobreakspace}{T1}{\leavevmode\nobreak\ }
\usepackage{polyglossia} % English please
	\setdefaultlanguage[variant=us]{english}
%\usepackage[charter,cal=cmcal]{mathdesign} %different font
%\usepackage{avant}
\usepackage{microtype} % Less badboxes


\usepackage[charter,cal=cmcal]{mathdesign} %different font
%\usepackage{euler}
 
\usepackage{blindtext}
\usepackage{calc, ifthen, xparse, xspace}
\usepackage{makeidx}
\usepackage[hidelinks, urlcolor=blue]{hyperref}   % Internal hyperlinks
\usepackage{mathtools} % replaces amsmath
\usepackage{bbm} %lower case blackboard font
\usepackage{amsthm, bm}
\usepackage{thmtools} % be able to repeat a theorem
\usepackage{thm-restate}
\usepackage{graphicx}
\usepackage{xcolor}
\usepackage{multicol}
\usepackage{fnpct} % fancy footnote spacing

 
\newcommand{\xh}{{{\mathbf e}_1}}
\newcommand{\yh}{{{\mathbf e}_2}}
\newcommand{\zh}{{{\mathbf e}_3}}
\newcommand{\R}{\mathbb{R}}
\newcommand{\Z}{\mathbb{Z}}
\newcommand{\N}{\mathbb{N}}
\newcommand{\proj}{\mathrm{proj}}
\newcommand{\Proj}{\mathrm{proj}}
\newcommand{\Perp}{\mathrm{perp}}
\renewcommand{\span}{\mathrm{span}\,}
\newcommand{\Span}{\mathrm{span}\,}
\newcommand{\Img}{\mathrm{img}\,}
\newcommand{\Null}{\mathrm{null}\,}
\newcommand{\Range}{\mathrm{range}\,}
\newcommand{\rref}{\mathrm{rref}}
\newcommand{\rank}{\mathrm{rank}}
\newcommand{\Rank}{\mathrm{rank}}
\newcommand{\nnul}{\mathrm{nullity}}
\newcommand{\mat}[1]{\begin{bmatrix}#1\end{bmatrix}}
\newcommand{\chr}{\mathrm{char}}
\renewcommand{\d}{\mathrm{d}}


\theoremstyle{definition}
\newtheorem{example}{Example}[section]
\newtheorem{defn}{Definition}[section]

\theoremstyle{theorem}
\newtheorem{thm}{Theorem}[section]

\pgfkeys{/tutorial,
	name={Tutorial 1},
	author={Jason Siefken},
	course={MAT 223},
	date={September 16},
	term={Fall 2019},
	title={Math \& the World}
	}

\begin{document}
	\begin{tutorial}
				\begin{objectives}
			In this tutorial you will work on rephrasing problems
			with mathematical language---this is an essential skill if you ever
			plan on applying mathematical techniques to the world!

			These problems relate to the following course learning
			objective: \textit{work independently to understand concepts and procedures that have not
			been previously explained to you}.
		\end{objectives}

	%	\bigskip


		\subsection*{Problems}


		\begin{enumerate}
			\item Use vectors, sets, set operations, and set-builder notation
				to describe the following as subsets of $\R^2$.
				\begin{enumerate}
					\item The $x$-axis.
					\item The corners of a square $S$, which is centered at the origin and whose
						sides have length 3 and are aligned with the axes.
					\item The diagonal of $S$ (from before) starting from the lower-left to the upper-right.
					\item Both diagonals of $S$.
					\item The line segment from $(2,3)$ to $(4,1)$, including the endpoints.
					\item The line segment from $(2,3)$ to $(4,1)$, not including the endpoints.
				\end{enumerate}

			\item Let's make a smiley face!\footnote{ This question is not a joke, and a version of
				it may show up on your midterm.}
				\begin{enumerate}
					\item Describe the lower half of a circle of radius 1 centered at the origin.
						Call this set $M$ (for mouth!).
					\item Pick a location for the eyes and describe them as small, filled in
						circles. Call the left eye $L$ and the right eye $R$.
					\item Describe the whole face using $M$, $L$, and $R$. Call the face $F$.
					\item When we draw a set, we usual draw black
						for points in the set and leave points not in the set white.
						Let's draw a reverse-face. Come up with a set $F_R$
						for a face where the ``skin'' of the face is included in $F_R$, but
						the eyes and mouth are not.
				\end{enumerate}
			\item \emph{Interpolation} is the process of filling in points that might not exist already.
				It's commonly used when zooming-in or rotating a picture on your computer.  The picture $P$ consists
				of four colored pixels, \emph{\color{red}red} at $(0,0)$, \emph{\color{green!70!black}green} at $(1,0)$,
				and \emph{\color{blue}blue} at $(2,0)$ and $(3,0)$. To your brain, what is important about this picture
				is the \emph{relative spacing} between the colors, not their absolute positions. We will interpolate the color
				positions for a transformed $P$.
				\begin{enumerate}
					\item Give the coordinates of each color if $P$ were translated, going from $(1,4)$ to $(4,4)$.
					\item Give the coordinates of each color if $P$ were twice as big, going from $(0,0)$ to $(6,0)$.
					\item Give the coordinates of each color if $P$ were rotated and zoomed, going from $(-1,-1)$ to $(7,0)$.
				\end{enumerate}

		\end{enumerate}
	\end{tutorial}


	\begin{solutions}
		\Heading{Solutions}

		\begin{enumerate}
			\item \begin{enumerate}
					\item $\left\{\mat{x\\y}\in\R^2:y=0\right\}$.
					\item $\left\{\mat{-3/2\\3/2},\mat{3/2\\3/2},\mat{3/2\\-3/2},\mat{-3/2\\-3/2}
							\right\}$.
					\item $\left\{\mat{t\\t}:t\in[-3/2,3/2]\right\}$.
					\item $\left\{\mat{t\\t}:t\in[-3/2,3/2]\right\}\cup \left\{\mat{t\\-t}:t\in[-3/2,3/2]\right\}$.
					\item $\left\{\vec v: \vec v=\alpha \mat{2\\3}+(1-\alpha)\mat{4\\1}\text{ for some }\alpha\in[0,1]\right\}$.
					\item $\left\{\vec v: \vec v=\alpha \mat{2\\3}+(1-\alpha)\mat{4\\1}\text{ for some }\alpha\in(0,1)\right\}$.
			\end{enumerate}
			\item \begin{enumerate}
					\item $M=
						\left\{\mat{x\\y}: x^2+y^2 = 1\text{ and }y\leq 0\right\}$.
					\item Define $\left\|\mat{x\\y}\right\|=\sqrt{x^2+y^2}$ to
						be the length of a vector in $\mathbb R^2$.
						Let $\vec l=\mat{-1/2\\1}$ and $\vec r=\mat{1/2\\1}$. Then \[L=\{\vec v:\|\vec v-\vec l\|\leq 1/4\}
						\qquad\text{and}\qquad R=\{\vec v:\|\vec v-\vec r\|\leq 1/4\}.\]
					\item $F=M\cup L\cup R$.
					\item $F_R=\{\vec v:\|\vec v\| \leq 3/2\text{ and }\vec v\notin F\}$.
			\end{enumerate}
			\item \begin{enumerate}
				\item red at $(1,4)$, green at $(2,4)$, and blue at $(3,4)$ and $(4,4)$.
				\item red at $(0,0)$, green at $(2,0)$, and blue at $(4,0)$ and $(6,0)$.
				\item red at $(-1,-1)$, green at $(5/3,-2/3)$, and blue at $(13/3,-1/3)$ and $(7,0)$.
			\end{enumerate}
		\end{enumerate}
	
	\end{solutions}
	\begin{instructions}
				\Heading{Learning Objectives} Students need to be able to\ldots
		\begin{itemize}
			\item Turn geometric descriptions and pictures into equations/formulas/sets
				suitable for manipulation with mathematics.

			\item Be comfortable enough with set notation and operations to combine
				the operations in new ways.
		\end{itemize}


		\Heading{Context} Students in class have gone over sets, set operations,
		vectors, linear combinations, vector form of lines and planes, and have just started
		span. Some sections may also have covered \emph{restricted} linear combinations, for example
		convex combinations. Sections have \emph{not} covered norm notation (i.e., $\|\vec x\|$) or
		lengths of vectors in general. However, they all know the Pythagorean theorem from high school.


		\Heading{What to Do} This is the first tutorial of the term, and
		it is your chance to win the students over! This is a groupwork tutorial,
		but students may not be used to working in groups.

		\begin{itemize}
			\item Arranged for group work. Reorganize the desks and chairs
				(if possible) to facilitate groups of 3 or 4. Ask
				students to form groups of 3 or 4 with other students
				nearby. Don't allow larger groups.

			\item Begin the tutorial by introducing yourself (your name,
				your program of study, and your year). You might
				also want to give them some more personal information,
				such as where you are from or when you first started liking math.

			\item Introduce the structure and purpose of tutorials: students
				will be working to (1) better understand concepts
				from lecture, (2) practice tackling concepts that
				have not been explained in lecture, and (3) effectively
				communicate. They can expect to spend most of the
				tutorial working in small groups.

			\item Emphasize the importance of working with others when
				learning mathematics---they should be working with
				others in this tutorial \emph{and} outside of
				class.
		\end{itemize}

		This introduction should take no more than 5 minutes.

		Next, introduce the learning objectives for the day's tutorial. Explain
		that the goal of this tutorial. Their worksheet has the ``formal'' objectives
		stated and these instructions have the ``hidden'' objectives. Feel free
		to share with them the hidden objectives as well.

		Ask the students to pair up and
		start working on the problem list. Circulate around the room during
		this time and ask groups what they're thinking. They will be tempted
		to move quickly through the list without thoroughly checking their
		new answers---encourage them to think deeply.

		Problem 1 is a straightforward question, but students will struggle starting
		with part (d) and especially with (e) and (f). They may have forgotten about unions! Ask
		them to review the set operations that they know and be creative. When most people are on
		parts (e) and (f), go over parts (a)--(d). Then, let them continue working through number 2.
		If most of the class gets stuck at any point, draw the class's attention to the front
		of the room and work on the difficult part together.

		There are too many problems to finish in 50 minutes and \emph{you should not be going
		over the solution to every problem}. Solutions will be posted for the students. The goal
		of tutorial is for students to spend time \emph{doing} mathematics with an expert around
		to help them if they get stuck. Don't feel any time pressure, even if you only get through 1.5
		questions, that's okay!

		During the last 6 minutes of class, pick one problem (perhaps a few parts of one problem)
		that most groups have at least started, and do this problem as a wrapup. Seeing an expert do the
		problem is the student's reward for working so hard.

		Notes:
		\begin{itemize}
			\item Students won't have a good conceptualization of convex combinations which make 1(f) and 3(c).
				These problems can also be done by describing a line in vector form (i.e., $\vec x=t\vec d+\vec p$)
				and restricting the scalar to get points on the line segment.
			\item For 1(e), some students might write $\{\vec x:\vec x\text{ is a convex linear combination
				of }\vec p\text{ and }\vec q\}$. Other students might think that this description
				is ``mathy'' enough. This description is mathy enough, but we can also expand it
				by inserting the definition of \emph{convex linear combination} into the set.
			\item Problem 2 is more open-ended than they're used to. Some will get excited about this, and others will
				be turned off because it's not a ``plug and chug'' question. Emphasize to them that
				they will be hired for their creativity and problem-solving, and not their ability to
				answer precisely laid out questions, and this is what we're practicing!
			\item Students will be confused what part 2(d) means. If so, take some time at the front of the room
				to make a chalk drawing of a face and a reverse face.
				Remember, if you're writing on a chalkboard, black and white are already reversed!
		\end{itemize}
	\end{instructions}

\end{document}
