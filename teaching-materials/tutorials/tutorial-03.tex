\documentclass[red]{tutorial}
\usepackage[no-math]{fontspec}
\usepackage{xpatch}
	\renewcommand{\ttdefault}{ul9}
	\xpatchcmd{\ttfamily}{\selectfont}{\fontencoding{T1}\selectfont}{}{}
	\DeclareTextCommand{\nobreakspace}{T1}{\leavevmode\nobreak\ }
\usepackage{polyglossia} % English please
	\setdefaultlanguage[variant=us]{english}
%\usepackage[charter,cal=cmcal]{mathdesign} %different font
%\usepackage{avant}
\usepackage{microtype} % Less badboxes


\usepackage[charter,cal=cmcal]{mathdesign} %different font
%\usepackage{euler}
 
\usepackage{blindtext}
\usepackage{calc, ifthen, xparse, xspace}
\usepackage{makeidx}
\usepackage[hidelinks, urlcolor=blue]{hyperref}   % Internal hyperlinks
\usepackage{mathtools} % replaces amsmath
\usepackage{bbm} %lower case blackboard font
\usepackage{amsthm, bm}
\usepackage{thmtools} % be able to repeat a theorem
\usepackage{thm-restate}
\usepackage{graphicx}
\usepackage{xcolor}
\usepackage{multicol}
\usepackage{fnpct} % fancy footnote spacing
\usepackage{autoaligne}

 
\newcommand{\xh}{{{\mathbf e}_1}}
\newcommand{\yh}{{{\mathbf e}_2}}
\newcommand{\zh}{{{\mathbf e}_3}}
\newcommand{\R}{\mathbb{R}}
\newcommand{\Z}{\mathbb{Z}}
\newcommand{\N}{\mathbb{N}}
\newcommand{\proj}{\mathrm{proj}}
\newcommand{\Proj}{\mathrm{proj}}
\newcommand{\Comp}{\mathrm{comp}}
\newcommand{\Perp}{\mathrm{perp}}
\renewcommand{\span}{\mathrm{span}\,}
\newcommand{\Span}{\mathrm{span}\,}
\newcommand{\Img}{\mathrm{img}\,}
\newcommand{\Null}{\mathrm{null}\,}
\newcommand{\Range}{\mathrm{range}\,}
\newcommand{\rref}{\mathrm{rref}}
\newcommand{\rank}{\mathrm{rank}}
\newcommand{\Rank}{\mathrm{rank}}
\newcommand{\nnul}{\mathrm{nullity}}
\newcommand{\mat}[1]{\begin{bmatrix}#1\end{bmatrix}}
\newcommand{\chr}{\mathrm{char}}
\renewcommand{\d}{\mathrm{d}}


\theoremstyle{definition}
\newtheorem{example}{Example}[section]
\newtheorem{defn}{Definition}[section]

\theoremstyle{theorem}
\newtheorem{thm}{Theorem}[section]

\pgfkeys{/tutorial,
	name={Tutorial 3},
	author={Jason Siefken},
	course={MAT 223},
	date={September 30},
	term={Fall 2019},
	title={Orthogonality}
	}

\begin{document}
	\begin{tutorial}

	\Heading{Definitions}
		Recall the vectors $\vec a$ and $\vec b$ are \emph{orthogonal} if $\vec a\cdot \vec b=0$.
		We say the \emph{sets $A$ and $B$ are orthogonal} if every vector in $A$ is orthogonal to every
		vector in $B$.

		
\Heading{Problems}
Let $\vec v_1=\mat{1\\1\\1\\1}$,
		$\vec v_2=\mat{-1\\1\\1\\1}$,
		$\vec v_3=\mat{-1\\-1\\1\\1}$,
		$\vec v_4=\mat{1\\0\\0\\1}$,
		$\vec v_5=\mat{3\\-1\\-1\\-1}$, and
		$\vec v_6=\mat{1\\1\\2\\0}$.

\begin{enumerate}
	\item
	\begin{enumerate}
		\item Identify all pairs of orthogonal vectors among $\vec v_1$, \ldots, $\vec v_6$.
		\item Let $A=\{\vec v_1, \vec v_2\}$ and $B=\{\vec v_3,\vec v_4\}$. Are $A$ and $B$ orthogonal sets?
			Why or why not?
		\item Let $P=\{\vec v_1, \vec v_6\}$ and $Q=\{\vec v_3,\vec v_5\}$. Are $P$ and $Q$ orthogonal sets?
			Why or why not?
		\item Can you split the vectors $\vec v_1$, \ldots, $\vec v_6$ into two non-empty sets that are orthogonal to
			each other? Explain.
	\end{enumerate}
	\item
	\begin{enumerate}
		\item Using guess-and-check, find two vectors that are orthogonal to both $\vec v_1$ and $\vec v_2$.
		\item Set up and solve a system of equations to find all vectors orthogonal to $\vec v_1$ and $\vec v_2$.
	\end{enumerate}
	\item The dot product is \emph{commutative} and \emph{distributive}. That is $\vec v\cdot (\alpha\vec a+\vec b)=
		(\alpha\vec a+\vec b)\cdot \vec v=\alpha(\vec a\cdot \vec v)+\vec b\cdot\vec v$. Use this to show that if
		the set $X=\{\vec x\}$ is orthogonal to the set $Y=\{\vec y_1,\vec y_2,\vec y_3,\vec y_4\}$,
		then $X$ is also orthogonal to $\Span Y$.
	
	\item We say that $\vec a$ and $\vec b$ are \emph{close} if $\|\vec a-\vec b\|$ is small. We will see if we can
		extend this concept to lines.

		Let the lines $\ell_1$, $\ell_2$, $\ell_3$, and $\ell_4$ be given by the equations $y=x$, $y=1.001x$,
				$y=2000x$, $y=3000x$.
		\begin{enumerate}
			\item Out of $\ell_1$, \ldots, $\ell_4$, which lines would you call ``close''? Can you come up
				with a mathematical definition to justify your conclusion?
			\item For $\ell_1$, \ldots, $\ell_4$, find unit normal vectors $\vec n_1$, \ldots, $\vec n_4$.
				For consistency, ensure each unit normal vector points towards the upper left (i.e., has negative first coordinate
				and positive second coordinate).
			\item Compute the distances between $\vec n_1$, \ldots, $\vec n_4$. Do these distances coincide with your
				intuition about closeness? Why might comparing normal vectors be preferable to comparing direction vectors?
		\end{enumerate}
	
\end{enumerate} 
	\end{tutorial}


	\begin{solutions}

		\Heading{Solutions}

\begin{enumerate}
			\item
			\begin{enumerate}
				\item
				$\vec v_1\perp \vec v_3$,\quad  $\vec v_1\perp \vec v_5$, \quad
				$\vec v_2\perp\vec v_4$,\quad  $\vec v_3\perp \vec v_4$,\quad
				$\vec v_3\perp \vec v_6$,\quad  $\vec v_5\perp \vec v_6$.

				\item No. $\vec v_2\cdot \vec v_3\neq 0$.
				\item Yes. $\vec v_1\cdot \vec v_3=\vec v_1\cdot \vec v_5=0$ and
					$\vec v_6\cdot \vec v_3=\vec v_6\cdot \vec v_5=0$.
				\item No. Whatever set contains $\vec v_1$ must also contain $\vec v_2$ and
					$\vec v_4$ and $\vec v_6$.
					However $\Span\{\vec v_1,\vec v_2,\vec v_4,
					\vec v_6\}=\R^4$, so the only non-empty set orthogonal
					to $\{\vec v_1,\vec v_2,\vec v_4,
					\vec v_6\}$ is $\{\vec 0\}$, which isn't a possibility in this case.
			\end{enumerate}
			\item \begin{enumerate}
					\item $\mat{0\\1\\-1\\0}$ and $\mat{0\\0\\1\\-1}$.
					\item \[
							\begin{cases}x_1+x_2+x_3+x_4&=0\\ -{x_1}+x_2+x_3+x_4&=0\end{cases}
						\]
					has complete solution
					\[
						\vec x = t\mat{0\\-1\\1\\0}+s\mat{0\\-1\\0\\1}.
					\]
			\end{enumerate}

			\item By definition, if $\vec u\in\Span(Y)$, then $\vec u=\alpha_1\vec y_1+\cdots+\alpha_4\vec y_4$.
				By distributivity of the dot product, we have
				\[
					\vec x\cdot \vec u = \vec x\cdot (\alpha_1\vec y_1+\cdots+\alpha_4\vec y_4)
					=\alpha_1(\vec x\cdot \vec y_1)+\cdots +\alpha_4(\vec x\cdot \vec y_4)=0,
				\]
				so $\vec x$ is orthogonal to every vector in $\Span(Y)$.

			\item \begin{enumerate}
				\item $\ell_1$ and $\ell_2$ are close and $\ell_3$ and $\ell_4$ are close.
				\item $\vec n_1\approx\mat{-0.70711\\0.70711}$,
					$\vec n_2\approx\mat{-0.70675\\0.70746}$,
					$\vec n_3\approx\mat{-0.0005\\1}$, and $\vec n_4\approx\mat{-0.0003\\1}$.
				\item
					$\|\vec n_1-\vec n_2\|\approx 0.0005$,
					$\|\vec n_1-\vec n_3\|\approx \|\vec n_1-\vec n_4\|
					\approx \|\vec n_2-\vec n_3\|\approx \|\vec n_2-\vec n_4\|\approx 0.765$,
					and
					$\|\vec n_3-\vec n_4\|\approx 0.0017$.

					These distances coincide with my intuitive idea of closeness. Using normal
					vectors might be preferable to using direction vectors, because it generalizes to
					planes. A plane has a unique
					normal direction but infinitely many direction vectors that might be hard to compare.
			\end{enumerate}
		\end{enumerate} 
	\end{solutions}
	\begin{instructions}
\Heading{Learning Objectives}
	Students need to be able to\ldots
	\begin{itemize}
		\item Apply the definition of orthogonality
		\item Interpret and apply a new definition, that of \emph{orthogonal sets}
		\item Set up a system of equations to produce orthogonal vectors
	\end{itemize}

\Heading{Context}
	Students will have covered dot products, orthogonality, normal form of lines and planes, and
		projections in the previous week's lectures\footnote{ Note that if you use the word \emph{projection},
		it has a specific definition in this class: $\Proj_{X}\vec a$ is the closest point in $X$
		to $\vec a$. In particular, $\Proj_{\vec b}\vec a$ is not a defined notation, since $\vec b$ is
		not a set. Instead we use the notation $\Comp_{\vec b}\vec a$.}.

\Heading{What to Do}
	Introduce the learning objectives for the day's tutorial. Explain that we will be extending
		the notion of orthogonality they know from lecture to sets. Further, explain that one of the skills
		we're developing in this class is to be able to read, understand, and apply a new definition, and that's
		what the first tutorial question is all about.

	
	After most groups have finished \#1, go over it as a class. There is no need to write a formal
		proof for part (d), though everyone should be able to give a convincing argument, even
		if it's not a full ``proof''. Students might be puzzled on how to explain (b),
		resorting to ``because they are\ldots''. Try to push them to refer to the definitions
		from which they can give a full argument (even if the argument seems silly).

	Continue as usual, walking around the room and asking
		questions while letting students work on the next problem and gathering them together
		for discussion when most groups have finished.

		7 minutes before class ends, pick a suitable problem to do as a wrap-up.



	
\Heading{Notes}
	\begin{itemize}
		\item Students will have a hard time explaining themselves for 1(d). Make sure you
			have some prompts in your back pocket.
		\item Students might get stuck on \#2. Remind them to think about the definitions and
			to start by writing them down---it's amazing how impossible a problem seems before
			you write down the definition\ldots.
		\item For \#3 we want an answer written as a proof. This will be hard for most students.
		\item \#4 is there as a challenge question. You probably won't be talking about it.

	\end{itemize}
	\end{instructions}

\end{document}
