\documentclass[red]{tutorial}
\usepackage[no-math]{fontspec}
\usepackage{xpatch}
	\renewcommand{\ttdefault}{ul9}
	\xpatchcmd{\ttfamily}{\selectfont}{\fontencoding{T1}\selectfont}{}{}
	\DeclareTextCommand{\nobreakspace}{T1}{\leavevmode\nobreak\ }
\usepackage{polyglossia} % English please
	\setdefaultlanguage[variant=us]{english}
%\usepackage[charter,cal=cmcal]{mathdesign} %different font
%\usepackage{avant}
\usepackage{microtype} % Less badboxes


\usepackage[charter,cal=cmcal]{mathdesign} %different font
%\usepackage{euler}
 
\usepackage{blindtext}
\usepackage{calc, ifthen, xparse, xspace}
\usepackage{makeidx}
\usepackage[hidelinks, urlcolor=blue]{hyperref}   % Internal hyperlinks
\usepackage{mathtools} % replaces amsmath
\usepackage{bbm} %lower case blackboard font
\usepackage{amsthm, bm}
\usepackage{thmtools} % be able to repeat a theorem
\usepackage{thm-restate}
\usepackage{graphicx}
\usepackage{multicol}
\usepackage{fnpct} % fancy footnote spacing
%\usepackage[dvipsnames]{color}
\usepackage{tikz}
\usepackage{pgfplots}
\pgfplotsset{compat=newest}

 
\newcommand{\xh}{{{\mathbf e}_1}}
\newcommand{\yh}{{{\mathbf e}_2}}
\newcommand{\zh}{{{\mathbf e}_3}}
\newcommand{\R}{\mathbb{R}}
\newcommand{\Z}{\mathbb{Z}}
\newcommand{\N}{\mathbb{N}}
\newcommand{\proj}{\mathrm{proj}}
\newcommand{\Proj}{\mathrm{proj}}
\newcommand{\Comp}{\mathrm{comp}}
\newcommand{\Perp}{\mathrm{perp}}
\renewcommand{\span}{\mathrm{span}\,}
\newcommand{\Span}{\mathrm{span}\,}
\newcommand{\Img}{\mathrm{img}\,}
\newcommand{\Null}{\mathrm{null}\,}
\newcommand{\Range}{\mathrm{range}\,}
\newcommand{\rref}{\mathrm{rref}}
\newcommand{\rank}{\mathrm{rank}}
\newcommand{\Rank}{\mathrm{rank}}
\newcommand{\nnul}{\mathrm{nullity}}
\newcommand{\mat}[1]{\begin{bmatrix}#1\end{bmatrix}}
\newcommand{\chr}{\mathrm{char}}
\renewcommand{\d}{\mathrm{d}}


\theoremstyle{definition}
\newtheorem{example}{Example}[section]
\newtheorem{defn}{Definition}[section]

\theoremstyle{theorem}
\newtheorem{thm}{Theorem}[section]

\pgfkeys{/tutorial,
	name={Tutorial 4},
	author={Jason Siefken},
	course={MAT 223},
	date={October 7},
	term={Fall 2019},
	title={Projections}
	}

\begin{document}
	\begin{tutorial}

		\begin{objectives}
			Projections are used in mathematics, physics, computer science, and
			statistics.
	In this tutorial you will obtain a deeper understanding of projections.

	These problems relate to the following course learning objectives:
			\textit{Translate between algebraic and geometric viewpoints to solve problems},
			and
			\textit{work independently to understand concepts and procedures that have not been previously
			explained to you}.
		\end{objectives}

		\vspace{-.5em}
		\subsection*{Problems}
		\vspace{-.5em}

		Let $R$ be the square with side-length 1 and lower-left corner at $(0,0)$;
		let $C$ be the corners of $R$; and, let $S$ be the circle of radius $\sqrt{2}$ centered
		at $(0,0)$.

\begin{enumerate}
	\item Write down a mathematically precise definition of the projection of the vector $\vec v$ onto the set $X$.
	\item Let $\vec v_1=\mat{1/3\\0}$, $\vec v_2=\mat{2\\2}$, $\vec v_3=\mat{2\\1}$ and $\vec v_4=\mat{1/3\\2}$.
	\begin{enumerate}
		\item Draw $R$, $C$, and $S$ on separate grids.
		\item Using your drawings, estimate the projections of $\vec v_1$, \ldots, $\vec v_4$ onto $R$, $C$, and $S$.
		\item Compute exactly the projections of $\vec v_1$, \ldots, $\vec v_4$ onto $R$, $C$, and $S$.
	\end{enumerate}


	\item Let $\ell$ be the line given in vector form by $\vec x=t\vec d$ where $\vec d=\mat{1\\2}$, and let $\vec r=\mat{2\\2}$.
		\begin{enumerate}
			\item Find a formula for the distance between $t\vec d$ and $\vec r$ in terms of $t$.
			\item Compute $\Proj_\ell \vec r$.
			\item What is the angle between $\vec d$ and $\vec r-\Proj_\ell \vec r$?
			\item Explain the link between $\Proj_\ell \vec r$ and $\Comp_{\vec d}\vec r$.
		\end{enumerate}

	\item In $\R^3$, the projection of a vector onto the $xy$-plane can be thought of as the shadow
		of the vector at high noon. But where would this shadow be when it's not high noon?\

		To simplify things, let's work in $\R^2$. Assume the $x$-axis is the ``ground'' and the $y$-axis is
		pointing straight up. If the sun is located far off in the distance in the direction of $\vec e_2$,
		the shadow of a vector will be its projection onto the $x$-axis.
		\begin{enumerate}
			\item Find a formula for where the shadow of the vector $\vec x=\mat{x\\y}$ will be at high noon.
			\item Suppose the ground is modeled with the equation $y=\tfrac{1}{2}x$ and that at 4:00pm, the sun shines
				perpendicularly to the ground. That is, the sun is far away in the direction $\mat{-1\\2}$.
				Find a formula for the shadow of the vector $\vec x=\mat{x\\y}$ on the slanted ground at 4:00pm.
			\item The vector $\vec y=\mat{a\\b}$ is above flat ground (i.e., the ground is modeled by the $x$-axis).
				Find a formula for the position of the shadow of $\vec y$ at 4:00pm.
		\end{enumerate}

\end{enumerate}
	\end{tutorial}


	\begin{solutions}
		\begin{enumerate}
			\item The \emph{projection} of $\vec v$ onto $X$ is the closest point in $X$ to $\vec v$.
			\item \begin{enumerate}
					\item $\phantom{x}$

\begin{center}
	\begin{tikzpicture}[scale=.8, >=latex]
    \begin{axis}[scale=1,
		    axis equal image,
		    %axis line style={draw=none},
		    axis lines=middle,
		    tick style={draw=none},
		    yticklabels={,,},
		    xticklabels={,,},
		 xmin=-.5,
		 xmax=1.5,
		 ymin=-.5,
		 ymax=1.5,
		 major grid style={dotted, gray},
                 xtick={-10,-9,...,10},
                 ytick={-10,-9,...,10},
                 grid=both,
		 anchor=origin]

	  \draw[green!50!black, very thick] 
	    (0,0)-- (0,1)--
	    (1,1) node[above right, black] {$R$}-- (1,0) -- cycle
	    ;
    \end{axis}
\end{tikzpicture}
	\begin{tikzpicture}[scale=.8, >=latex]
    \begin{axis}[scale=1,
		    axis equal image,
		    %axis line style={draw=none},
		    axis lines=middle,
		    tick style={draw=none},
		    yticklabels={,,},
		    xticklabels={,,},
		 xmin=-.5,
		 xmax=1.5,
		 ymin=-.5,
		 ymax=1.5,
		 major grid style={dotted, gray},
                 xtick={-10,-9,...,10},
                 ytick={-10,-9,...,10},
                 grid=both,
		 anchor=origin]

	  \fill[fill=blue] 
	    (0,0) circle[radius=2pt] (0,1) circle[radius=2pt]
	    (1,1) circle[radius=2pt] node[above right] {$C$} (1,0) circle[radius=2pt]
	    ;
    \end{axis}
\end{tikzpicture}
	\begin{tikzpicture}[scale=.8, >=latex]
    \begin{axis}[scale=1,
		    axis equal image,
		    %axis line style={draw=none},
		    axis lines=middle,
		    tick style={draw=none},
		    yticklabels={,,},
		    xticklabels={,,},
		 xmin=-1.5,
		 xmax=1.5,
		 ymin=-1.5,
		 ymax=1.5,
		 major grid style={dotted, gray},
                 xtick={-10,-9,...,10},
                 ytick={-10,-9,...,10},
                 grid=both,
		 anchor=origin]

	  \draw[red, very thick] 
	    (0,0) circle[radius=1.41]
	    ;
	   \path (1,1) node[above right] {$S$};
    \end{axis}
\end{tikzpicture}
\end{center}
					\item $\phantom{x}$

\begin{center}
	\begin{tikzpicture}[scale=.8, >=latex]
    \begin{axis}[scale=1,
		    axis equal image,
		    %axis line style={draw=none},
		    axis lines=middle,
		    tick style={draw=none},
		    yticklabels={,,},
		    xticklabels={,,},
		 xmin=-.5,
		 xmax=2.5,
		 ymin=-.5,
		 ymax=2.5,
		 major grid style={dotted, gray},
                 xtick={-10,-9,...,10},
                 ytick={-10,-9,...,10},
                 grid=both,
		 anchor=origin]

	  \draw[green!50!black, very thick] 
	    (0,0)-- (0,1)--
	    (1,1) node[above right, black] {$R$}-- (1,0) -- cycle
	    ;
	    \draw[orange, ->, very thick] (0,0) -- (.333,0) node[below] {$\vec v_1$};
	    \draw[orange, ->, very thick] (0,0) -- (2,2) node[below right] {$\vec v_2$};
	    \draw[orange, ->, very thick] (0,0) -- (2,1) node[right] {$\vec v_3$};
	    \draw[orange, ->, very thick] (0,0) -- (.333,2) node[above right] {$\vec v_4$};
    \end{axis}
\end{tikzpicture}
	\begin{tikzpicture}[scale=.8, >=latex]
    \begin{axis}[scale=1,
		    axis equal image,
		    %axis line style={draw=none},
		    axis lines=middle,
		    tick style={draw=none},
		    yticklabels={,,},
		    xticklabels={,,},
		 xmin=-.5,
		 xmax=2.5,
		 ymin=-.5,
		 ymax=2.5,
		 major grid style={dotted, gray},
                 xtick={-10,-9,...,10},
                 ytick={-10,-9,...,10},
                 grid=both,
		 anchor=origin]

	  \fill[fill=blue] 
	    (0,0) circle[radius=2pt] (0,1) circle[radius=2pt]
	    (1,1) circle[radius=2pt] node[above right] {$C$} (1,0) circle[radius=2pt]
	    ;
	    \draw[orange, ->, very thick] (0,0) -- (.333,0) node[below] {$\vec v_1$};
	    \draw[orange, ->, very thick] (0,0) -- (2,2) node[below right] {$\vec v_2$};
	    \draw[orange, ->, very thick] (0,0) -- (2,1) node[right] {$\vec v_3$};
	    \draw[orange, ->, very thick] (0,0) -- (.333,2) node[above right] {$\vec v_4$};
    \end{axis}
\end{tikzpicture}
	\begin{tikzpicture}[scale=.8, >=latex]
    \begin{axis}[scale=1,
		    axis equal image,
		    %axis line style={draw=none},
		    axis lines=middle,
		    tick style={draw=none},
		    yticklabels={,,},
		    xticklabels={,,},
		 xmin=-1.5,
		 xmax=2.5,
		 ymin=-1.5,
		 ymax=2.5,
		 major grid style={dotted, gray},
                 xtick={-10,-9,...,10},
                 ytick={-10,-9,...,10},
                 grid=both,
		 anchor=origin]

	  \draw[red, very thick] 
	    (0,0) circle[radius=1.41]
	    ;
	   \path (1,1) node[above right] {$S$};
	    \draw[orange, ->, very thick] (0,0) -- (.333,0) node[below] {$\vec v_1$};
	    \draw[orange, ->, very thick] (0,0) -- (2,2) node[below right] {$\vec v_2$};
	    \draw[orange, ->, very thick] (0,0) -- (2,1) node[right] {$\vec v_3$};
	    \draw[orange, ->, very thick] (0,0) -- (.333,2) node[above right] {$\vec v_4$};
    \end{axis}
\end{tikzpicture}
\end{center}

	\item 
		\[
			\Proj_R\vec v_1=\vec v_1\qquad
			\Proj_R\vec v_2=\mat{1\\1}\qquad
			\Proj_R\vec v_3=\mat{1\\1}\qquad
			\Proj_R\vec v_4=\mat{1/3\\1}
		\]
		\[
			\Proj_C\vec v_1=\mat{0\\0}\qquad
			\Proj_C\vec v_2=\mat{1\\1}\qquad
			\Proj_C\vec v_3=\mat{1\\1}\qquad
			\Proj_C\vec v_4=\mat{0\\1}
		\]
		\[
			\Proj_S\vec v_1=\mat{\sqrt{2}\\0}\qquad
			\Proj_S\vec v_2=\mat{1\\1}\qquad
			\Proj_S\vec v_3=\frac{\sqrt{2}\vec v_3}{\|\vec v_3\|}\qquad
			\Proj_S\vec v_4=\frac{\sqrt{2}\vec v_4}{\|\vec v_4\|}
		\]

\end{enumerate}

	\item \begin{enumerate}
			\item $f(t) = \sqrt{(t-2)^2+(2t-2)^2} = \sqrt{5t^2-12t+8}$.
			\item $\Proj_\ell \vec r$ must be the multiple of $\vec d$ that is
				closest to $\vec r$. Notice that $f(t)$ is minimized when
				$5t^2-12t+8$ is minimized which happens when $t=6/5$. Therefore
				\[
					\Proj_\ell \vec r = \tfrac{6}{5}\mat{1\\2}.
				\]
			\item $90^\circ$.
			\item They are the same thing! To find the closest point on a line
				to a point, you draw a perpendicular to the line passing through the point.
				The intersection with the line is the closest point and is therefore the projection.
				However, the line in this case is given by
				all multiples of $\vec d$, so geometrically, 
				$\Comp_{\vec d}\vec r$ is defined in the exact same way as $\Proj_\ell \vec r$ (in this case).
	\end{enumerate}
		\item  \begin{enumerate}
				\item $\mat{x\\y}\mapsto \mat{x\\0}$.
			\item Let $\vec d=\mat{2\\1}$ and $\vec x=\mat{x\\y}$. 
				Then $\vec x\mapsto \Comp_{\vec d}\vec x = \frac{2x+y}{5}\mat{2\\1}$.
			\item Let $\vec s=\mat{-1\\2}$ be the direction of the sunlight and let $\vec x$ be as before. 
				We are looking for when $\vec x+t\vec s$ has a zero $y$-coordinate, which happens when
				$t=-y/2$. Therefore
				\[
					\mat{x\\y}\mapsto \mat{x\\y}+\tfrac{-y}{2}\mat{-1\\2} = \mat{x+y/2\\0}.
				\]
		\end{enumerate}

		\end{enumerate}
	


	\end{solutions}
	\begin{instructions}
\subsection*{Learning Objectives}
	Students need to be able to\ldots
	\begin{itemize}
		\item Compute projections from the definition without memorizing a formula
		\item Exploit right angles to compute projections that would otherwise be
			difficult
	\end{itemize}

\subsection*{Context}
	Students have covered projections and components in class two weeks ago. Last week they started
		working with matrices/matrix multiplication.

	In this class, $\Proj_X\vec v$ is defined as the closest vector in $X$ to $\vec v$. It is \emph{not}
		defined in terms of a formula. Also, to avoid confusion, we \emph{don't write}
		$\Proj_{\vec u}\vec v$, since this is easily confused with projection onto a singleton.
		Instead, the ``component of $\vec v$ in the direction $\vec u$'', written $\Comp_{\vec u}\vec v$
		is defined to be the vector in the direction of $\vec u$ so that $\vec v-\Comp_{\vec u}\vec v$
		is orthogonal to $\vec u$. There is a formula for this operation (unlike projections), and they
		should all know it since they've used it on their homework.

\subsection*{What to Do}
	Introduce the learning objectives for the day's tutorial. Explain that, like most things in life,
		there is no algorithm to compute projections---the only way to do it is to understand
		the idea. Since we're practicing learning and understanding non-algorithmic tasks in this
		class, projections provide the perfect place to practice.

	Have students pair up and write the definition of projection. Again, make them
		write it. These definitions will show up on the midterms and many will
		write them wrong. Again, you might have some groups come up to the board and
		write their definitions and then have a short class discussion on whether they are right.
		This definition will be easier for them since there aren't any quantifiers.

		After everyone is on the same page with the definition, have them continue on \#2.
		The point of \#2 is to have them cement in their minds the link between projections
		and orthogonality. Don't give this point away too readily--it's best if they discover it themselves, so
		they \emph{own} it.

		After most of finished \#2, have a class discussion and repeat with \#3. Remember, your goal
		is not to get through problems. This tutorial has \emph{way} more problems than the students
		will be able to get through.

		7 minutes before the end of class, pick a problem that most students have started working on
		to do as a wrap-up.

\subsection*{Notes}
	\begin{itemize}
		\item The definition of $\Proj_X\vec v$ is written mostly with words instead of with
			set notation. This might throw some students off---they won't be able to tell that it's precise.
		\item Projections relate to orthogonality, but not always. \#2 is designed to tease this out---that
			for non-smooth shapes projections might not relate to orthogonality.
		\item With our definition, $\Proj_X\vec v$ might not be unique and so might not be well defined. We will
			never try to trick a student by giving them a non-unique projection. If a student asks about
			this, you can tell them that on tests the projection will always be unique.
		\item For \#3, student \emph{should} know how to minimize a parabola, but many won't. However, almost all
			have taken calculus in high school, so feel free to use basic calculus to answer that question.
		\item You won't get to \#4, but if students are working on it, make sure they draw a picture. It's very
			hard to do in one's head.
	\end{itemize}

	\end{instructions}

\end{document}
